\documentclass[a4paper,12pt]{article}

	\usepackage{xcolor}
	\usepackage{listings}	\lstset{language=[LaTeX]TeX,basicstyle=\ttfamily,texcsstyle=*\color{blue},identifierstyle=\color{brown},commentstyle=\color{gray}\itshape,escapechar=!,moretexcs={}}
	\usepackage[utf8]{inputenc}
	\usepackage{hyperref} 

	\usepackage{Raf_Notations_Unites}



	\newcommand{\rac}{({\color{red}Raccourci})}
	\newcommand{\ren}{({\color{blue}Renommé})}

\begin{document}


	\begin{center}
		\hrule{\Large Notations d'unités}\\\hrule
	\end{center}

	
	Ce package propose des unités (de la physique ou autre),
	écrites un peu plus petites que la police normale,
	et gérant automatiquement les espaces avec les nombres qu'ils suivent.
	
	\tableofcontents
	
	
	
	\section{Packages requis}
	%-------------------------------------

	\begin{itemize}
		\item \href{http://www.ctan.org/pkg/ifthen}{\textbf{ifthen}} : Package permettant une compilation à choix multiple,
		\item \href{http://www.ams.org/publications/authors/tex/amslatex}{\textbf{amsmath}} : Notations mathématiques (Ici : pour utiliser la fonction \verb!\text! dans les formules)
	\end{itemize}

	\section{Appel du package}
	%-------------------------------------

	Le package est appelé en début de document par la commande :
	\begin{verbatim}
\usepackage{Raf_Notations_Unites}
	\end{verbatim}

	Par défaut, ce package utilise un certain nombre de notations raccourcies, susceptibles de rentrer en conflit avec d'autre package (mais tellement plus rapide à taper !).
	De plus, certaines commandes ont été rebaptisée.
	Ces raccourcis et renommages seront cités (\rac\ ou \ren) dans les tableaux suivants.
	Pour ne pas créer ces raccourcis/renommage, il faut rentre l'option \verb!noRaccourci! à l'appel du package.

	\begin{verbatim}
usepackage[noRaccourci]{Raf_Notations_Unites}
	\end{verbatim}


	
	\section{Fonction de base}
	%-----------------------------------
	\noindent
	\begin{tabular}{|p{0.35\linewidth}|p{0.3\linewidth}|p{0.3\linewidth}|}
		\hline
			\textbf{Commandes}&\textbf{Rendus}&\textbf{Commentaires}
		\\\hline\hline
			\verb!\unite{raf}!			&	\unite{raf}			&	Une unité (générique...)
		\\\hline
			\verb!\unite[-3]{raf}!			&	\unite[-3]{raf}		&	Puissance d'une unité
		\\\hline
			\verb!\unite{\text{raf}}!		&	\unite{\text{raf}}		&	Avec police textuelle
		\\\hline
			\verb!12\micron!		&	12\micron		&	Un exemple d'unité sur ces bases...
		\\\hline
			\verb!12\micron[-2]!		&	12\micron[-2]		&	...avec une puissance
		\\\hline
			\verb!12\micron\volt/\ampere!	&	12\micron\volt/\ampere	&	Chaque unité est précédée d'une espace, sauf après un divisé ``$/$''.
		\\\hline
	\end{tabular}


	
	\section{Sans unité (autre que les angles)}
	%---------------------------------------------
	\noindent
	\begin{tabular}{|p{0.35\linewidth}|p{0.3\linewidth}|p{0.3\linewidth}|}
		\hline
 			\textbf{Commandes}&\textbf{Rendus}&\textbf{Commentaires}
 		\\\hline\hline
 			\verb!\dent!	& 	\dent	& Nombre de dents (singulier)	\\
 		\hline
			\verb!\dents!		& 	\dents	&Nombre de dents (pluriel)	\rac\\
		\hline
	\end{tabular}
	
	\section{Longueurs}
	%-----------------------------------
	\noindent
	\begin{tabular}{|p{0.35\linewidth}|p{0.3\linewidth}|p{0.3\linewidth}|}
		\hline
 			\textbf{Commandes}&\textbf{Rendus}&\textbf{Commentaires}
 		\\\hline\hline
 			\verb!\micrometre!	& 	\micrometre	&	\\
 		\hline
			\verb!\micron!		& 	\micron	&	\rac\\
		\hline
			\verb!\millimetre!	& 	\millimetre	&	\\
		\hline
			\verb!\mm!		& 	\mm	&	\rac\\
		\hline
			\verb!\centimetre!	& 	\centimetre	&	\\
		\hline
			\verb!\cm!		& 	\cm	&	\rac\\
		\hline
			\verb!\decimetre!	& 	\decimetre	&	\\
		\hline
			\verb!\dm!		& 	\dm	&	\rac\\
		\hline
			\verb!\metre!		& 	\metre		&	\\
		\hline
			\verb!\m!		& 	\m	&	\rac\\
		\hline
			\verb!\decametre!	& 	\decametre	&	\\
		\hline
			\verb!\dam!		& 	\dam	&	\rac\\
		\hline
			\verb!\hectometre!	& 	\hectometre	&	\\
		\hline
			\verb!\hhm!		& 	\hhm	&	\rac\\
		\hline
			\verb!\kilometre!	& 	\kilometre	&	\\
		\hline
			\verb!\km!		& 	\km	&	\rac\\
		\hline
	\end{tabular}
	
	
	\section{Volumes}
	%-----------------------------------
	\noindent
	\begin{tabular}{|p{0.35\linewidth}|p{0.3\linewidth}|p{0.3\linewidth}|}
		\hline
 			\textbf{Commandes}&\textbf{Rendus}&\textbf{Commentaires}
 		\\\hline\hline
 			\verb!\millilitre!	& 	\millilitre	&	\\
 		\hline
			\verb!\mL!		& 	\mL		&	\rac\\
		\hline
			\verb!\centilitre!	& 	\centilitre	&	\\
		\hline
			\verb!\cL!		& 	\cL		&	\rac\\
		\hline
			\verb!\decilitre!	& 	\decilitre	&	\\
		\hline
			\verb!\dL!		& 	\dL		&	\rac\\
		\hline
			\verb!\litre!		& 	\litre		&	\\
		\hline
			\verb!\Ltr!		& 	\Ltr		&	\rac (Le \verb!L! est déjà utilisé ailleurs)\\
		\hline
			\verb!\decalitre!	& 	\decalitre	&	\\
		\hline
			\verb!\daL!		& 	\daL		&	\rac\\
		\hline
			\verb!\hectolitre!	& 	\hectolitre	&	\\
		\hline
			\verb!\hL!		& 	\hL		&	\rac\\
		\hline
			\verb!\kilolitre!	& 	\kilolitre	&	\\
		\hline
			\verb!\kL!		& 	\kL		&	\\
		\hline
	\end{tabular}
	
	
	\section{Forces}
	%------------------------------------------
	
	
	\noindent
	\begin{tabular}{|p{0.35\linewidth}|p{0.3\linewidth}|p{0.3\linewidth}|}
		\hline
 			\textbf{Commandes}&\textbf{Rendus}&\textbf{Commentaires}
 		\\\hline\hline
 			\verb!\millinewton!	& 	\millinewton	&	\\
 		\hline
			\verb!\mN!		& 	\mN		&	\rac\\
		\hline
			\verb!\centinewton!	& 	\centinewton	&	\\
		\hline
			\verb!\cN!		& 	\cN		&	\rac\\
		\hline
			\verb!\decinewton!	& 	\decinewton	&	\\
		\hline
			\verb!\dN!		& 	\dN		&	\rac\\
		\hline
			\verb!\newton!		& 	\newton		&	\\
		\hline
			\verb!\N!		& 	\N		&	\rac\\
		\hline
			\verb!\decanewton!	& 	\decanewton	&	\\
		\hline
			\verb!\daN!		& 	\daN		&	\rac\\
		\hline
			\verb!\hectonewton!	& 	\hectonewton	&	\\
		\hline
			\verb!\hN!		& 	\hN		&	\rac\\
		\hline
			\verb!\kilonewton!	& 	\kilonewton	&	\\
		\hline
			\verb!\kN!		& 	\kN		&	\rac\\
		\hline
	\end{tabular}

	
	
	\section{Moments de force}
	%------------------------------------------
	
	
	\noindent
	\begin{tabular}{|p{0.35\linewidth}|p{0.3\linewidth}|p{0.3\linewidth}|}
		\hline
 			\textbf{Commandes}&\textbf{Rendus}&\textbf{Commentaires}
 		\\\hline\hline
			\verb!\newtonMetre!	& 	\newtonMetre	&	\\
		\hline
			\verb!\Nm!		& 	\Nm		&	\rac\\
		\hline
			\verb!\newtonMillimetre!& 	\newtonMillimetre	&	\\
		\hline
			\verb!\Nmm!		& 	\Nmm		&	\rac\\
		\hline
			\verb!\milliNewtonMetre!& 	\milliNewtonMetre		&	\\
		\hline
			\verb!\mNm!		& 	\mNm		&	\rac\\
		\hline
	\end{tabular}
		
	
	\section{Pression}
	%------------------------------------------
	
	\noindent
	\begin{tabular}{|p{0.35\linewidth}|p{0.3\linewidth}|p{0.3\linewidth}|}
		\hline
 			\textbf{Commandes}&\textbf{Rendus}&\textbf{Commentaires}
 		\\\hline\hline
			\verb!\millipascal!	& 	\millipascal	&	\\
		\hline
			\verb!\centipascal!	& 	\centipascal	&	\\
		\hline
			\verb!\decipascal!	& 	\decipascal	&	\\
		\hline
			\verb!\pascal!		& 	\pascal	&	\\
		\hline
			\verb!\Pa!		& 	\Pa	&	\rac\\
		\hline
			\verb!\decapascal!	& 	\decapascal	&	\\
		\hline
			\verb!\hectopascal!	& 	\hectopascal	&	\\
		\hline
			\verb!\hPa!		& 	\hPa		&	\rac\\
		\hline
			\verb!\kilopascal!	& 	\kilopascal	&	\\
		\hline
			\verb!\kPa!	& 	\kPa	&	\rac\\
		\hline
			\verb!\megapascal!	& 	\megapascal	&	\\
		\hline
			\verb!\MPa!		& 	\MPa		&	\rac\\
		\hline
			\verb!\gigapascal!	& 	\gigapascal	&	\\
		\hline
			\verb!\GPa!		& 	\GPa		&	\rac\\
		\hline
			\verb!\baros!		& 	\baros		&	\\
		\hline
			\verb!\br!		& 	\br		&	\rac\\
		\hline
	\end{tabular}
	
	
	
	
	\section{Temps}
	%------------------------------------------
	
	\noindent
	\begin{tabular}{|p{0.35\linewidth}|p{0.3\linewidth}|p{0.3\linewidth}|}
		\hline
 			\textbf{Commandes}&\textbf{Rendus}&\textbf{Commentaires}
 		\\\hline\hline
			\verb!\milliseconde!	& 	\milliseconde	&	\\
		\hline
			\verb!\ms!		& 	\ms	&	\rac\\
		\hline
			\verb!\seconde!		& 	\seconde	&	\\
		\hline
			\verb!\sec!		& 	\sec	&	\rac\\
		\hline
			\verb!\minute!		& 	\minute	&	\\
		\hline
			\verb!\mn!		& 	\mn	&	\rac\\
		\hline
			\verb!\heure!		& 	\heure	&	\\
		\hline
			\verb!\hr!		& 	\hr	&	\rac\\
		\hline
			\verb!\annee!		& 	\annee		&	\\
		\hline
			\verb!\an!		& 	\an	&	\rac\\
		\hline
			\verb!\annees!		& 	\annees	&	\\
		\hline
			\verb!\ans!		& 	\ans		&	\rac\\
		\hline
	\end{tabular}
	
	
		
	
	\section{Fréquences}
	%------------------------------------------
	
	\noindent
	\begin{tabular}{|p{0.35\linewidth}|p{0.3\linewidth}|p{0.3\linewidth}|}
		\hline
 			\textbf{Commandes}&\textbf{Rendus}&\textbf{Commentaires}
 		\\\hline\hline
			\verb!\millihertz!	& 	\millihertz	&	\\
		\hline
			\verb!\mHz!		& 	\mHz	&	\rac\\
		\hline
			\verb!\centihertz!	& 	\centihertz	&	\\
		\hline
			\verb!\cHz!		& 	\cHz	&	\rac\\
		\hline
			\verb!\decihertz!	& 	\decihertz	&	\\
		\hline
			\verb!\dHz!		& 	\dHz	&	\rac\\
		\hline
			\verb!\hertz!		& 	\hertz	&	\\
		\hline
			\verb!\Hz!		& 	\Hz	&	\rac\\
		\hline
			\verb!\decahertz!	& 	\decahertz		&	\\
		\hline
			\verb!\daHz!		& 	\daHz	&	\rac\\
		\hline
			\verb!\hectohertz!	& 	\hectohertz	&	\\
		\hline
			\verb!\hHz!		& 	\hHz		&	\rac\\
		\hline
			\verb!\kilohertz!	& 	\kilohertz	&	\\
		\hline
			\verb!\kHz!		& 	\kHz		&	\rac\\
		\hline
			\verb!\megahertz!	& 	\megahertz	&	\\
		\hline
			\verb!\MHz!		& 	\MHz		&	\rac\\
		\hline
			\verb!\gigahertz!	& 	\gigahertz	&	\\
		\hline
			\verb!\GHz!		& 	\GHz		&	\rac\\
		\hline
	\end{tabular}

	\section{Énergie}
	%------------------------------------------

	
	\noindent
	\begin{tabular}{|p{0.35\linewidth}|p{0.3\linewidth}|p{0.3\linewidth}|}
		\hline
 			\textbf{Commandes}&\textbf{Rendus}&\textbf{Commentaires}
 		\\\hline\hline
			\verb!\millijoule!	& 	\millijoule	&	\\
		\hline
			\verb!\centijoule!	& 	\centijoule	&	\\
		\hline
			\verb!\decijoule!	& 	\decijoule	&	\\
		\hline
			\verb!\joule!		& 	\joule		&	\\
		\hline
			\verb!\J!		& 	\J		&	\rac\\
		\hline
			\verb!\decajoule!	& 	\decajoule	&	\\
		\hline
			\verb!\hectojoule!	& 	\hectojoule	&	\\
		\hline
			\verb!\kilojoule!	& 	\kilojoule	&	\\
		\hline
			\verb!\kJ!		& 	\kJ		&	\rac\\
		\hline
			\verb!\megajoule!	& 	\megajoule	&	\\
		\hline
			\verb!\MJ!		& 	\MJ		&	\rac\\
		\hline
			\verb!\wattheure!		& 	\wattheure		&	\\
		\hline
			\verb!\Wh!		& 	\Wh		&	\\
		\hline
			\verb!\kilowattheure!		& 	\kilowattheure		&	\\
		\hline
			\verb!\kWh!		& 	\kWh		&	\\
		\hline
	\end{tabular}
	
	

	\section{Puissance}
	%------------------------------------------
	

	\noindent
	\begin{tabular}{|p{0.35\linewidth}|p{0.3\linewidth}|p{0.3\linewidth}|}
		\hline
 			\textbf{Commandes}&\textbf{Rendus}&\textbf{Commentaires}
 		\\\hline\hline
			\verb!\milliwatt!	& 	\milliwatt	&	\\
		\hline
			\verb!\centiwatt!	& 	\centiwatt	&	\\
		\hline
			\verb!\deciwatt!	& 	\deciwatt	&	\\
		\hline
			\verb!\watt!		& 	\watt		&	\\
		\hline
			\verb!\W!		& 	\W		&	\rac\\
		\hline
			\verb!\decawatt!	& 	\decawatt	&	\\
		\hline
			\verb!\hectowatt!	& 	\hectowatt	&	\\
		\hline
			\verb!\kilowatt!	& 	\kilowatt	&	\\
		\hline
			\verb!\kW!		& 	\kW		&	\rac\\
		\hline
			\verb!\megawatt!	& 	\megawatt	&	\\
		\hline
			\verb!\MW!		& 	\MW		&	\rac\\
		\hline
			\verb!\cheval!		& 	\cheval		&	\\
%		\hline
%			\verb!\VA!		& 	\VA		&	Volts Ampère\\
		\hline
			\verb!\VAR!		& 	\VAR		&	Volts Ampère Réactifs\\
		\hline
	\end{tabular}
	
	
	\section{Intensité lumineuse}
	%------------------------------------------

	\noindent
	\begin{tabular}{|p{0.35\linewidth}|p{0.3\linewidth}|p{0.3\linewidth}|}
		\hline
 			\textbf{Commandes}&\textbf{Rendus}&\textbf{Commentaires}
 		\\\hline\hline
			\verb!\lumen!	& 	\lumen	&	\\
		\hline
			\verb!\lm!	& 	\lm	&	\rac\\
		\hline
			\verb!\candela!	& 	\candela	&	\\
		\hline
			\verb!\cd!	& 	\cd		&	\rac\\
		\hline
	\end{tabular}
	
	
	\section{Masse}
	%------------------------------------------
	
	\noindent
	\begin{tabular}{|p{0.35\linewidth}|p{0.3\linewidth}|p{0.3\linewidth}|}
		\hline
 			\textbf{Commandes}&\textbf{Rendus}&\textbf{Commentaires}
 		\\\hline\hline
			\verb!\milligramme!	& 	\milligramme	&	\\
		\hline
			\verb!\centigramme!	& 	\centigramme	&	\\
		\hline
			\verb!\decigramme!	& 	\decigramme	&	\\
		\hline
			\verb!\gramme!		& 	\gramme		&	\\
		\hline
			\verb!\g!		& 	\g		&	\rac\\
		\hline
			\verb!\decagramme!	& 	\decagramme	&	\\
		\hline
			\verb!\hectogramme!	& 	\hectogramme		&	\\
		\hline
			\verb!\kilogramme!	& 	\kilogramme		&	\\
		\hline
			\verb!\kg!		& 	\kg		&	\rac\\
		\hline
			\verb!\tonne!		& 	\tonne		&	\\
		\hline
	\end{tabular}
	
	
	\section{Angles}
	%------------------------------------------
	
	\noindent
	\begin{tabular}{|p{0.35\linewidth}|p{0.3\linewidth}|p{0.3\linewidth}|}
		\hline
 			\textbf{Commandes}&\textbf{Rendus}&\textbf{Commentaires}
 		\\\hline\hline
			\verb!\degree!	& 	\degree	&	\\
		\hline
			\verb!\deg!	& 	\deg	&	\rac\\
		\hline
			\verb!\radian!	& 	\radian	&	\\
		\hline
			\verb!\rad!	& 	\rad	&	\rac\\
		\hline
			\verb!\tour!	& 	\tour	&	\\
		\hline
			\verb!\tr!	& 	\tr	&	\rac\\
		\hline
	\end{tabular}
	
	\section{Électricité (Général -- encore partiel)}
	%------------------------------------------
	
	
	\noindent
	\begin{tabular}{|p{0.35\linewidth}|p{0.3\linewidth}|p{0.3\linewidth}|}
		\hline
 			\textbf{Commandes}&\textbf{Rendus}&\textbf{Commentaires}
 		\\\hline\hline
			\verb!\volt!	& 	\volt	&	\\
		\hline
			\verb!\V!	& 	\V	&	\rac\\
		\hline
			\verb!\ampere!	& 	\ampere	&	\\
		\hline
			\verb!\A!	& 	\A	&	\rac\\
		\hline
			\verb!\ohm!	& 	\ohm	&	\\
		\hline
			\verb!\kiloohm!	& 	\kiloohm	&	\\
		\hline
			\verb!\megaohm!	& 	\megaohm	&	\\
		\hline
			\verb!\farad!	& 	\farad	&	\\
		\hline
			\verb!\millifarad!	& 	\millifarad	&	\\
		\hline
			\verb!\mF!	& 	\mF	&	\\
		\hline
			\verb!\microfarad!	& 	\microfarad	&	\\
		\hline
			\verb!\microF!	& 	\microF	&	\\
		\hline
			\verb!\picofarad!	& 	\picofarad	&	\\
		\hline
			\verb!\pF!	& 	\pF	&	\\
		\hline
			\verb!\nanofarad!	& 	\nanofarad	&	\\
		\hline
			\verb!\nF!	& 	\nF	&	\\
		\hline
			\verb!\ampereheure!	& 	\ampereheure	&	\\
		\hline
			\verb!\Ah!	& 	\Ah	&	\rac\\
		\hline
			\verb!\milliampereheure!	& 	\milliampereheure	&	\\
		\hline
			\verb!\mAh!	& 	\mAh	&	\rac\\
		\hline
	\end{tabular}
	
	
	
	\section{Inductance}
	%------------------------------------------
	
	
	\noindent
	\begin{tabular}{|p{0.35\linewidth}|p{0.3\linewidth}|p{0.3\linewidth}|}
		\hline
 			\textbf{Commandes}&\textbf{Rendus}&\textbf{Commentaires}
 		\\\hline\hline
			\verb!\microhenry!	& 	\microhenry	&	\\
		\hline
			\verb!\microH!	& 	\microH	&	\\
		\hline
			\verb!\millihenry!	& 	\millihenry	&	\\
		\hline
			\verb!\mH!		& 	\mH		&	\rac\\
		\hline
			\verb!\centihenry!	& 	\centihenry	&	\\
		\hline
			\verb!\cH!		& 	\cH		&	\rac\\
		\hline
			\verb!\decihenry!	& 	\decihenry	&	\\
		\hline
			\verb!\dH!		& 	\dH		&	\rac\\
		\hline
			\verb!\henry!		& 	\henry		&	\\
		\hline
			\verb!\H!		& 	\H		&	\ren \rac\\
		\hline
			\verb!\decahenry!	& 	\decahenry	&	\\
		\hline
			\verb!\daH!		& 	\daH		&	\rac\\
		\hline
			\verb!\hectohenry!	& 	\hectohenry		&	\\
		\hline
			\verb!\hH!		& 	\hH		&	\rac\\
		\hline
			\verb!\kilohenry!	& 	\kilohenry		&	\\
		\hline
			\verb!\kH!		& 	\kH		&	\rac\\
		\hline
	\end{tabular}
	

	\section{Dureté matériaux}
	%------------------------------------------
	
	
	\noindent
	\begin{tabular}{|p{0.35\linewidth}|p{0.3\linewidth}|p{0.3\linewidth}|}
		\hline
 			\textbf{Commandes}&\textbf{Rendus}&\textbf{Commentaires}
 		\\\hline\hline
			\verb!\dureteHB!	& 	\dureteHB	&	Dureté Brinell\\
		\hline
			\verb!\HB!		& 	\HB		&	\rac\\
		\hline
			\verb!\dureteHBS!	& 	\dureteHBS	&	Dureté Brinell avec bille acier\\
		\hline
			\verb!\HBS!		& 	\HBS		&	\rac\\
		\hline
			\verb!\dureteHBW!	& 	\dureteHBW	&	Dureté Brinell avec bille en carbure de tungstène\\
		\hline
			\verb!\HBW!		& 	\HBW		&	\rac\\
		\hline
			\verb!\dureteHRB!	& 	\dureteHRB	&	Dureté Rockwell	\\
		\hline
			\verb!\HRB!		& 	\HRB		&	\ren \rac\\
		\hline
			\verb!\dureteHRC!	& 	\dureteHRC	&	Dureté Rockwell\\
		\hline
			\verb!\HRC!		& 	\HRC		&	\rac\\
		\hline
			\verb!\dureteHV!	& 	\dureteHV	&	Dureté Vikers\\
		\hline
			\verb!\HV!		& 	\HV		&	\rac\\
		\hline
			\verb!\dureteHSh!	& 	\dureteHSh	&	Dureté Shore\\
		\hline
			\verb!\HSh!		& 	\HSh		&	\rac\\
		\hline
			\verb!\dureteKCU!	& 	\dureteKCU	&	Coefficient résilience en U\\
		\hline
			\verb!\KCU!		& 	\KCU		&	\rac\\
		\hline
			\verb!\dureteKCV!	& 	\dureteKCV	&	Coefficient résilience en V\\
		\hline
			\verb!\KCV!		& 	\KCV		&	\rac\\
		\hline
	\end{tabular}
	
	
	
	\section{Acoustique}
	%------------------------------------------
	
	\noindent
	\begin{tabular}{|p{0.35\linewidth}|p{0.3\linewidth}|p{0.3\linewidth}|}
		\hline
 			\textbf{Commandes}&\textbf{Rendus}&\textbf{Commentaires}
 		\\\hline\hline
			\verb!\decibell!	& 	\decibell	&	Décibell\\
		\hline
			\verb!\dB!		& 	\dB		&	Décibell \rac\\
		\hline
			\verb!\decade!		& 	\decade		&	Décade (logarithme décimal)\\
		\hline
			\verb!\dec!		& 	\dec		&	Décade \rac\\
		\hline
			\verb!\decades!		& 	\decades	&	Décade au pluriel\\
		\hline
			\verb!\decs!		& 	\decs		&	Décade au pluriel \rac\\
		\hline
	\end{tabular}

	
	\section{Température}
	%------------------------------------------
	
	\noindent
	\begin{tabular}{|p{0.35\linewidth}|p{0.3\linewidth}|p{0.3\linewidth}|}
		\hline
 			\textbf{Commandes}&\textbf{Rendus}&\textbf{Commentaires}
 		\\\hline\hline
			\verb!\degreCelcius!	& 	\degreCelcius	&	Degrés Celcius\\
		\hline
			\verb!\degC!		& 	\degC		&	Degrés Celcius \rac\\
		\hline
			\verb!\kelvin!		& 	\kelvin		&	Kelvin\\
		\hline
			\verb!\K!		& 	\K		&	Kelvin \rac\\
		\hline
	\end{tabular}
	
	\section{Informatique}
	%------------------------------------------
	
	\noindent
	\begin{tabular}{|p{0.35\linewidth}|p{0.3\linewidth}|p{0.3\linewidth}|}
		\hline
 			\textbf{Commandes}&\textbf{Rendus}&\textbf{Commentaires}
 		\\\hline\hline
			\verb!\octet!		& 	\octet		&	Octet\\
		\hline
			\verb!\kilooctet!	& 	\kilooctet	&	Kilooctet ($10^3\octet$)\\
		\hline
			\verb!\ko!		& 	\ko		&	Kilooctet \rac\\
		\hline
			\verb!\megaoctet!	& 	\megaoctet	&	Mégaoctet ($10^6\octet$)\\
		\hline
			\verb!\Mo!		& 	\Mo		&	Mégaoctet \rac\\
		\hline
			\verb!\gigaoctet!	& 	\gigaoctet	&	Gigaoctet ($10^9\octet$)\\
		\hline
			\verb!\Go!		& 	\Go		&	Gigaoctet \rac\\
		\hline
			\verb!\teraoctet!	& 	\teraoctet	&	Téraoctet ($10^{12}\octet$)\\
		\hline
			\verb!\To!		& 	\To		&	Téraoctet \rac\\
		\hline
	\end{tabular}
	
	\noindent
	\begin{tabular}{|p{0.35\linewidth}|p{0.3\linewidth}|p{0.3\linewidth}|}
		\hline
 			\textbf{Commandes}&\textbf{Rendus}&\textbf{Commentaires}
 		\\\hline\hline
			\verb!\kibioctet!	& 	\kibioctet	&	Kibi-octet ($2^{10}\octet=1024\octet$)\\
		\hline
			\verb!\kio!		& 	\kio		&	Kibi-octet \rac\\
		\hline
			\verb!\mebioctet!	& 	\mebioctet	&	Mébi-octet ($2^{20}\octet=1024\kibioctet$)\\
		\hline
			\verb!\Mio!		& 	\Mio		&	Mébi-octet \rac\\
		\hline
			\verb!\gibioctet!	& 	\gibioctet	&	Gibi-octet ($2^{30}\octet=1024\mebioctet$)\\
		\hline
			\verb!\Gio!		& 	\Gio		&	Gibi-octet \rac\\
		\hline
			\verb!\tebioctet!	& 	\tebioctet	&	Tébi-octet ($2^{40}\octet=1024\gibioctet$)\\
		\hline
			\verb!\Tio!		& 	\Tio		&	Tébi-octet \rac\\
		\hline
	\end{tabular}

	\noindent
	\begin{tabular}{|p{0.35\linewidth}|p{0.3\linewidth}|p{0.3\linewidth}|}
		\hline
 			\textbf{Commandes}&\textbf{Rendus}&\textbf{Commentaires}
 		\\\hline\hline
			\verb!\bit!		& 	\bit		&	bit\\
		\hline
			\verb!\kilobit!		& 	\kilobit	&	Kilo-bit ($10^{3}\bit$)\\
		\hline
			\verb!\kbit!		& 	\kbit		&	Kilo-bit \rac\\
		\hline
			\verb!\megabit!		& 	\megabit	&	Méga-bit ($10^{6}\bit$)\\
		\hline
			\verb!\Mbit!		& 	\Mbit		&	Méga-bit \rac\\
		\hline
			\verb!\gigabit!		& 	\gigabit	&	Giga-bit ($10^{9}\bit$)\\
		\hline
			\verb!\Gbit!		& 	\Gbit		&	Giga-bit \rac\\
		\hline
			\verb!\terabit!		& 	\terabit	&	Téra-bit ($10^{12}\bit$)\\
		\hline
			\verb!\Tbit!		& 	\Tbit		&	Téra-bit \rac\\
		\hline
	\end{tabular}

\end{document}
