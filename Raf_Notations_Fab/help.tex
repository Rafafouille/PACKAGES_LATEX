\documentclass[a4paper,12pt]{article}

	\usepackage{xcolor}
	\usepackage{listings}	\lstset{language=[LaTeX]TeX,basicstyle=\ttfamily,texcsstyle=*\color{blue},identifierstyle=\color{brown},commentstyle=\color{gray}\itshape,escapechar=!,moretexcs={}}
	\usepackage[latin1]{inputenc} 
	\usepackage{hyperref}

	\usepackage{Raf_Notations_Fab}




\begin{document}

	\begin{center}
		\hrule{\Large Fabrication}\\\hrule
	\end{center}

	\section{Packages requis}
	%-------------------------------------

		\begin{itemize}
			\item \href{http://www.ctan.org/pkg/ifthen}{\textbf{ifthen}} : Package pour faire des compilations conditionnelles (if...then...else....)
		\end{itemize}

	\section{Appel du package}
	%-------------------------------------
	
Le package est appel� en d�but de document par la commande :

\begin{verbatim}
\usepackage{Raf_Notations_Fab}
\end{verbatim}

	%	\textbf{Note :} Oui, je sais... l'option \verb!raccourcis! est un peu cr�tine dans le sens o� il n'y a que des raccourcis dans ce package.
	%							C'est simplement pour avoir une uniformit� dans tous mes packages.


	\section{Mouvements de coup}
	%-----------------------------
		\noindent
		\begin{tabular}{|p{0.4\linewidth}|p{0.2\linewidth}|p{0.4\linewidth}|}
			\hline
				\textbf{Commandes}&\textbf{Rendus}&\textbf{Commentaires}
			\\\hline\hline
				\verb!\VCoupe!		&	\VCoupe		&	Vitesse de coupe.
			\\\hline
				\verb!\Vc!		&	\Vc		&	raccourci de \verb!\VCoupe!
			\\\hline
				\verb!\VAvance!		&	\VAvance		&	Vitesse d'avance.
			\\\hline
				\verb!\Vf!		&	\Vf		&	Raccourci de \verb!\VAvance!.
			\\\hline
				\verb!\VEco!		&	\VEco		&	Vitesse �conomique de coupe
			\\\hline
				\verb!\avance!		&	\avance		&	Avance de l'outil.
			\\\hline
				\verb!\pPasse!		&	\pPasse		&	Profondeur de passe.
			\\\hline
		\end{tabular}

	\section{Angles de coupe}
	%-----------------------------
		\noindent
		\begin{tabular}{|p{0.4\linewidth}|p{0.2\linewidth}|p{0.4\linewidth}|}
			\hline
				\textbf{Commandes}&\textbf{Rendus}&\textbf{Commentaires}
			\\\hline\hline
				\verb!\aCoupe!		&	\aCoupe		&	Angle de coupe.
			\\\hline
				\verb!\aDepouille!	&	\aDepouille	&	Angle de d�pouille
			\\\hline
				\verb!\aTaillant!	&	\aTaillant	&	Angle de taillant ?
			\\\hline
				\verb!\aInclinaison!	&	\aInclinaison	&	Angle d'inclinaison.
			\\\hline
				\verb!\aInc!		&	\aInc		&	Raccourci de \verb!\aInclinaison!.
			\\\hline
				\verb!\aDirectionArete!	&	\aDirectionArete	&	Angle de direction d'ar�te.
			\\\hline
				\verb!\aDir!		&	\aDir		&	Raccourci de \verb!\aDirectionArete!.
			\\\hline
				\verb!\Kr!		&	\Kr		&	Raccourci de \verb!\aDirectionArete!.
			\\\hline
				\verb!\aPointe!		&	\aPointe		&	????.
			\\\hline
		\end{tabular}



	\section{Plans de la coupe}
	%-----------------------------
		\noindent
		\begin{tabular}{|p{0.4\linewidth}|p{0.2\linewidth}|p{0.4\linewidth}|}
			\hline
				\textbf{Commandes}&\textbf{Rendus}&\textbf{Commentaires}
			\\\hline\hline
				\verb!\fCoupe!		&	\fCoupe		&	Face de coupe.
			\\\hline
				\verb!\fDepouille!	&	\fDepouille	&	Face de d�pouille
			\\\hline
				\verb!\pRef!		&	\pRef		&	Plan de r�f�rence.
			\\\hline
				\verb!\pTravail!		&	\pTravail	&	Plan de travail conventionnel.
			\\\hline
				\verb!\pArete!		&	\pArete		&	Plan d'ar�te de l'outil.
			\\\hline
				\verb!\pNormal!		&	\pNormal		&	Plan normal d'ar�te de l'outil ?.
			\\\hline
		\end{tabular}

	\section{Autre}
	%-----------------------------
		\noindent
		\begin{tabular}{|p{0.4\linewidth}|p{0.2\linewidth}|p{0.4\linewidth}|}
			\hline
				\textbf{Commandes}&\textbf{Rendus}&\textbf{Commentaires}
			\\\hline\hline
				\verb!\rBec!		&	\rBec		&	Rayon de bec de l'outil.
			\\\hline
				\verb!\usureDepouille!	&	\usureDepouille	&	Usure en d�pouille
			\\\hline
				\verb!\cTotal!	&	\usureDepouille	&	Co�t total
			\\\hline
				\verb!\CT!	&	\usureDepouille	&	Co�t total (raccourci)
			\\\hline
		\end{tabular}

	\section{Efforts et puissances}
	%-----------------------------
		\noindent
		\begin{tabular}{|p{0.4\linewidth}|p{0.2\linewidth}|p{0.4\linewidth}|}
			\hline
				\textbf{Commandes}&\textbf{Rendus}&\textbf{Commentaires}
			\\\hline\hline
				\verb!\FCoupe!		&	\FCoupe		&	Effort de coupe (tangentiel)
			\\\hline
				\verb!\Fc!		&	\Fc		&	Raccourci de \verb!\FCoupe!
			\\\hline
				\verb!\pSpecifique!	&	\pSpecifique	&	Pression sp�cifique de coupe
			\\\hline
				\verb!\Kc!		&	\Kc		&	Raccourci de \verb!pSpecifique!
			\\\hline
				\verb!\PCoupe!		&	\PCoupe		&	Puissance de coupe
			\\\hline
				\verb!\Pc!		&	\Pc		&	Raccourci de \verb!PCoupe!
			\\\hline
		\end{tabular}

\end{document}
