\documentclass[a4paper,10pt]{article}

	\usepackage{xcolor}
	\usepackage{listings}	\lstset{language=[LaTeX]TeX,basicstyle=\ttfamily,texcsstyle=*\color{blue},identifierstyle=\color{brown},commentstyle=\color{gray}\itshape,escapechar=!,moretexcs={}}
	\usepackage[latin1]{inputenc}
	\usepackage{hyperref}


	\usepackage{Raf_Notations_RDM}
	\usepackage[francais]{babel}
	\everymath{\displaystyle}


	\newcommand{\rac}{({\color{red}Raccourci})}
	\newcommand{\ren}{({\color{blue}Renommé})}

\begin{document}





	\begin{center}
		\hrule{\Large Notations de RDM}\\\hrule
	\end{center}

	(Version du 01/08/18)

	\section{Packages requis}
	%-------------------------------------

	\begin{itemize}
		\item \href{http://www.ctan.org/pkg/ifthen}{\textbf{ifthen}} : Package permettant une compilation à choix multiple,
		%\item \href{http://tug.ctan.org/tex-archive/macros/latex/contrib/xargs}{\textbf{xarg}} : Package permettant de créer des commandes à plusieurs arguments optionnels.
		%\item \href{http://www.ams.org/publications/authors/tex/amsfonts}{\textbf{amsfonts}} : Package qui ajoute des polices d'écritures mathématiques.
		%\item \href{http://www.ams.org/publications/authors/tex/amslatex}{\textbf{amsmath}} : Package qui ajoute des fonctions mathématiques non-standards.
		\item \href{http://www.ctan.org/pkg/mathrsfs}{\textbf{mathrsfs}} : Package qui rajoute des polices d'écritures mathématiques.
		%\item \href{http://www.ctan.org/pkg/color}{\textbf{color}} : Package permettant de mettre en couleur du texte, des lignes, etc.
		%\item \href{http://www.ctan.org/pkg/xspace}{\textbf{xspace}} : Package permettant de mettre des espaces après les commandes.
		%\item \href{http://www.ctan.org/pkg/xstring}{\textbf{xstring}} : Package permettant travailler sur les chaînes de caractères (chercher/remplacer, etc.)
		\item \href{http://enseignement.allais.eu/page-latex}{\textbf{Raf\_Notations\_Actions-Meca}} : Package de notations d'actions mécaniques.
	\end{itemize}

	\section{Appel du package}
	%-------------------------------------

	Le package est appelé en début de document par la commande :
	\begin{verbatim}
\usepackage{Raf_Notations_RDM}
	\end{verbatim}

	Par défaut, ce package utilise un certain nombre de notations raccourcies, susceptibles de rentrer en conflit avec d'autre package (mais tellement plus rapide à taper !).
	De plus, certaines commandes ont été rebaptisée.
	Ces raccourcis et renommages seront cités (\rac\ ou \ren) dans les tableaux suivants.
	Pour ne pas créer ces raccourcis/renommage, il faut rentre l'option \verb!noRaccourci! à l'appel du package.

	\begin{verbatim}
usepackage[noRaccourci]{Raf_Notations_RDM}
	\end{verbatim}

	\section{Poutre}
	%-------------------------------------------
	\noindent
	\begin{tabular}{|p{0.35\linewidth}|p{0.3\linewidth}|p{0.3\linewidth}|}
		\hline
			\textbf{Commandes}&\textbf{Rendus}&\textbf{Commentaires}
		\\\hline\hline
			\verb!\PCallig!			&	\PCallig			&	"P" Calligraphié.
		\\\hline
			\verb!\ensemblePoutre!		&	\ensemblePoutre		&	Ensemble de points constituant la poutre.
		\\\hline
			\verb!\eP!			&	\eP			&	Raccourci de \verb!\ensemblePoutre!. \rac
		\\\hline
			\verb!\ensemblePoutreSup!	&	\ensemblePoutreSup	&	Partie de poutre " à droite ".
		\\\hline
			\verb!\ePp!			&	\ePp			&	Raccourci de \verb!\ensemblePoutreSup!. \rac
		\\\hline
			\verb!\ensemblePoutreInf!	&	\ensemblePoutreInf	&	Partie de poutre " à gauche ".
		\\\hline
			\verb!\ePm!			&	\ePm			&	Raccourci de \verb!\ensemblePoutreInf!.
		\\\hline
            \verb!\Sx!          &   \Sx             &   Section de poutre à l'bascisse $x$ (par défaut).
		\\\hline
            \verb!\Sx[2]!          &   \Sx[2]             &   Section de poutre à l'abscisse $2$.
		\\\hline
	\end{tabular}




	\section{Torseur de cohésion}
	%-------------------------------------------
	\noindent
	\begin{tabular}{|p{0.35\linewidth}|p{0.3\linewidth}|p{0.3\linewidth}|}
		\hline
			\textbf{Commandes}&\textbf{Rendus}&\textbf{Commentaires}
		\\\hline\hline
			\verb!\tCohesion!		&	\tCohesion		&	Torseur de cohésion
		\\\hline
			\verb!\composanteTCohesion{N}!		&	\composanteTCohesion{N}		&	Composante du torseur de cohésion
		\\\hline
			\verb!\composanteTCohesion! \verb!{N}[x]!		&	\composanteTCohesion{N}[x]		&	Idem, avec une variable.
		\\\hline
			\verb!\Nt,\Ty,\Tz,\Mt,\Mfy,! \verb!\Mfz!		&	\Nt,\Ty,\Tz,\Mt,\Mfy,\Mfz		&	Composantes du torseur de cohésions (les quatre premiers sont des raccourcis \rac). Attention : \Ty et \Tz existent déjà dans d'autres de mes packages (je ne sais plus lesquels). Les commandes sont écrasées, dans ce cas.
		\\\hline
			\verb!\Nt[x],\Ty[x],\Tz[x]! \verb!,\Mt[x],\Mfy[x],\Mfz[x]!		&	\Nt[x],\Ty[x],\Tz[x],\Mt[x],\Mfy[x],\Mfz[x]		&	Idem avec une variable
		\\\hline
	\end{tabular}


    \section{Coefficients...}
    %---------------------------------------
    
    \noindent
	\begin{tabular}{|p{0.35\linewidth}|p{0.3\linewidth}|p{0.3\linewidth}|}
		\hline
			\textbf{Commandes}&\textbf{Rendus}&\textbf{Commentaires}
		\\\hline\hline
			\verb!\momentQuadratique{A}! \verb!{\vec V}!		&	\momentQuadratique{A}{\vec V}		&	Moment quadratique autour de l'axe $(A,\vec V)$.
		\\\hline
			\verb!\IGz!		&	\IGz		&	Moment quadratique autour de \vz{}.
		\\\hline
			\verb!\IGz[P]!		&	\IGz[P]		&	Choix d'un autre point.
		\\\hline
			\verb!\IGz[P][\vec z_2]!		&	\IGz[P][\vec z_2]		&	Choix d'une autre direction.
		\\\hline
			\verb!\IGy!		&	\IGy		&	Moment quadratique autour de \vy{} (fonctionnement identique à \verb!\IGz!).
		\\\hline
			\verb!\IG!		&	\IG		&	Moment quadratique polaire.
		\\\hline
			\verb!\IG[P]!		&	\IG[P]		&	Moment quadratique polaire en un autre point.
		\\\hline
	\end{tabular}

	\section{Graphe}
	%-------------------------------------------
	\begin{verbatim}
%{Label y}[x_min=0]{x_max}{y_min}{y_max}[scale_x=8][scale_y=0.2]
\begin{grapheRDM}{\Nt[x]}[0]{1.5}{-20}{10}
    \grapheRDMTrace{15*\x-15}[0][0.5]
    \grapheRDMTrace{7.5*\x-11.25}[0.5][1.5]
\end{grapheRDM}
    \end{verbatim}

%{Label y}[x_min=0]{x_max}{y_min}{y_max}[scale_x=8][scale_y=0.2]
\begin{grapheRDM}{\Nt[x]}[0]{1.5}{-20}{10}
    \grapheRDMTrace{15*\x-15}[0][0.5]
    \grapheRDMTrace{7.5*\x-11.25}[0.5][1.5]
\end{grapheRDM}


	\section{Poutre}
	%-------------------------------------------

	Voici une série d'exemples commentés pour voir les différentes macros/
	
	\subsection{Environnement poutre}
	%------------------------------------
	%
        Pour créer une poutre, dont l'origine est à gauche, et de longueur L (10cm dans l'exemple) :

        \begin{verbatim}
\begin{poutre}{10cm}    %On donne la longueur de la poutre en argument
\end{poutre}
        \end{verbatim}

        \begin{poutre}{10cm}
        \end{poutre}
	
    
    \subsection{Liaison encastrement}
	%------------------------------------
	
	\begin{center}
        \begin{tabular}{|c|}
            \hline
            \verb!\encastrement[position][orientation]!
            \\\hline
        \end{tabular}
    \end{center}
	
	\subsubsection{Par défaut :}
	
        \begin{verbatim}
\begin{poutre}{10cm}
    \encastrement   %Encastrement en (0,0) par défaut
\end{poutre}
        \end{verbatim}

\begin{poutre}{10cm}
    \encastrement   %Encastrement en (0,0) par défaut
\end{poutre}


        \subsubsection{Pour positionner (horizontalement) ailleur :}
       
        \begin{verbatim}
\begin{poutre}{10cm}
    \encastrement[5cm]   %Encastrement à 5cm de l'origine
\end{poutre}
        \end{verbatim}

\begin{poutre}{10cm}
    \encastrement[5cm]  %Encastrement à 5cm de l'origine
\end{poutre}
	

        \subsubsection{Pour effectuer une rotation (en degrés) :}
       
        \begin{verbatim}
\begin{poutre}{10cm}
    \encastrement[10cm][180]   %Encastrement à 180°
\end{poutre}
        \end{verbatim}

\begin{poutre}{10cm}
    \encastrement[10cm][180]   %Encastrement à 180°
\end{poutre}

    Attention : pour préciser l'angle, il faut obligatoirement préciser la position (même si c'est zéro).
	

	
	
    
    \subsection{Liaison Articulation}
	%------------------------------------
	
	\begin{center}
        \begin{tabular}{|c|}
            \hline
            \verb!\articulation{position}[orientation][noCercle]!
            \\\hline
        \end{tabular}
    \end{center}
	
	
        \subsubsection{Par défaut}
	
        \begin{verbatim}
\begin{poutre}{10cm}
    \articulation{5cm}   %Articulation 5cm de l'origine
\end{poutre}
        \end{verbatim}

\begin{poutre}{10cm}
    \articulation{5cm}   %Articulation 5cm de l'origine
\end{poutre}


        \subsubsection{Orientation :}
	
        \begin{verbatim}
\begin{poutre}{10cm}
    \articulation{5cm}[30]   %Articulation inclinée à 30°
\end{poutre}
        \end{verbatim}

\begin{poutre}{10cm}
    \articulation{5cm}[30]   %Articulation inclinée à 30°
\end{poutre}

    Attention : pour préciser l'angle, il faut obligatoirement préciser la position (même si c'est zéro).
    
    
        \subsubsection{Pas de "petit cercle" au bout de l'articulation :}
	
        \begin{verbatim}
\begin{poutre}{10cm}
    \articulation{5cm}[0][noCercle]   %Articulation sans le petit cercle
\end{poutre}
        \end{verbatim}

\begin{poutre}{10cm}
    \articulation{5cm}[0][noCercle]   %Articulation sans le petit cercle
\end{poutre}

    Attention : le mot clé doit être exactement \verb!noCercle! (avec le \verb!C! majuscule). De plus, il doit être précéder des autres options, (même si elles sont dans leur valeur par défaut).
	

	
  
    \subsection{Liaison Appui Simple}
	%------------------------------------
	
	\begin{center}
        \begin{tabular}{|c|}
            \hline
            \verb!\appuiSimple{position}[orientation][noCercle]!
            \\\hline
        \end{tabular}
    \end{center}
	
        Cette macro fonctionne exactement pareil que \verb!articulation! :
	
        \begin{verbatim}
\begin{poutre}{10cm}
    \appuiSimple{5cm}[0][noCercle]
    \appuiSimple{10cm}[90]
\end{poutre}
        \end{verbatim}

\begin{poutre}{10cm}
    \appuiSimple{5cm}[0][noCercle]
    \appuiSimple{10cm}[90]
\end{poutre}
	
	
    \subsection{Ajouter (graphiquement) un point}
	%------------------------------------
	
           \begin{verbatim}
\begin{poutre}{10cm}
    \point{5cm}{A}
\end{poutre}
        \end{verbatim}
	
        \begin{poutre}{10cm}
            \point{5cm}{A}
        \end{poutre}
	

    \subsection{Composantes inconnues (ou pas)}
	%------------------------------------
	
	
        \subsubsection{Composantes indépendantes}
        %.................................................
	
            \begin{center}
                \begin{tabular}{|c|}
                    \hline
                    \verb!\composanteX{position}{texte}!\\
                    \verb!\composanteY{position}{texte}!\\
                    \verb!\composanteN{position}{texte}!
                    \\\hline
                \end{tabular}
            \end{center}
	
	
           \begin{verbatim}
\begin{poutre}{10cm}
    \composanteX{3cm}{$X_0$}
    \composanteY{6cm}{$Y_2$}
    \composanteN{9cm}{C}
\end{poutre}
        \end{verbatim}
	
\begin{poutre}{10cm}
    \composanteX{3cm}{$X_0$}
    \composanteY{6cm}{$Y_2$}
    \composanteN{9cm}{C}
\end{poutre}
	
        \subsubsection{Composantes groupées}
        %.................................................
        
            \begin{center}
                \begin{tabular}{|l|}
                    \hline
                    \verb!\composanteEncastrement{distance}{texte X}{texte Y}{texte N}!\\
                    \verb!\composanteArticulation{distance}{texte X}{texte Y}!
                    \\\hline
                \end{tabular}
            \end{center}
        
           \begin{verbatim}
\begin{poutre}{10cm}
    \composanteEncastrement{0}{$X_O$}{$Y_O$}{$N_O$}
    \composanteArticulation{10cm}{$X_A$}{$Y_A$}
\end{poutre}
        \end{verbatim}
        
\begin{poutre}{10cm}
    \composanteEncastrement{0}{$X_O$}{$Y_O$}{$N_O$}
    \composanteArticulation{10cm}{$X_A$}{$Y_A$}
\end{poutre}
        
        
        \subsection{Actions mécaniques extérieures}
        %----------------------------------------------
        
            \subsubsection{Forces}
            %------------------------

            \begin{center}
                \begin{tabular}{|l|}
                    \hline
                    \verb!\Pforce{distance}{X}{Y}{texte}!
                    \\\hline
                \end{tabular}
            \end{center}
        
           \begin{verbatim}
\begin{poutre}{10cm}
    \Pforce{0cm}{1}{2}{$\vec R$}
    \Pforce{2cm}{-1}{2}{$\vec R$}
    \Pforce{4cm}{1}{-2}{$\vec R$}
    \Pforce{6cm}{-1}{-2}{$\vec R$}
    \Pforce{8cm}{0}{2}{$\vec R$}
\end{poutre}
        \end{verbatim}
        
\begin{poutre}{10cm}
    \Pforce{0cm}{1}{2}{$\vec R$}
    \Pforce{2cm}{-1}{2}{$\vec R$}
    \Pforce{4cm}{1}{-2}{$\vec R$}
    \Pforce{6cm}{-1}{-2}{$\vec R$}
    \Pforce{8cm}{0}{2}{$\vec R$}
\end{poutre}
        
        Remarque : le label se place tout seul à droite ou à gauche. Si ça ne vous plait pas, le mieux est de ne rien mettre et de faire un autre n\oe ud à l'emplacement qui vous plait.
	
	
	  
            \subsubsection{Moment}
            %------------------------

            \begin{center}
                \begin{tabular}{|l|}
                    \hline
                    \verb!\pmoment{distance}{texte}[inverse]!
                    \\\hline
                \end{tabular}
            \end{center}
        
           \begin{verbatim}
\begin{poutre}{10cm}
    \pmoment{3cm}{$vec M$}
    \pmoment{6cm}{$vec M$}[inverse]
\end{poutre}
        \end{verbatim}
	
\begin{poutre}{10cm}
    \pmoment{3cm}{$\vec M$}
    \pmoment{6cm}{$\vec M$}[inverse]
\end{poutre}


	  
            \subsubsection{Force Répartie}
            %-----------------------------------

            \begin{center}
                \begin{tabular}{|l|}
                    \hline
                    \verb!\pforceRepartie{x_min}{x_max}{texte}!
                    \\\hline
                \end{tabular}
            \end{center}
        
           \begin{verbatim}
\begin{poutre}{10cm}
    \pforceRepartie{1}{4}{$f$}
    \pforceRepartie{5}{9}{$f$}[2*(\x-5)]
\end{poutre}
        \end{verbatim}
	
\begin{poutre}{10cm}
    \pforceRepartie{1}{4}{$f$}
    \pforceRepartie{5}{9}{$f$}[2*(\x-5)]
\end{poutre}

  

    \subsection{Cotes}
	%------------------------------------      
	
            \begin{center}
                \begin{tabular}{|l|}
                    \hline
                    \verb!\abscisse[x_départ]{x_arrivée}{texte}[position verticale]!\\
                    \verb![options de la flèche][options du texte][options des côtes latérales]!
                    \\\hline
                \end{tabular}
            \end{center}
            
           \begin{verbatim}
\begin{poutre}{10cm}
	\abscisse[1]{4}{$x$}
	\abscisse{5}{$y$}[0.5]
	\abscisse{6}{$y$}[-1.5][red][yellow][blue]
\end{poutre}
        \end{verbatim}

\begin{poutre}{10cm}
	\abscisse[1]{4}{$x$}
	\abscisse{5}{$y$}[0.5]
	\abscisse{6}{$y$}[-1.5][red][yellow][blue]
\end{poutre}

    \subsection{Base}
	%------------------------------------ 
	
	
           \begin{verbatim}
\begin{poutre}{10cm}
	\Pbase{$\vec x_0$}{$\vec y_0$}
\end{poutre}
        \end{verbatim}

\begin{poutre}{10cm}
	\Pbase{$\vec x_0$}{$\vec y_0$}
\end{poutre}
	
	
	
           \begin{verbatim}
\begin{poutre}{10cm}
	\Pbase[1cm]{$\vec x_0$}{$\vec y_0$}[red]
\end{poutre}
        \end{verbatim}

\begin{poutre}{10cm}
	\Pbase[1cm]{$\vec x_0$}{$\vec y_0$}[red]
\end{poutre}


% 	\begin{poutre}{10}
%         \encastrement
%         \articulation[3cm]
%         \articulation[5cm][0][noCercle]
%         \appuiSimple[7cm][0][noCercle]
%         \appuiSimple[10cm][90]
%         \composanteEncastrement{5cm}{X}{Y}{Z}
% 	\end{poutre}
\end{document}
