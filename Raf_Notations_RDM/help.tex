\documentclass[a4paper,10pt]{article}

	\usepackage{xcolor}
	\usepackage{listings}	\lstset{language=[LaTeX]TeX,basicstyle=\ttfamily,texcsstyle=*\color{blue},identifierstyle=\color{brown},commentstyle=\color{gray}\itshape,escapechar=!,moretexcs={}}
	\usepackage[latin1]{inputenc}
	\usepackage{hyperref}


	\usepackage{Raf_Notations_RDM}
	\usepackage[francais]{babel}
	\everymath{\displaystyle}


	\newcommand{\rac}{({\color{red}Raccourci})}
	\newcommand{\ren}{({\color{blue}Renomm�})}

\begin{document}





	\begin{center}
		\hrule{\Large Notations de RDM}\\\hrule
	\end{center}

	(Version du 01/08/18)

	\section{Packages requis}
	%-------------------------------------

	\begin{itemize}
		\item \href{http://www.ctan.org/pkg/ifthen}{\textbf{ifthen}} : Package permettant une compilation � choix multiple,
		%\item \href{http://tug.ctan.org/tex-archive/macros/latex/contrib/xargs}{\textbf{xarg}} : Package permettant de cr�er des commandes � plusieurs arguments optionnels.
		%\item \href{http://www.ams.org/publications/authors/tex/amsfonts}{\textbf{amsfonts}} : Package qui ajoute des polices d'�critures math�matiques.
		%\item \href{http://www.ams.org/publications/authors/tex/amslatex}{\textbf{amsmath}} : Package qui ajoute des fonctions math�matiques non-standards.
		\item \href{http://www.ctan.org/pkg/mathrsfs}{\textbf{mathrsfs}} : Package qui rajoute des polices d'�critures math�matiques.
		%\item \href{http://www.ctan.org/pkg/color}{\textbf{color}} : Package permettant de mettre en couleur du texte, des lignes, etc.
		%\item \href{http://www.ctan.org/pkg/xspace}{\textbf{xspace}} : Package permettant de mettre des espaces apr�s les commandes.
		%\item \href{http://www.ctan.org/pkg/xstring}{\textbf{xstring}} : Package permettant travailler sur les cha�nes de caract�res (chercher/remplacer, etc.)
		\item \href{http://enseignement.allais.eu/page-latex}{\textbf{Raf\_Notations\_Actions-Meca}} : Package de notations d'actions m�caniques.
	\end{itemize}

	\section{Appel du package}
	%-------------------------------------

	Le package est appel� en d�but de document par la commande :
	\begin{verbatim}
\usepackage{Raf_Notations_RDM}
	\end{verbatim}

	Par d�faut, ce package utilise un certain nombre de notations raccourcies, susceptibles de rentrer en conflit avec d'autre package (mais tellement plus rapide � taper !).
	De plus, certaines commandes ont �t� rebaptis�e.
	Ces raccourcis et renommages seront cit�s (\rac\ ou \ren) dans les tableaux suivants.
	Pour ne pas cr�er ces raccourcis/renommage, il faut rentre l'option \verb!noRaccourci! � l'appel du package.

	\begin{verbatim}
usepackage[noRaccourci]{Raf_Notations_RDM}
	\end{verbatim}

	\section{Notations/annotations}
	%-------------------------------------------
	\noindent
	\begin{tabular}{|p{0.35\linewidth}|p{0.3\linewidth}|p{0.3\linewidth}|}
		\hline
			\textbf{Commandes}&\textbf{Rendus}&\textbf{Commentaires}
		\\\hline\hline
			\verb!\PCallig!			&	\PCallig			&	"P" Calligraphi�.
		\\\hline
			\verb!\ensemblePoutre!		&	\ensemblePoutre		&	Ensemble de points constituant la poutre.
		\\\hline
			\verb!\eP!			&	\eP			&	Raccourci de \verb!\ensemblePoutre!. \rac
		\\\hline
			\verb!\ensemblePoutreSup!	&	\ensemblePoutreSup	&	Partie de poutre " � droite ".
		\\\hline
			\verb!\ePp!			&	\ePp			&	Raccourci de \verb!\ensemblePoutreSup!. \rac
		\\\hline
			\verb!\ensemblePoutreInf!	&	\ensemblePoutreInf	&	Partie de poutre " � gauche ".
		\\\hline
			\verb!\ePm!			&	\ePm			&	Raccourci de \verb!\ensemblePoutreInf!.
		\\\hline
	\end{tabular}




	\section{Torseur de coh�sion}
	%-------------------------------------------
	\noindent
	\begin{tabular}{|p{0.35\linewidth}|p{0.3\linewidth}|p{0.3\linewidth}|}
		\hline
			\textbf{Commandes}&\textbf{Rendus}&\textbf{Commentaires}
		\\\hline\hline
			\verb!\tCohesion!		&	\tCohesion		&	Torseur de coh�sion
		\\\hline
			\verb!\composanteTCohesion{N}!		&	\composanteTCohesion{N}		&	Composante du torseur de coh�sion
		\\\hline
			\verb!\composanteTCohesion! \verb!{N}[x]!		&	\composanteTCohesion{N}[x]		&	Idem, avec une variable.
		\\\hline
			\verb!\Nt,\Ty,\Tz,\Mt,\Mfy,! \verb!\Mfz!		&	\Nt,\Ty,\Tz,\Mt,\Mfy,\Mfz		&	Composantes du torseur de coh�sions (les quatre premiers sont des raccourcis \rac). Attention : \Ty et \Tz existent d�j� dans d'autres de mes packages (je ne sais plus lesquels). Les commandes sont �cras�es, dans ce cas.
		\\\hline
			\verb!\Nt[x],\Ty[x],\Tz[x]! \verb!,\Mt[x],\Mfy[x],\Mfz[x]!		&	\Nt[x],\Ty[x],\Tz[x],\Mt[x],\Mfy[x],\Mfz[x]		&	Idem avec une variable
		\\\hline
	\end{tabular}




	\section{Graphe}
	%-------------------------------------------
	\begin{verbatim}
%{Label y}[x_min=0]{x_max}{y_min}{y_max}[scale_x=8][scale_y=0.2]
\begin{grapheRDM}{\Nt[x]}[0]{1.5}{-20}{10}
    \grapheRDMTrace{15*\x-15}[0][0.5]
    \grapheRDMTrace{7.5*\x-11.25}[0.5][1.5]
\end{grapheRDM}
    \end{verbatim}

%{Label y}[x_min=0]{x_max}{y_min}{y_max}[scale_x=8][scale_y=0.2]
\begin{grapheRDM}{\Nt[x]}[0]{1.5}{-20}{10}
    \grapheRDMTrace{15*\x-15}[0][0.5]
    \grapheRDMTrace{7.5*\x-11.25}[0.5][1.5]
\end{grapheRDM}

\end{document}
