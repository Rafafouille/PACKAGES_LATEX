\documentclass[a4paper,12pt]{article}

	\usepackage{xcolor}
	%\usepackage{listings}
	\usepackage{listingsutf8}
	\lstset{language=[LaTeX]TeX,basicstyle=\ttfamily,texcsstyle=*\color{blue},identifierstyle=\color{brown},commentstyle=\color{gray}\itshape,escapechar=!,moretexcs={}}
	%\usepackage[utf8]{inputenc}
	\usepackage{titlesec}
	\usepackage{hyperref}
		\hypersetup{colorlinks,citecolor=black,filecolor=black,linkcolor=red,urlcolor=blue}

	\usepackage{ifpdf}
	\usepackage{Raf_Boites}

\definecolor{couleurExemple}{RGB}{240,240,240}%{250,250,250} %Couleur du fond
\newenvironment{Exemple}{\begin{bclogo}[couleur=couleurExemple,arrondi=0.3,noborder = true,logo=\bcloupe,epBarre = 0]{}}{\end{bclogo}\vspace{0.5cm}}
\lstnewenvironment{code}{\footnotesize\setbox1=\vbox\bgroup}
			{\egroup\begin{center}\begin{minipage}{0.8\linewidth}\begin{bclogo}[couleur=white,logo=\bccrayon,noborder = true]{Code}\box1\end{bclogo}\end{minipage}\end{center}}


\begin{document}

	\begin{center}
		\hrule{\Large Boites pour les cours}\\\hrule
	\end{center}

	\tableofcontents
	\newcommand{\sectionbreak}{\newpage}


\newenvironment{ttt}{\begin{center}}{\end{center}}




	\section{Introduction}
	%-----------------------------

		Ce package permet de tracer rapidement des boîtes au travers du package \textbf{bclogo}.

		\subsection{Packages nécessaires}

		Pour fonctionner, ce package nécessite les packages suivants :
		\begin{itemize}
			\item \href{http://melusine.eu.org/syracuse/G/bclogo/dev/latex/}{\textbf{bclogo}} : pour faire de jolies boîtes (ce package appelle Tikz). %Attention ! Il s'agit de la version 3 du package ! À l'heure où j'écris ce document, elle n'est pas officielle et est téléchargeable sur le site du projet (à installer manuellement).
			\item \href{http://ctan.org/pkg/graphicx}{\textbf{graphicx}} : Permet notamment de faire des logos personnalisés
			\item \href{http://www.ctan.org/tex-archive/macros/latex/contrib/xcolor}{\textbf{xcolor}} : Permet la gestion des couleurs.
			\item \href{http://www.ctan.org/tex-archive/macros/latex/contrib/xargs}{\textbf{xargs}} : Permet de faire des commandes avec plusieurs arguments optionnels.
		\end{itemize}

		\subsection{Appel du package}

		Pour appeler ce package, il suffit d'ajouter en début de document :

		\begin{lstlisting}
\usepackage{Raf_Boites}
		\end{lstlisting}



	\section{Boite ``Astuce''}
	%--------------------------------------

\begin{code}%==========================================
\begin{astuce}[Titre (optionnel)]
	Trucs et astuces
\end{astuce}listingsutf8
\end{code}

\begin{astuce}[Titre (optionnel)]
	Trucs et astuces
\end{astuce}%--------------------------------------------

\begin{code}%==========================================
\begin{astuce*}[Titre (optionnel)]
	Trucs et astuces
\end{astuce*}
\end{code}

\begin{astuce*}[Titre (optionnel)]
	Trucs et astuces
\end{astuce*}%--------------------------------------------

\begin{code}%==========================================
\begin{astuces}[Titre (optionnel)]
	\item Trucs et astuces 1...
	\item Trucs et astuces 2...
\end{astuces}
\end{code}

\begin{astuces}[Titre (optionnel)]
	\item Trucs et astuces 1...
	\item Trucs et astuces 2...
\end{astuces}%--------------------------------------------

\begin{code}%==========================================
\begin{astuces*}[Titre (optionnel)]
	\item Trucs et astuces 1...
	\item Trucs et astuces 2...
\end{astuces*}
\end{code}

\begin{astuces*}[Titre (optionnel)]
	\item Trucs et astuces 1...
	\item Trucs et astuces 2...
\end{astuces*}%--------------------------------------------










	\section{Boite ``Attention''}
	%--------------------------------------

\begin{code}%==========================================
\begin{attention}[Titre (optionnel)]
	Danger...
\end{attention}
\end{code}

\begin{attention}[Titre (optionnel)]
	Danger...
\end{attention}%--------------------------------------------




	\section{Boite ``définitions''}
	%===============================


\begin{code}%==========================================
\begin{definition}[Titre (optionnel)]
	Ma définition
\end{definition}
\end{code}

\begin{definition}[Titre (optionnel)]
	Ma définition
\end{definition}%--------------------------------------------


\begin{code}%==========================================	
\begin{definition*}[Titre (optionnel)]
	Ma définition
\end{definition*}
\end{code}

\begin{definition*}[Titre (optionnel)]
	Ma définition
\end{definition*}%--------------------------------------------

\begin{code}%==========================================
\begin{definitions}[Titre (optionnel)]
	\item[Définition 1] Ceci est ma définition 1
	\item[Définition 2] Ceci est ma définition 2
	\item ...
\end{definitions}
\end{code}

\begin{definitions}[Titre (optionnel)]
	\item[Définition 1] Ceci est ma définition 1
	\item[Définition 2] Ceci est ma définition 2
	\item ...
\end{definitions}%--------------------------------------------

\begin{code}%==========================================
\begin{definitions*}[Titre (optionnel)]
	\item[Définition 1] Ceci est ma définition 1
	\item[Définition 2] Ceci est ma définition 2
	\item ...
\end{definitions*}
\end{code}

\begin{definitions*}[Titre (optionnel)]
	\item[Définition 1] Ceci est ma définition 1
	\item[Définition 2] Ceci est ma définition 2
	\item ...
\end{definitions*}%--------------------------------------------



	\section{Boite ``Hypothèses''}
	%===============================

\begin{code}%==========================================
\begin{hypothese}[Titre (optionnel)]
	Mon hypothèse
\end{hypothese}
\end{code}

\begin{hypothese}[Titre (optionnel)]
	Mon hypothèse
\end{hypothese}%--------------------------------------------


\begin{code}%==========================================	
\begin{hypothese*}[Titre (optionnel)]
	Mon hypothèse
\end{hypothese*}
\end{code}

\begin{hypothese*}[Titre (optionnel)]
	Mon hypothèse
\end{hypothese*}%--------------------------------------------

\begin{code}%==========================================
\begin{hypotheses}[Titre (optionnel)]
	\item Définition 1
	\item Définition 2
	\item ...
\end{hypotheses}
\end{code}

\begin{hypotheses}[Titre (optionnel)]
	\item Définition 1
	\item Définition 2
	\item ...
\end{hypotheses}%--------------------------------------------

\begin{code}%==========================================
\begin{hypotheses*}[Titre (optionnel)]
	\item Définition 1
	\item Définition 2
	\item ...
\end{hypotheses*}
\end{code}

\begin{hypotheses*}[Titre (optionnel)]
	\item Définition 1
	\item Définition 2
	\item ...
\end{hypotheses*}%--------------------------------------------






	\section{Boite ``Démonstration''}
	%---------------------------------------

\begin{code}%==========================================
\begin{demonstration}[Titre (optionnel)]
	Ma démo
\end{demonstration}
\end{code}

\begin{demonstration}[Titre (optionnel)]
	Ma démo
\end{demonstration}%--------------------------------------------

\begin{code}%==========================================
\begin{demonstration*}[Titre (optionnel)]
	Ma démo
\end{demonstration*}
\end{code}

\begin{demonstration*}[Titre (optionnel)]
	Ma démo
\end{demonstration*}%--------------------------------------------

\begin{code}%==========================================
\begin{demonstrations}[Titre (optionnel)]
	\item Démo 1
	\item Démo 2
\end{demonstrations}
\end{code}

\begin{demonstrations}[Titre (optionnel)]
	\item Démo 1
	\item Démo 2
\end{demonstrations}%--------------------------------------------

\begin{code}%==========================================
\begin{demonstrations*}[Titre (optionnel)]
	\item Démo 1
	\item Démo 2
\end{demonstrations*}
\end{code}

\begin{demonstrations*}[Titre (optionnel)]
	\item Démo 1
	\item Démo 2
\end{demonstrations*}%--------------------------------------------




	\section{Boite ``Exemple''}
	%---------------------------------------

\begin{code}%==========================================
\begin{exemple}[Titre (optionnel)]
	Mon exemple
\end{exemple}
\end{code}

\begin{exemple}[Titre (optionnel)]
	Mon exemple
\end{exemple}%--------------------------------------------

\begin{code}%==========================================
\begin{exemple*}[Titre (optionnel)]
	Mon exemple
\end{exemple*}
\end{code}

\begin{exemple*}[Titre (optionnel)]
	Mon exemple
\end{exemple*}%--------------------------------------------

\begin{code}%==========================================
\begin{exemples}[Titre (optionnel)]
	\item Exemple 1...
	\item Exemple 2...
\end{exemples}
\end{code}

\begin{exemples}[Titre (optionnel)]
	\item Exemple 1...
	\item Exemple 2...
\end{exemples}%--------------------------------------------

\begin{code}%==========================================
\begin{exemples*}[Titre (optionnel)]
	\item Exemple 1...
	\item Exemple 2...
\end{exemples*}
\end{code}

\begin{exemples*}[Titre (optionnel)]
	\item Exemple 1...
	\item Exemple 2...
\end{exemples*}%--------------------------------------------






	\section{Boite ``Important'}
	%--------------------------------------

\begin{code}%==========================================
\begin{important}[Titre (optionnel)]
	Truc important
\end{important}
\end{code}

\begin{important}[Titre (optionnel)]
	Truc important
\end{important}%--------------------------------------------


	\section{Boite ``Making Of'}
	%--------------------------------------

	Visible si le MakinOf est actif. (\verb!\setMakingOfOn! et \verb!\setMakingOfOff! peuvent être placé en début de document)

\begin{code}%==========================================
\setMakingOfOn
\begin{makingOf}
	Avec making of actif
\end{makingOf}
\setMakingOfOff
\begin{makingOf}
	Sans making of actif
\end{makingOf}
\end{code}

\setMakingOfOn
\begin{makingOf}
	Avec making of actif
\end{makingOf}
\setMakingOfOff
\begin{makingOf}
	Sans making of actif
\end{makingOf}%--------------------------------------------


	\section{Boite ``Modèle''}
	%---------------------------------------

\begin{code}%==========================================
\begin{modele}[Titre (optionnel)]
	Mon modele
\end{modele}
\end{code}

\begin{modele}[Titre (optionnel)]
	modele
\end{modele}%--------------------------------------------

\begin{code}%==========================================
\begin{modele*}[Titre (optionnel)]
	Objectif
\end{modele*}

\end{code}
\begin{modele*}[Titre (optionnel)]
	modele
\end{modele*}%--------------------------------------------


	\section{Boite ``Objectif''}
	%---------------------------------------

\begin{code}%==========================================
\begin{objectif}[Titre (optionnel)]
	Objectif
\end{objectif}
\end{code}

\begin{objectif}[Titre (optionnel)]
	Objectif
\end{objectif}%--------------------------------------------

\begin{code}%==========================================
\begin{objectif*}[Titre (optionnel)]
	Objectif
\end{objectif*}
\end{code}

\begin{objectif*}[Titre (optionnel)]
	Objectif
\end{objectif*}%--------------------------------------------

\begin{code}%==========================================
\begin{objectifs}[Titre (optionnel)]
	\item Objectif 1...
	\item Objectif 2...
\end{objectifs}
\end{code}

\begin{objectifs}[Titre (optionnel)]
	\item Objectif 1...
	\item Objectif 2...
\end{objectifs}%--------------------------------------------

\begin{code}%==========================================
\begin{objectifs*}[Titre (optionnel)]
	\item Objectif 1...
	\item Objectif 2...
\end{objectifs*}
\end{code}

\begin{objectifs*}[Titre (optionnel)]
	\item Objectif 1...
	\item Objectif 2...
\end{objectifs*}%--------------------------------------------




	\section{Boite ``Principe''}
	%------------------------------------

\begin{code}%==========================================
\begin{principe}[Titre (optionnel)]
	Principe...
\end{principe}
\end{code}

\begin{principe}[Titre (optionnel)]
	Principe...
\end{principe}%--------------------------------------------





	\section{Boite ``Propriete'}
	%--------------------------------------

\begin{code}%==========================================
\begin{propriete}[Titre (optionnel)]
	Ma propriété...
\end{propriete}
\end{code}

\begin{propriete}[Titre (optionnel)]
	Ma propriété...
\end{propriete}%--------------------------------------------

\begin{code}%==========================================
\begin{propriete*}[Titre (optionnel)]
	Ma propriété...
\end{propriete*}
\end{code}

\begin{propriete*}[Titre (optionnel)]
	Ma propriété...
\end{propriete*}%--------------------------------------------

\begin{code}%==========================================
\begin{proprietes}[Titre (optionnel)]
	\item[prop1] Ma propriété 1
	\item[prop1] Ma propriété 2
	\item ..
\end{proprietes}
\end{code}

\begin{proprietes}[Titre (optionnel)]
	\item[prop1] Ma propriété 1
	\item[prop1] Ma propriété 2
	\item ..
\end{proprietes}%--------------------------------------------

\begin{code}%==========================================
\begin{proprietes*}[Titre (optionnel)]
	\item Ma propriété 1
	\item Ma propriété 2
\end{proprietes*}
\end{code}

\begin{proprietes*}[Titre (optionnel)]
	\item Ma propriété 1
	\item Ma propriété 2
\end{proprietes*}%--------------------------------------------





	

	\section{Boite ``Remarque''}
	%---------------------------------------


\begin{code}%==========================================
\begin{remarque}[Titre (optionnel)]
	Ma remarque...
\end{remarque}
\end{code}

\begin{remarque}[Titre (optionnel)]
	Ma remarque...
\end{remarque}%--------------------------------------------


\begin{code}%==========================================
\begin{remarque*}[Titre (optionnel)]
	Ma remarque...
\end{remarque*}
\end{code}

\begin{remarque*}[Titre (optionnel)]
	Ma remarque...
\end{remarque*}%--------------------------------------------


\begin{code}%==========================================
\begin{remarques}[Titre (optionnel)]
	\item Remarque 1...
	\item Remarque 2...
\end{remarques}
\end{code}

\begin{remarques}[Titre (optionnel)]
	\item Remarque 1...
	\item Remarque 2...
\end{remarques}%--------------------------------------------

\begin{code}%==========================================
\begin{remarques*}[Titre (optionnel)]
	\item Remarque 1...
	\item Remarque 2...
\end{remarques*}
\end{code}

\begin{remarques*}[Titre (optionnel)]
	\item Remarque 1...
	\item Remarque 2...
\end{remarques*}%--------------------------------------------



	\section{Boite ``Résumé''}
	%---------------------------------------

\begin{code}%==========================================
\begin{resume}[Titre (optionnel)]
	Résumé...
\end{resume}
\end{code}

\begin{resume}[Titre (optionnel)]
	Résumé...
\end{resume}%--------------------------------------------

\begin{code}%==========================================
\begin{resume*}[Titre (optionnel)]
	Résumé...
\end{resume*}
\end{code}

\begin{resume*}[Titre (optionnel)]
	Résumé...
\end{resume*}%--------------------------------------------


	\section{Boite ``Théorème''}
	%---------------------------------------

\begin{code}%==========================================
\begin{theoreme}[Titre (optionnel)]
	Mon théorème...
\end{theoreme}
\end{code}

\begin{theoreme}[Titre (optionnel)]
	Mon théorème...
\end{theoreme}%--------------------------------------------

\begin{code}%==========================================
\begin{theoreme*}[Titre (optionnel)]
	Mon théorème...
\end{theoreme*}
\end{code}

\begin{theoreme*}[Titre (optionnel)]
	Mon théorème...
\end{theoreme*}%--------------------------------------------

\begin{code}%==========================================
\begin{theoremes}[Titre (optionnel)]
	\item[Mon théorème 1] Mon théorème 1
	\item[Mon théorème 2] Mon théorème 2
\end{theoremes}
\end{code}

\begin{theoremes}[Titre (optionnel)]
	\item[Mon théorème 1] Mon théorème 1
	\item[Mon théorème 2] Mon théorème 2
\end{theoremes}%--------------------------------------------

\begin{code}%==========================================
\begin{theoremes*}[Titre (optionnel)]
	\item[Mon théorème 1] Mon théorème 1
	\item[Mon théorème 2] Mon théorème 2
\end{theoremes*}
\end{code}

\begin{theoremes*}[Titre (optionnel)]
	\item[Mon théorème 1] Mon théorème 1
	\item[Mon théorème 2] Mon théorème 2
\end{theoremes*}%--------------------------------------------



	\section{Boite ``Méthode''}
	%---------------------------------------

\begin{code}%==========================================
\begin{methode}[Titre (optionnel)]
	Ma méthode...
\end{methode}
\end{code}

\begin{methode}[Titre (optionnel)]
	Ma méthode...
\end{methode}%--------------------------------------------

\begin{code}%==========================================
\begin{methode*}[Titre (optionnel)]
	Ma méthode...
\end{methode*}
\end{code}

\begin{methode*}[Titre (optionnel)]
	Ma méthode...
\end{methode*}%--------------------------------------------

\begin{code}%==========================================
\begin{methodes}[Titre (optionnel)]
	\item[Ma méthode 1] Ma méthode 1
	\item[Ma méthode 2] Ma méthode 2
\end{methodes}
\end{code}

\begin{methodes}[Titre (optionnel)]
	\item[Ma méthode 1] Ma méthode 1
	\item[Ma méthode 2] Ma méthode 2
\end{methodes}%--------------------------------------------

\begin{code}%==========================================
\begin{methodes*}[Titre (optionnel)]
	\item[Ma méthode 1] Ma méthode 1
	\item[Ma méthode 2] Ma méthode 2
\end{methodes*}
\end{code}

\begin{methodes*}[Titre (optionnel)]
	\item[Ma méthode 1] Ma méthode 1
	\item[Ma méthode 2] Ma méthode 2
\end{methodes*}%--------------------------------------------


	\section{Boite ``reponse'' et `` bigReponse ''}
	%---------------------------------------

	\subsection{Version élève}
	
	
\begin{code}%==========================================
\version{eleve}
Liste des commandes en mode élève :
La fin de cette phrase est \reponse{cachée !!!!}.
Par contre, on remarque qu'il y a un blanc à la place du texte.

La fin de cette phrase est \reponse{cachée !!!!}[........].
Cette fois, le texte a été remplacé par un autre texte (des points).

\begin{bigReponse}
	Ceci est tout un paragraphe de caché,
	avec des listes, etc...
	\begin{itemize}
		\item exemple itemize 1
		\item exemple itemize 2
		\item exemple itemize 3
	\end{itemize}
\end{bigReponse}
\begin{bigReponse}[Ceci est un texte de remplacement]
	Ceci est tout un paragraphe de caché,
	avec des listes, etc...
	\begin{itemize}
		\item exemple itemize 1
		\item exemple itemize 2
		\item exemple itemize 3
	\end{itemize}
\end{bigReponse}
\end{code}

\version{eleve}
Liste des commandes en mode élève :
La fin de cette phrase est \reponse{cachée !!!!}. Par contre, on remarque qu'il y a un blanc à la place du texte.

La fin de cette phrase est \reponse{cachée !!!!}[........]. Cette fois, le texte a été remplacé par un autre texte (des points).

\begin{bigReponse}
	Ceci est tout un paragraphe de caché,
	avec des listes, etc...
	\begin{itemize}
		\item exemple itemize 1
		\item exemple itemize 2
		\item exemple itemize 3
	\end{itemize}
\end{bigReponse}
\begin{bigReponse}[Ceci est un texte de remplacement]
	Ceci est tout un paragraphe de caché,
	avec des listes, etc...
	\begin{itemize}
		\item exemple itemize 1
		\item exemple itemize 2
		\item exemple itemize 3
	\end{itemize}
\end{bigReponse}




	\subsection{Version prof}

\begin{code}
\version{prof}
Liste des commandes en mode prof :
La fin de cette phrase est \reponse{cachée !!!!}. Par contre, on remarque qu'il y a un blanc à la place du texte.

La fin de cette phrase est \reponse{cachée !!!!}[........]. Cette fois, le texte a été remplacé par un autre texte (des points).

\begin{bigReponse}
	Ceci est tout un paragraphe de caché,
	avec des listes, etc...
	\begin{itemize}
		\item exemple itemize 1
		\item exemple itemize 2
		\item exemple itemize 3
	\end{itemize}
\end{bigReponse}
\begin{bigReponse}[Ceci est un texte de remplacement]
	Ceci est tout un paragraphe de caché,
	avec des listes, etc...
	\begin{itemize}
		\item exemple itemize 1
		\item exemple itemize 2
		\item exemple itemize 3
	\end{itemize}
\end{bigReponse}

\end{code}


\version{prof}
Liste des commandes en mode prof :
La fin de cette phrase est \reponse{cachée !!!!}. Par contre, on remarque qu'il y a un blanc à la place du texte.

La fin de cette phrase est \reponse{cachée !!!!}[........]. Cette fois, le texte a été remplacé par un autre texte (des points).

\begin{bigReponse}
	Ceci est tout un paragraphe de caché,
	avec des listes, etc...
	\begin{itemize}
		\item exemple itemize 1
		\item exemple itemize 2
		\item exemple itemize 3
	\end{itemize}
\end{bigReponse}
\begin{bigReponse}[Ceci est un texte de remplacement]
	Ceci est tout un paragraphe de caché,
	avec des listes, etc...
	\begin{itemize}
		\item exemple itemize 1
		\item exemple itemize 2
		\item exemple itemize 3
	\end{itemize}
\end{bigReponse}




	\subsection{Version correction}

\begin{code}
\version{correction}
Liste des commandes en mode corrige :
La fin de cette phrase est \reponse{cachée !!!!}. Par contre, on remarque qu'il y a un blanc à la place du texte.

La fin de cette phrase est \reponse{cachée !!!!}[........]. Cette fois, le texte a été remplacé par un autre texte (des points).

\begin{bigReponse}
	Ceci est tout un paragraphe de caché,
	avec des listes, etc...
	\begin{itemize}
		\item exemple itemize 1
		\item exemple itemize 2
		\item exemple itemize 3
	\end{itemize}
\end{bigReponse}
\begin{bigReponse}[Ceci est un texte de remplacement]
	Ceci est tout un paragraphe de caché,
	avec des listes, etc...
	\begin{itemize}
		\item exemple itemize 1
		\item exemple itemize 2
		\item exemple itemize 3
	\end{itemize}
\end{bigReponse}
\end{code}




\version{correction}
Liste des commandes en mode corrige :
La fin de cette phrase est \reponse{cachée !!!!}. Par contre, on remarque qu'il y a un blanc à la place du texte.

La fin de cette phrase est \reponse{cachée !!!!}[........]. Cette fois, le texte a été remplacé par un autre texte (des points).

\begin{bigReponse}
	Ceci est tout un paragraphe de caché,
	avec des listes, etc...
	\begin{itemize}
		\item exemple itemize 1
		\item exemple itemize 2
		\item exemple itemize 3
	\end{itemize}
\end{bigReponse}
\begin{bigReponse}[Ceci est un texte de remplacement]
	Ceci est tout un paragraphe de caché,
	avec des listes, etc...
	\begin{itemize}
		\item exemple itemize 1
		\item exemple itemize 2
		\item exemple itemize 3
	\end{itemize}
\end{bigReponse}





	\subsection{Version correction sans couleur}

\begin{code}
\version{correction}
\noColorCorrection
\begin{bigReponse}
	Ceci est tout un paragraphe de caché,
	avec des listes, etc...
	\begin{itemize}
		\item exemple itemize 1
		\item exemple itemize 2
		\item exemple itemize 3
	\end{itemize}
\end{bigReponse}
\end{code}


\version{correction}
\noColorCorrection
\begin{bigReponse}
	Ceci est tout un paragraphe de caché,
	avec des listes, etc...
	\begin{itemize}
		\item exemple itemize 1
		\item exemple itemize 2
		\item exemple itemize 3
	\end{itemize}
\end{bigReponse}



	\subsection{afficheCorrection}

\begin{code}
\version{prof}
\ifthenelse{\boolean{afficheCorrection}}
	{J'affiche la correction}
	{Je n'affiche pas la correction}
	
\version{eleve}
\ifthenelse{\boolean{afficheCorrection}}
	{J'affiche la correction}
	{Je n'affiche pas la correction}
\end{code}

\version{prof}
\ifthenelse{\boolean{afficheCorrection}}
	{J'affiche la correction}
	{Je n'affiche pas la correction}
	
\version{eleve}
\ifthenelse{\boolean{afficheCorrection}}
	{J'affiche la correction}
	{Je n'affiche pas la correction}


%--------------------------------------------


	\section{Boite ``texteCache'' (déprécié)}
	%---------------------------------------

\begin{code}%==========================================
Par défaut, le texte caché est apparent :
\begin{texteCache}
	Ce texte est Caché.\\
\end{texteCache}
Pour le cacher (ie : le remplacer par un blanc de même taille),
 on active la commande :
\setTexteATrouOn
\begin{texteCache}
	Ce texte est Caché.\\
\end{texteCache}
Le même texte caché Texte caché avec un blanc additionnel de taille régulée :
\setTexteATrouOn
\begin{texteCache}[3cm]
	Ce texte est Caché.\\
\end{texteCache}
Avec le making of (utiles pour les corrections) :
\setMakingOfOn
\begin{texteCache}
	Ce texte est Caché.\\
\end{texteCache}
Pour afficher à nouveau le texte :
\setTexteATrouOff
\begin{texteCache}
	Ce texte est Caché.\\
\end{texteCache}
\end{code}

Par défaut, le texte caché est apparent :
\begin{texteCache}
	Ce texte est Caché.\\
\end{texteCache}
Pour le cacher (ie : le remplacer par un blanc de même taille), on active la commande :
\setTexteATrouOn
\begin{texteCache}
	Ce texte est Caché.\\
\end{texteCache}
Le même texte caché avec un blanc additionnel de taille régulée :
\setTexteATrouOn
\begin{texteCache}[3cm]
	Ce texte est Caché.\\
\end{texteCache}
Avec le making of (utiles pour les corrections) :
\setMakingOfOn
\begin{texteCache}
	Ce texte est Caché.\\
\end{texteCache}
Pour afficher à nouveau le texte :
\setTexteATrouOff
\begin{texteCache}
	Ce texte est Caché.\\
\end{texteCache}
%--------------------------------------------


\section{Questions}
%==========================

\begin{code}%==========================================
\question{Ceci est une question.}
\end{code}

\question{Ceci est une question.}

\begin{code}%==========================================
\question[2]{Ceci est une question avec barème.}
\end{code}

\question[2]{Ceci est une question avec barème.}

\begin{code}%==========================================
\noBareme
\question[2]{Ceci est une question avec barème désactivé...}
\yesBareme
\question[2]{...puis réactivé.}
\end{code}

\noBareme
\question[2]{Ceci est une question avec barème désactivé...}
\yesBareme
\question[2]{...puis réactivé.}


\begin{code}%==========================================
\resetQuestions
\question{Numéro de question réinitialisé (automatique quand on commence un nouvel exercice)}
\end{code}

\resetQuestions
\question{Numéro de question réinitialisé (automatique quand on commence un nouvel exercice)}




\section{Exercice}
%==========================


\begin{code}%==========================================
\sectionExercice{Un exercice}
\question{Compteur de question remis à zéro}
\subsectionExercice{Un autre exercice}
\question{Compteur de question remis à zéro}
\subsubsectionExercice{Encore un exercice}
\question{Compteur de question remis à zéro}
\end{code}

\sectionExercice{Un exercice}
\question{Compteur de question remis à zéro}
\subsectionExercice{Un autre exercice}
\question{Compteur de question remis à zéro}
\subsubsectionExercice{Encore un exercice}
\question{Compteur de question remis à zéro}


\begin{code}%==========================================
%\ begin{exerice}{un probleme}
Enonce du probleme...
%\ end{exerice}
\end{code}

\begin{exercice}{un probleme}
Enonce du probleme...
\question{Une question qui repart à Q1}
\end{exercice}





%==========================
\section{Document-réponse}
%==========================


\begin{code}%=============\ =============================
\begin{docReponse}
Ceci est un document réponse qui change
de page et change sa numérotation seul.
\end{docReponse}
\end{code}
(Voir page suivante)

\begin{docReponse}
Ceci est un document réponse qui change
de page et change sa numérotation seul.
\end{docReponse}


%==========================
\section{Document-ressource}
%==========================


\begin{code}%==========================================
\begin{docRessource}
Ceci est un document ressource qui
change de page et change sa numérotation seul.
\end{docRessource}
\end{code}
(Voir page suivante)

\begin{docRessource}
Ceci est un document réponse qui
change de page et change sa numérotation seul.
\end{docRessource}


\end{document}
