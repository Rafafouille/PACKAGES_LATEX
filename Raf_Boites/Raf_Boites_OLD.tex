%%%%%%%%%%%%%%%%%%%%%%%%%%%%%%%%%%%%%%%%%%%%
%
%	Boites
%
%%%%%%%%%%%%%%%%%%%%%%%%%%%%%%%%%%%%%%%%%%%%
\NeedsTeXFormat{LaTeX2e}[1999/01/01]
\ProvidesPackage{Raf_Boites}[2016/08/15]

%\pdfminorversion=4	%POUR L'ANCIENNE VERSION DE LINUX : On diminue la version de PDF pour les logos perso de BCLOGO (car ca marche plus avec les nouvelles versions de PDFlatex)
 
%chargement des extensions requises au bon fonctionnement de l'extension et des documents
\RequirePackage[tikz]{bclogo}	%Pour faire des boites !!!! VERSION 3 !!!!!! http://melusine.eu.org/syracuse/G/bclogo/dev/latex/
\RequirePackage{graphicx}	%Pour les logos perso
\RequirePackage{xcolor}
\RequirePackage{xargs}
\RequirePackage{enumitem}

%\DeclareOption{raccourcis}{\setboolean{boolRaccourcisAF}{true}}
\ProcessOptions

%%%%%%%%%%%%%%%%%%%%%%%%%%%%%%%%%%%%%%%%%%%%%% 
%commandes et/ou environnements personnalisés
 

%TAILLE DES BOITES : pourcentage de la taille de la ligne
\newcommand*{\tailleDesBoites}	{1}	
\newcommand*{\margesBoites}	{0}


%%%%%%%%%%%%%%%%%%%%%%%%%%%%%%%%%%%%%%%%%%
% DEFINITION
%%%%%%%%%%%%%%%%%%%%%%%%%%%%%%%%%%%%%%%%%%

	\definecolor{fond_definition}{RGB}{255,245,245}
	\definecolor{bord_definition}{RGB}{0,0,0} 

	\newcounter{cptDefinition}	%Compteur de definitions

	\newenvironmentx{definition}[2][1=,2=]	{	\refstepcounter{cptDefinition}
							\begin{bclogo}[margeG=\margesBoites,margeD=\margesBoites,couleur=fond_definition,couleurBord=bord_definition,arrondi=0.2,logo=\bcplume,#2]{Définition \thecptDefinition\ : \emph{#1}}
						}
						{	\end{bclogo}
						}

	\newenvironmentx{definitions}[2][1=,2=]	{\refstepcounter{cptDefinition}
							\renewcommand{\descriptionlabel}[1] {	\hspace{\labelsep}
												$\bullet$
												\ifthenelse{\equal{##1}{}}
													{}
													{\textbf{##1} :}
											}
							\begin{bclogo}[margeG=\margesBoites,margeD=\margesBoites,couleur=fond_definition,couleurBord=bord_definition,arrondi=0.2,logo=\bcplume,#2]{Définitions \thecptDefinition\ : \emph{#1}}
									\begin{description}[leftmargin=2cm]%,style=nextline]
						}
						{			\end{description}
							\end{bclogo}
						}

	\newenvironmentx{definition*}[2][1=,2=]	{	\begin{bclogo}[margeG=\margesBoites,margeD=\margesBoites,couleur=fond_definition,couleurBord=bord_definition,arrondi=0.2,logo=\bcplume,#2]{Définition : \emph{#1}}
						}
						{		\end{bclogo}
						}

	\newenvironmentx{definitions*}[2][1=,2=]{	\renewcommand{\descriptionlabel}[1] {	\hspace{\labelsep}
													$\bullet$
													\ifthenelse{\equal{##1}{}}
														{}
														{\textbf{##1} :}
												}
							\begin{bclogo}[margeG=\margesBoites,margeD=\margesBoites,couleur=fond_definition,couleurBord=bord_definition,arrondi=0.2,logo=\bcplume,#2]{Définitions : \emph{#1}
						}
								\begin{description}[leftmargin=2cm]}%,style=nextline]}
						{		\end{description}
							\end{bclogo}
						}


%%%%%%%%%%%%%%%%%%%%%%%%%%%%%%%%%%%%%%%%%%
% HYPOTHESE
%%%%%%%%%%%%%%%%%%%%%%%%%%%%%%%%%%%%%%%%%%

	\definecolor{fond_hypothese}{RGB}{255,245,245}
	\definecolor{bord_hypothese}{RGB}{0,0,0} 

	\newcounter{cptHypothese}	%Compteur de definitions

	\newenvironmentx{hypothese}[2][1=,2=]	{\refstepcounter{cptHypothese}
						\begin{center}
							\begin{minipage}{\tailleDesBoites\linewidth}
								\begin{bclogo}[couleur=fond_hypothese,couleurBord=bord_hypothese,arrondi=0.2,logo=\bctrombone,#2]{Hypothèse \thecptHypothese\ : \emph{#1}}}
						{		\end{bclogo}
							\end{minipage}
						\end{center}}

	\newenvironmentx{hypotheses}[2][1=,2=]	{\refstepcounter{cptHypothese}
						\begin{center}
							\begin{minipage}{\tailleDesBoites\linewidth}
								\begin{bclogo}[couleur=fond_hypothese,couleurBord=bord_hypothese,arrondi=0.2,logo=\bctrombone,#2]{Hypothèses \thecptHypothese\ : \emph{#1}}
									\begin{itemize}}
						{			\end{itemize}
								\end{bclogo}
							\end{minipage}
						\end{center}}

	\newenvironmentx{hypothese*}[2][1=,2=]	{\begin{center}
							\begin{minipage}{\tailleDesBoites\linewidth}
								\begin{bclogo}[couleur=fond_hypothese,couleurBord=bord_hypothese,arrondi=0.2,logo=\bctrombone,#2]{Hypothèse : \emph{#1}}}
						{		\end{bclogo}
							\end{minipage}
						\end{center}}

	\newenvironmentx{hypotheses*}[2][1=,2=]	{\begin{center}
							\begin{minipage}{\tailleDesBoites\linewidth}
								\begin{bclogo}[couleur=fond_hypothese,couleurBord=bord_hypothese,arrondi=0.2,logo=\bctrombone,#2]{Hypothèses : \emph{#1}}
									\begin{itemize}}
						{			\end{itemize}
								\end{bclogo}
							\end{minipage}
						\end{center}}
%%%%%%%%%%%%%%%%%%%%%%%%%%%%%%%%%%%%%%%%%%
% PRINCIPE
%%%%%%%%%%%%%%%%%%%%%%%%%%%%%%%%%%%%%%%%%%

	\definecolor{fond_principe}{RGB}{255,245,245}
	\definecolor{bord_principe}{RGB}{0,0,0} 

	\newenvironment{principe}[1][]		{\begin{center}
							\begin{minipage}{\tailleDesBoites\linewidth}
								\begin{bclogo}[couleur=fond_principe,couleurBord=bord_principe,arrondi=0.2,logo=\bcplume]{Principe : \emph{#1}}}
						{		\end{bclogo}
							\end{minipage}
						\end{center}}


%%%%%%%%%%%%%%%%%%%%%%%%%%%%%%%%%%%%%%%%%%
% REMARQUES
%%%%%%%%%%%%%%%%%%%%%%%%%%%%%%%%%%%%%%%%%%

	\definecolor{fond_remarque}{RGB}{255,255,255}%{245,245,255} %Couleur du fond
	\definecolor{bord_remarque}{RGB}{200,200,200}

	\newcounter{cptRemarque}	%Compteur de remarques

	\newenvironmentx{remarque}[2][1=,2=]		{	\refstepcounter{cptRemarque}	%Incremente le compteur
							\begin{bclogo}[margeG=\margesBoites,margeD=\margesBoites,couleur=fond_remarque,couleurBord=bord_remarque,arrondi=0.2,logo=\bcinfo,#2]{Remarque \thecptRemarque\ : \emph{#1}}}
						{	\end{bclogo}
						}


						
						
	\newenvironmentx{remarques}[2][1=,2=]		{	\refstepcounter{cptRemarque}
							\begin{bclogo}[margeG=\margesBoites,margeD=\margesBoites,couleur=fond_remarque,couleurBord=bord_remarque,arrondi=0.2,logo=\bcinfo,#2]{Remarques \thecptRemarque\ : \emph{#1}}
								\begin{itemize}}
						{		\end{itemize}
							\end{bclogo}
						}

	\newenvironmentx{remarque*}[2][1=,2=]		{	\begin{bclogo}[margeG=\margesBoites,margeD=\margesBoites,couleur=fond_remarque,couleurBord=bord_remarque,arrondi=0.2,logo=\bcinfo,#2]{Remarque : \emph{#1}}}
						{	\end{bclogo}
						}

	\newenvironmentx{remarques*}[2][1=,2=]	{	\begin{bclogo}[margeG=\margesBoites,margeD=\margesBoites,couleur=fond_remarque,couleurBord=bord_remarque,arrondi=0.2,logo=\bcinfo,#2]{Remarques : \emph{#1}}
								\begin{itemize}}
						{		\end{itemize}
							\end{bclogo}
						}

%%%%%%%%%%%%%%%%%%%%%%%%%%%%%%%%%%%%%%%%%%
% ATTENTION
%%%%%%%%%%%%%%%%%%%%%%%%%%%%%%%%%%%%%%%%%%

	\definecolor{fond_attention}{RGB}{255,255,255}%{255,240,240} 
	\definecolor{bord_attention}{RGB}{255,200,200}%{255,240,240} 

	\newenvironment{attention}[1][]		{\begin{center}
							\begin{minipage}{\tailleDesBoites\linewidth}
								\begin{bclogo}[couleur=fond_attention,couleurBord=bord_attention,arrondi=0.2,logo=\bcattention]{Attention : \emph{#1}}}
						{		\end{bclogo}
							\end{minipage}
						\end{center}}


%%%%%%%%%%%%%%%%%%%%%%%%%%%%%%%%%%%%%%%%%%
% IMPORTANT
%%%%%%%%%%%%%%%%%%%%%%%%%%%%%%%%%%%%%%%%%%

	\definecolor{fond_important}{RGB}{255,255,255}%{255,240,240} 
	\definecolor{bord_important}{RGB}{255,200,200}%{255,240,240} 

	\newenvironment{important}[1][]		{\begin{center}
							\begin{minipage}{\tailleDesBoites\linewidth}
								\begin{bclogo}[couleur=fond_important,couleurBord=bord_important,arrondi=0.2,logo=\bcattention]{Important : \emph{#1}}}
						{		\end{bclogo}
							\end{minipage}
						\end{center}}


%%%%%%%%%%%%%%%%%%%%%%%%%%%%%%%%%%%%%%%%%%
% PROPRIETES
%%%%%%%%%%%%%%%%%%%%%%%%%%%%%%%%%%%%%%%%%%

	\definecolor{bord_propriete}{RGB}{200,200,200}
	\definecolor{fond_propriete}{RGB} {245,245,255}

	\newcounter{cptPropriete}

	\newenvironmentx{propriete}[2][1=,2=]		{\refstepcounter{cptPropriete}
							\begin{bclogo}[margeG=\margesBoites,margeD=\margesBoites,couleur=fond_propriete,couleurBord=bord_propriete,arrondi=0.2,logo=\bccrayon,#2]{Propriété \thecptPropriete\ : \emph{#1}}}
						{	\end{bclogo}
						}

	\newenvironmentx{proprietes}[2][1=,2=]	{	\refstepcounter{cptPropriete}
							\renewcommand{\descriptionlabel}[1] {	\hspace{\labelsep}
												$\bullet$
												\ifthenelse{\equal{##1}{}}
													{}
													{\textbf{##1} :}
											}
							\begin{bclogo}[margeG=\margesBoites,margeD=\margesBoites,couleur=fond_propriete,couleurBord=bord_propriete,arrondi=0.2,logo=\bccrayon,#2]{Propriétés \thecptPropriete\ : \emph{#1}}
								\begin{description}[leftmargin=2cm,style=nextline]}
						{		\end{description}
							\end{bclogo}}
							


	\newenvironmentx{propriete*}[2][1=,2=]	{	\begin{bclogo}[margeG=\margesBoites,margeD=\margesBoites,couleur=fond_propriete,couleurBord=bord_propriete,arrondi=0.2,logo=\bccrayon,#2]{Propriété : \emph{#1}}}
						{	\end{bclogo}
						}

	\newenvironmentx{proprietes*}[2][1=,2=]	{	\renewcommand{\descriptionlabel}[1] {	\hspace{\labelsep}
													$\bullet$
													\ifthenelse{\equal{##1}{}}
														{}
														{\textbf{##1} :}
												}
							\begin{bclogo}[couleur=fond_propriete,couleurBord=bord_propriete,arrondi=0.2,logo=\bccrayon,#2]{Propriétés : \emph{#1}}
								\begin{description}[leftmargin=2cm,style=nextline]}
						{		\end{description}
							\end{bclogo}
						}

%%%%%%%%%%%%%%%%%%%%%%%%%%%%%%%%%%%%%%%%%%
% THEOREME
%%%%%%%%%%%%%%%%%%%%%%%%%%%%%%%%%%%%%%%%%%

	\definecolor{bord_theoreme}{RGB}{200,200,200}
	\definecolor{fond_theoreme}{RGB} {245,245,255}

	\newcounter{cptTheoreme}

	\newenvironmentx*{theoreme}[2][1=,2=]		{\refstepcounter{cptTheoreme}
						\begin{center}
							\begin{minipage}{\tailleDesBoites\linewidth}
								\begin{bclogo}[couleur=fond_theoreme,couleurBord=bord_theoreme,arrondi=0.2,logo=\bccrayon,#2]{Théorème \thecptTheoreme\ : \emph{#1}}}
						{		\end{bclogo}
							\end{minipage}
						\end{center}}

	\newenvironmentx*{theoremes}[2][1=,2=]		{\refstepcounter{cptTheoreme}
						\begin{center}
							\begin{minipage}{\tailleDesBoites\linewidth}
							\renewcommand{\descriptionlabel}[1] {	\hspace{\labelsep}
													$\bullet$
													\ifthenelse{\equal{##1}{}}
														{}
														{\textbf{##1} :}
												}
								\begin{bclogo}[couleur=fond_theoreme,couleurBord=bord_theoreme,arrondi=0.2,logo=\bccrayon,#2]{Théorèmes \thecptTheoreme\ : \emph{#1}}
									\begin{description}[leftmargin=2cm]}
						{			\end{description}
								\end{bclogo}
							\end{minipage}
						\end{center}}

	\newenvironmentx*{theoreme*}[2][1=,2=]		{\begin{center}
							\begin{minipage}{\tailleDesBoites\linewidth}
								\begin{bclogo}[couleur=fond_theoreme,couleurBord=bord_theoreme,arrondi=0.2,logo=\bccrayon,#2]{Théorème : \emph{#1}}}
						{		\end{bclogo}
							\end{minipage}
						\end{center}}

	\newenvironmentx*{theoremes*}[2][1=,2=]	{\begin{center}
							\begin{minipage}{\tailleDesBoites\linewidth}
							\renewcommand{\descriptionlabel}[1] {	\hspace{\labelsep}
													$\bullet$
													\ifthenelse{\equal{##1}{}}
														{}
														{\textbf{##1} :}
												}
								\begin{bclogo}[couleur=fond_theoreme,couleurBord=bord_theoreme,arrondi=0.2,logo=\bccrayon,#2]{Théorèmes : \emph{#1}}
									\begin{description}[leftmargin=2cm]}
						{			\end{description}
								\end{bclogo}
							\end{minipage}
						\end{center}}

%%%%%%%%%%%%%%%%%%%%%%%%%%%%%%%%%%%%%%%%%%
% EXEMPLES
%%%%%%%%%%%%%%%%%%%%%%%%%%%%%%%%%%%%%%%%%%

	\definecolor{fond_exemple}{RGB}{255,255,255} %Couleur du fond
	\definecolor{bord_exemple}{RGB}{255,255,255} %Couleur du fond

	\newcounter{cptExemple}

	\newenvironmentx*{exemple}[2][1=,2=]		{	\refstepcounter{cptExemple}
							\begin{bclogo}[couleur=fond_exemple,couleurBord=bord_exemple,arrondi=0.2,logo=\bcbook,ombre=true,couleurOmbre = gray,blur,#2]{Exemple \thecptExemple\ : \emph{#1}}
						}
						{	\end{bclogo}
						}

	\newenvironmentx*{exemples}[2][1=,2=]		{	\refstepcounter{cptExemple}
							\begin{bclogo}[couleur=fond_exemple,couleurBord=bord_exemple,arrondi=0.2,logo=\bcbook,ombre=true,couleurOmbre = gray,blur,#2]{Exemples \thecptExemple\ : \emph{#1}}
								\begin{itemize}
						}
						{		\end{itemize}
							\end{bclogo}
						}

	\newenvironmentx*{exemple*}[2][1=,2=]		{	\begin{bclogo}[couleur=fond_exemple,couleurBord=bord_exemple,arrondi=0.2,logo=\bcbook,ombre=true,couleurOmbre = gray,blur,#2]{Exemple : \emph{#1}}
						}
						{	\end{bclogo}
						}

	\newenvironmentx*{exemples*}[2][1=,2=]		{	\begin{bclogo}[couleur=fond_exemple,couleurBord=bord_exemple,arrondi=0.2,logo=\bcbook,ombre=true,couleurOmbre = gray,blur,#2]{Exemples : \emph{#1}}
								\begin{itemize}
						}
						{		\end{itemize}
							\end{bclogo}
						}

%%%%%%%%%%%%%%%%%%%%%%%%%%%%%%%%%%%%%%%%%%
% ASTUCE
%%%%%%%%%%%%%%%%%%%%%%%%%%%%%%%%%%%%%%%%%%

	\definecolor{fond_astuce}{RGB}{255,255,255}%{250,250,250} %Couleur du fond
	\definecolor{bord_astuce}{RGB}{0,0,0}%{250,250,250} %Couleur du fond

	\newcommand*\bcastuce {\begingroup \includegraphics[width=0.5cm]{logos/friend.pdf}\endgroup}
	%\newcommand*\bcastuce {\includegraphics{./logos/friend}}

	\newcounter{cptAstuce}

	\newenvironmentx*{astuce}[2][1=,2=]	{\refstepcounter{cptAstuce}
						\begin{center}
							\begin{bclogo}[couleur=fond_exemple,couleurBord=bord_astuce,arrondi=0.2,logo=\bcastuce,#2]{Astuce \thecptAstuce\ : \emph{#1}}}
						{	\end{bclogo}
						\end{center}}

	\newenvironmentx*{astuces}[2][1=,2=]	{\refstepcounter{cptAstuce}
						\begin{center}
								\begin{bclogo}[couleur=fond_exemple,couleurBord=bord_astuce,arrondi=0.2,logo=\bcastuce,#2]{Astuces \thecptAstuce\ : \emph{#1}}
									\begin{itemize}}
						{			\end{itemize}
								\end{bclogo}
						\end{center}}

	\newenvironmentx*{astuce*}[2][1=,2=]	{\begin{center}
								\begin{bclogo}[couleur=fond_exemple,couleurBord=bord_astuce,arrondi=0.2,logo=\bcastuce,#2]{Astuce : \emph{#1}}}
						{		\end{bclogo}
						\end{center}}

	\newenvironmentx*{astuces*}[2][1=,2=]	{\begin{center}
								\begin{bclogo}[couleur=fond_exemple,couleurBord=bord_astuce,arrondi=0.2,logo=\bcastuce,#2]{Astuces : \emph{#1}}
									\begin{itemize}}
						{			\end{itemize}
								\end{bclogo}
						\end{center}}

%%%%%%%%%%%%%%%%%%%%%%%%%%%%%%%%%%%%%%%%%%
% DEMONSTRATION
%%%%%%%%%%%%%%%%%%%%%%%%%%%%%%%%%%%%%%%%%%

	\definecolor{fond_demo}{RGB}{255,255,255}%{250,250,250} %Couleur du fond
	\definecolor{bord_demo}{RGB}{0,0,0}%{250,250,250} %Couleur du fond

	\newcounter{cptDemonstration}

	\newenvironmentx*{demonstration}[2][1=,2=]	{
							\stepcounter{cptDemonstration}
							\begin{bclogo}[margeG=\margesBoites,margeD=\margesBoites,couleur=fond_demo,couleurBord=bord_demo,arrondi=0.2,logo=\bcpanchant,#2]{Démonstration \thecptDemonstration\ : \emph{#1}}}
						{	\end{bclogo}
						}

	\newenvironmentx*{demonstration*}[2][1=,2=]	{	\begin{bclogo}[margeG=\margesBoites,margeD=\margesBoites,couleur=fond_demo,couleurBord=bord_demo,arrondi=0.2,logo=\bcpanchant,#2]{Démonstration : \emph{#1}}}
						{	\end{bclogo}
						}

	\newenvironmentx*{demonstrations}[2][1=,2=]	{	\stepcounter{cptDemonstration}
							\begin{bclogo}[margeG=\margesBoites,margeD=\margesBoites,couleur=fond_demo,couleurBord=bord_demo,arrondi=0.2,logo=\bcpanchant,#2]{Démonstrations \thecptDemonstration\ : \emph{#1}}
								\begin{itemize}}
						{		\end{itemize}
							\end{bclogo}
						}

	\newenvironmentx*{demonstrations*}[2][1=,2=]	{	\begin{bclogo}[margeG=\margesBoites,margeD=\margesBoites,couleur=fond_demo,couleurBord=bord_demo,arrondi=0.2,logo=\bcpanchant,#2]{Démonstration : \emph{#1}}
								\begin{itemize}}
						{		\end{itemize}
							\end{bclogo}
						}


%%%%%%%%%%%%%%%%%%%%%%%%%%%%%%%%%%%%%%%%%%
% RESUME
%%%%%%%%%%%%%%%%%%%%%%%%%%%%%%%%%%%%%%%%%%

	\definecolor{fond_resume}{RGB}{255,255,255}%{250,250,250} %Couleur du fond
	\definecolor{bord_resume}{RGB}{0,0,0}%{250,250,250} %Couleur du fond

	\newcounter{cptResume}


	\newenvironmentx{resume}[2][1=,2=]	{\stepcounter{cptResume}
							\begin{bclogo}[margeG=\margesBoites,margeD=\margesBoites,couleur=fond_resume,couleurBord=bord_resume,arrondi=0.2,logo=\bcpanchant,#2]{Résumé \thecptResume\ : \emph{#1}}}
						{	\end{bclogo}
						}

	\newenvironmentx{resume*}[2][1=,2=]	{	\begin{bclogo}[margeG=\margesBoites,margeD=\margesBoites,couleur=fond_resume,couleurBord=bord_resume,arrondi=0.2,logo=\bcpanchant,#2]{Résumé : \emph{#1}}}
						{	\end{bclogo}
						}

%%%%%%%%%%%%%%%%%%%%%%%%%%%%%%%%%%%%%%%%%%
% MAKING-OF
%%%%%%%%%%%%%%%%%%%%%%%%%%%%%%%%%%%%%%%%%%

	%Booleen qui affiche les notes - commentaires
	\newboolean{makingOf}



	\newcommand*{\setMakingOfOn}{\setboolean{makingOf}{true}}	%Active le texte à trous
	\newcommand*{\setMakingOfOff}{\setboolean{makingOf}{false}}	%Active le texte à trous
	\setMakingOfOff

	\definecolor{fond_makingOf}{RGB}{230,255,230} %Couleur du fond
	\definecolor{bord_makingOf}{RGB}{0,0,0} %Couleur du bord
	\newcommand{\titreMakingOf}{}



	\newenvironment{makingOf}[1][]		{
						\renewcommand{\titreMakingOf}{#1}
						\setbox0=\vbox\bgroup
						}
						{
						\egroup
						\ifthenelse{\boolean{makingOf}}
							{
								\begin{bclogo}[couleur=fond_makingOf,couleurBord=bord_makingOf,arrondi=0.2,logo=\bcetoile]{Commentaire : \emph{\titreMakingOf}}
									\box0
								\end{bclogo}
							}
							{
							}
						}


%%%%%%%%%%%%%%%%%%%%%%%%%%%%%%%%%%%%%%%%%%
% CORRECTION 
%%%%%%%%%%%%%%%%%%%%%%%%%%%%%%%%%%%%%%%%%%

	%Booleen qui affiche les notes - commentaires
%	\newboolean{makingOf}


\newboolean{correction}

\newenvironment{correction}[1][]	{	\par
							%\ifthenelse{\boolean{texteATrou} \AND \NOT \boolean{makingOf}}
							%	{\vspace{#1}}
							%	{}
							\setbox1=\vbox
							\bgroup
						}
						{	\egroup
							\ifthenelse{\boolean{texteATrou}}
								{\ifthenelse{\boolean{makingOf}}
										{\textcolor{couleurTrou}{\box1}}
										{
										}%{\vphantom{\box1}}
								}
								{\box1}
							\par
						}


%\newcommand*{\setCorrectionOn}{\setboolean{correction}{true}
%				\specialcomment{correction}{\par\begingroup\color{red}}{\endgroup\par}}	%Active le texte à trous
%\newcommand*{\setCorrectionOff}{\setboolean{correction}{false}
%				\excludecomment{correction}}	%Active le texte à trous
%\setCorrectionOff
%\newcommand*{\reponse}[2][]{\ifthenelse{\boolean{correction} \OR \boolean{makingOf}}{{\color{red}#2}}{#1}}



%%%%%%%%%%%%%%%%%%%%%%%%%%%%%%%%%%%%%%%%%%
% TEXTE CACHE - TROU
%%%%%%%%%%%%%%%%%%%%%%%%%%%%%%%%%%%%%%%%%%

	\newboolean{texteATrou}
	\setboolean{texteATrou}{false}	%Petite condition qui choisit entre 2 formats d'image
	\newcommand*{\setTexteATrouOn}{\setboolean{texteATrou}{true}}	%Active le texte à trous
	\newcommand*{\setTexteATrouOff}{\setboolean{texteATrou}{false}}	%Désactive le texte à trous
	\definecolor{couleurTrou}{rgb}{0,0.5,0}

	\newcommandx*{\trou}[2][2=totolabricot]{%
				\ifthenelse{\boolean{texteATrou}}%
					{\ifthenelse{\boolean{makingOf}}%
						{\textcolor{couleurTrou}{#1}}
						%{\textcolor{white}{#1}}
						{
							\ifthenelse{\equal{#2}{totolabricot}}
								{\phantom{#1}}
								{#2}
						}
					}
					{#1}
				}%\phantom{#1}}}{#1}}

	\definecolor{couleurCache}{rgb}{0,0.5,0}
	\newcommand*{\texteATrouSetPhantom}{0}
	\newlength{\texteATrouEspace}

	%deprecié
	\newenvironment{texteCache}[1][]	{	\par
							%\ifthenelse{\boolean{texteATrou} \AND \NOT \boolean{makingOf}}
							%	{\vspace{#1}}
							%	{}
							\ifthenelse{\equal{#1}{}}
								{\renewcommand*{\texteATrouSetPhantom}{1}\setlength{\texteATrouEspace}{0em}}
								{\renewcommand*{\texteATrouSetPhantom}{0}\setlength{\texteATrouEspace}{#1}}
							
							\setbox1=\vbox
							\bgroup
						}
						{	\egroup
							\ifthenelse{\boolean{texteATrou}}
								{\ifthenelse{\boolean{makingOf}}
										{\textcolor{couleurTrou}{\box1}}
										{\ifthenelse{\equal{\texteATrouSetPhantom}{0}}
											{\vspace{\texteATrouEspace}}
											{\phantom{\box1}}
										}%{\vphantom{\box1}}
								}
								{\box1}
							\par
						}

	\newcommand*{\remplaceCache}[2]	{\ifthenelse{\boolean{texteATrou} \AND \NOT \boolean{makingOf}}{#1}{#2}}




%%%%%%%%%%%%%%%%%%%%%%%%%%%%%%%%%%%%%%%%%%
% VERSION PROF/CORRIGE
%%%%%%%%%%%%%%%%%%%%%%%%%%%%%%%%%%%%%%%%%%

	%Mode de compilation :
	\newboolean{versionProf}
	\setboolean{versionProf}{false}
	\newcommand*{\setVersionProf}	%Commande utilisée par "version"
		{
			\setboolean{versionProf}{true}
			\setboolean{versionCorrigee}{false}
			\setboolean{versionEleve}{false}
			\setboolean{texteATrou}{true}	%Ancien
			\setMakingOfOn	%Ancien
			\setboolean{afficheCorrection}{true}
			\definecolor{couleurCorrection}{RGB}{250,0,0}
		}

	\newboolean{versionCorrigee}
	\setboolean{versionCorrigee}{false}
	\newcommand*{\setVersionCorrigee}
		{
			\setboolean{versionProf}{false}
			\setboolean{versionCorrigee}{true}
			\setboolean{versionEleve}{false}
			\setboolean{texteATrou}{true}	%Ancien
			\setMakingOfOn	%Ancien
			\setboolean{afficheCorrection}{true}
			\definecolor{couleurCorrection}{RGB}{0,150,0}
		}

	\newboolean{versionEleve}
	\setboolean{versionEleve}{true}	%ACTIF PAR DEFAUT
	\newcommand*{\setVersionEleve}
		{
			\setboolean{versionProf}{false}
			\setboolean{versionCorrigee}{false}
			\setboolean{versionEleve}{true}
			\setboolean{texteATrou}{true}	%Ancien
			\setMakingOfOff	%Ancien
			\setboolean{afficheCorrection}{false}
		}
		
	%Les options associées :
	\newboolean{afficheCorrection}
	\setboolean{afficheCorrection}{false}
	
	\definecolor{couleurCorrection}{RGB}{0,0,0}%{250,250,250} %Couleur de la correction
	\newcommand*{\noColorCorrection}{\definecolor{couleurCorrection}{RGB}{0,0,0}}


	
	
	
		\newcommand*{\version}[1]
			{
				\ifthenelse{\equal{#1}{prof}}
					{
						\setVersionProf
					}
					{
						\ifthenelse{\equal{#1}{corrige} \OR \equal{#1}{corrigée} \OR \equal{#1}{corrigee} \OR \equal{#1}{corrigé} \OR \equal{#1}{correction}}
							{
								\setVersionCorrigee
							}
							{
								\setVersionEleve
							}
					}
			}

	\version{eleve}%Initialisation
	
%%%%%%%%%%%%%%%%%%%%%%%%%%%%%%%%%%%
% REPONSES
%%%%%%%%%%%%%%%%%%%%%%%%%%%%%%%%%%%

	\newcommandx*{\reponse}[2][2=\phantom{\#1}]{\ifthenelse{\boolean{afficheCorrection}}%
								{{\color{couleurCorrection}#1}}%
								{%
									%\ifthenelse{\equal{#2}{totolabricot}}%
									%	{\phantom{#1}}%
										{#2}%
								}%
							}

	\newenvironment{bigReponse}[1][totolabricot]
				{
					\newcommand{\totolabricot}{#1}
					\setbox1=\vbox
					\bgroup
				}
				{
					\egroup
					%\setbox2=\vsplit1to15cm   %Pour tenter de couper quand il y a plusieurs pages
					\ifthenelse{\boolean{afficheCorrection}}
							{	{\color{couleurCorrection}\box1}		}
							{
								\ifthenelse{\equal{\totolabricot}{totolabricot}}
									{\phantom{\box1}}
									{\totolabricot}
							}
				}




%%%%%%%%%%%%%%%%%%%%%%%%%%%%%%%%%%%%%%%%%%
% OBJECTIF
%%%%%%%%%%%%%%%%%%%%%%%%%%%%%%%%%%%%%%%%%%

	\definecolor{fond_objectif}{RGB}{255,255,255}%{250,250,250} %Couleur du fond
	\definecolor{bord_objectif}{RGB}{0,0,0}%{250,250,250} %Couleur du fond

	\newcounter{cptObjectif}
	\newcommand*\bccake {\includegraphics{logos/cake.pdf}}

	\newenvironment{objectif}[1][]		{\refstepcounter{cptObjectif}
						\begin{center}
							\begin{minipage}{0.95\linewidth}
								\begin{bclogo}[couleur=fond_objectif,couleurBord=bord_objectif,arrondi=0.2,logo=\bccake]{Objectif \thecptObjectif\ : \emph{#1}}}
						{		\end{bclogo}
							\end{minipage}
						\end{center}}

	\newenvironment{objectifs}[1][]		{\refstepcounter{cptObjectif}
						\begin{center}
							\begin{minipage}{0.95\linewidth}
								\begin{bclogo}[couleur=fond_objectif,couleurBord=bord_objectif,arrondi=0.2,logo=\bccake]{Objectifs \thecptObjectif\ : \emph{#1}}
									\begin{itemize}}
						{			\end{itemize}
								\end{bclogo}
							\end{minipage}
						\end{center}}

	\newenvironment{objectif*}[1][]		{\begin{center}
							\begin{minipage}{0.95\linewidth}
								\begin{bclogo}[couleur=fond_objectif,couleurBord=bord_objectif,arrondi=0.2,logo=\bccake]{Objectif : \emph{#1}}}
						{		\end{bclogo}
							\end{minipage}
						\end{center}}

	\newenvironment{objectifs*}[1][]	{\begin{center}
							\begin{minipage}{0.95\linewidth}
								\begin{bclogo}[couleur=fond_objectif,couleurBord=bord_objectif,arrondi=0.2,logo=\bccake]{Objectifs : \emph{#1}}
									\begin{itemize}}
						{			\end{itemize}
								\end{bclogo}
							\end{minipage}
						\end{center}}

%%%%%%%%%%%%%%%%%%%%%%%%%%%%%%%%%%%%%%%%%%
% EXERCICE
%%%%%%%%%%%%%%%%%%%%%%%%%%%%%%%%%%%%%%%%%%


	\definecolor{fond_exercice}{RGB}{250,250,250}
	\definecolor{bord_exercice}{RGB}{0,0,0} 


	\newcounter{cptExos}
%	\newcommand*{\exercice}[1]		{\refstepcounter{cptExos}
%						\section*{Exercice \thecptExos\ : #1}
%						\resetQuestions}


	\newcommand{\sectionExercice}[1]	{\refstepcounter{cptExos}
						\setcounter{subsection}{0}
						\section*{Exercice \thecptExos\ : #1}
						\resetQuestions}

    
	\newcommand{\subsectionExercice}[1]	{\refstepcounter{cptExos}
						\setcounter{subsection}{0}
						\subsection*{Exercice \thecptExos\ : #1}
						\resetQuestions}
                    
	\newcommand{\subsubsectionExercice}[1]	{\refstepcounter{cptExos}
						\setcounter{subsection}{0}
						\subsubsection*{Exercice \thecptExos\ : #1}
						\resetQuestions}
						
                    
% 	\newenvironment{exercice}[1][]		{%\refstepcounter{cptExos}
% 						%\setcounter{subsection}{0}
% 						%\section*{Exercice \thecptExos\ : #1}
% 						%\resetQuestions
% 						\sectionExercice{#1}}
% 						{%\clearpage
% 						\vspace{0.5cm}\hrule}
						
						
    \newenvironmentx{exercice}[2][1=,2=]	{
                            \refstepcounter{cptExos}
                            \resetQuestions
							\begin{bclogo}[margeG=\margesBoites,margeD=\margesBoites,couleur=fond_exercice,couleurBord=bord_exercice,arrondi=0.2,logo=\bcplume,#2]
                            {Exercice \thecptExos\ : \emph{#1}}
						}
						{	\end{bclogo}
						}

						

%%%%%%%%%%%%%%%%%%%%%%%%%%%%%%%%%%%%%%%%%%
% QUESTION
%%%%%%%%%%%%%%%%%%%%%%%%%%%%%%%%%%%%%%%%%%

	\newcounter{cptQuestions}
	
	\newboolean{afficheBareme}
	\setboolean{afficheBareme}{true}
	\newcommand*{\noBareme}{\setboolean{afficheBareme}{false}}
	\newcommand*{\yesBareme}{\setboolean{afficheBareme}{true}}
	
	\newcommand*{\resetQuestions}{\setcounter{cptQuestions}{0}}

	\newcommand*{\question}[2][]		{	\par	%Fini un paragraphe
 							\refstepcounter{cptQuestions}
 							\vspace{0.3cm}
 							\fbox{\textbf{Q\thecptQuestions .}}
 							\ifthenelse{\equal{#1}{}}
								{}
								{
									\bareme{#1}
 								} 
 							{\textbf{\itshape #2}}
 							\par\vspace{0.0cm}
						}

	\newenvironment{questions}[1][]		{	\itshape \vspace{0.3cm}
							%\addtocounter{cptQuestions}{-1}	%Désincrémenter le compteur
							%\global\advance\cptQuestions-\@ne
							%\def\labelitemi{\global\advance\cptQuestions\@ne Q\the\cptQuestions.}
							\renewcommand*{\labelitemi}{\refstepcounter{cptQuestions}\fbox{Q\thecptQuestions.}}
							\begin{itemize}
						}
						{
							\end{itemize}\vspace{0.3cm}
						}

	\newenvironment{question*}[1][]		{	\par	%Fini un paragraphe
							\itshape\vspace{0.3cm}
							\fbox{Q.} 
						}
						{
							\par\vspace{0.3cm}
						}


	\newenvironment{questions*}[1][]	{

						\itshape \vspace{0.3cm}
						\renewcommand*{\labelitemi}{\fbox{Q.}}
						\begin{itemize}
					}
					{
						\end{itemize}\vspace{0.3cm}
					}

	\newcounter{cptQuestionsSection}[section] %Pour les question dont le numro fait intervenir les seciotns
	\newcommand*{\questionSection}[1]	{	\par	%Fini un paragraphe
							\refstepcounter{cptQuestionsSection}
							%\global\advance\cptQuestions\@ne	%Incrmente le compteur
							\vspace{0.3cm}
							\fbox{\gras{\thesection.\thecptQuestionsSection}} 
							{\itshape #1 }
							\par\vspace{0.3cm}
						}







%%%%%%%%%%%%%%%%%%%%%%%%%%%%%%%%%%%%%%%%%%
% DOCUMENT RESSOURCE
%%%%%%%%%%%%%%%%%%%%%%%%%%%%%%%%%%%%%%%%%%


	\newcounter{cptDocRessource}

	\newenvironment{docRessource}[1][]	{	\renewcommand*{\thecptDocRessource}{DOC-\arabic{cptDocRessource}}
							\refstepcounter{cptDocRessource}
							\clearpage
							\newpage %Nouvelle page
							\appendix
							\fbox{\huge{\thecptDocRessource}}
							
						}
						{	
							\clearpage
							\newpage
						}

%%%%%%%%%%%%%%%%%%%%%%%%%%%%%%%%%%%%%%%%%%
% DOCUMENT REPONSE
%%%%%%%%%%%%%%%%%%%%%%%%%%%%%%%%%%%%%%%%%%

	\newcounter{cptDocReponse}

	\newenvironment{docReponse}	{	\renewcommand*{\thecptDocReponse}{DR-\arabic{cptDocReponse}}
							\refstepcounter{cptDocReponse}
							\clearpage
							\newpage %Nouvelle page
							\noindent
							\begin{tabular}{cc}
							\noindent
								\begin{minipage}{0.3\linewidth}
									\fbox{\huge{\thecptDocReponse}}
								\end{minipage}
								&
								\begin{minipage}{0.5\linewidth}
									\begin{flushright}\fbox{\huge{Nom : \hspace{6cm}}}\end{flushright}
								\end{minipage}
							\end{tabular}
							
						}
						{
							\clearpage
							\newpage
						}

\newenvironment{boiteReponse}[1][totolabricot]
				{
					\newcommand{\totolabricote}{#1}
					\setbox1=\vbox
					\bgroup
				}
				{
					\egroup
					%\setbox2=\vsplit1to15cm   %Pour tenter de couper quand il y a plusieurs pages
					\noindent\fbox{\ifthenelse{\boolean{afficheCorrection}}
							{	{\color{couleurCorrection}\box1}		}
							{
								\ifthenelse{\equal{\totolabricote}{totolabricot}}
									{\phantom{\box1}}
									{\totolabricote}
							}
							\box1}
				}
						
%%%%%%%%%%%%%%%%%%%%%%%%%%%%%%%%%%%%%%%%%%
% MODELE
%%%%%%%%%%%%%%%%%%%%%%%%%%%%%%%%%%%%%%%%%%

	\definecolor{fond_modele}{RGB}{250,250,250} %Couleur du fond
	\definecolor{bord_modele}{RGB}{0,0,0} %Couleur du fond

	\newcounter{cptModele}

	\newenvironment*{modele}[1][]		{\refstepcounter{cptModele}
						\begin{center}
							\begin{minipage}{0.95\linewidth}
								\begin{bclogo}[couleur=fond_modele,couleurBord=bord_modele,arrondi=0.2,logo=\bcicosaedre]{Modèle \thecptModele\ : \emph{#1}}}
						{		\end{bclogo}
							\end{minipage}
						\end{center}}

	\newenvironment*{modele*}[1][]		{\begin{center}
							\begin{minipage}{0.95\linewidth}
								\begin{bclogo}[couleur=fond_modele,couleurBord=bord_modele,arrondi=0.2,logo=\bcicosaedre]{Modèle : \emph{#1}}}
						{		\end{bclogo}
							\end{minipage}
						\end{center}}

%%%%%%%%%%%%%%%%%%%%%%%%%%%%%%%%%%%%%%%%%%
% METHODE
%%%%%%%%%%%%%%%%%%%%%%%%%%%%%%%%%%%%%%%%%%

	\definecolor{fond_methode}{RGB}{250,250,250} %Couleur du fond
	\definecolor{bord_methode}{RGB}{0,0,0} %Couleur du fond

	\newcounter{cptMethode}


	\newenvironmentx*{methode}[2][1=,2=]	{	\refstepcounter{cptMethode}
							\begin{bclogo}[margeG=\margesBoites,margeD=\margesBoites,couleur=fond_methode,couleurBord=bord_methode,arrondi=0.2,logo=\bclampe,#2]{Méthode \thecptMethode\ : \emph{#1}}
						}
						{	\end{bclogo}
						}


	\newenvironmentx*{methode*}[2][1=,2=]	{	\begin{bclogo}[margeG=\margesBoites,margeD=\margesBoites,couleur=fond_methode,couleurBord=bord_methode,arrondi=0.2,logo=\bclampe,#2]{Méthode : \emph{#1}}}
						{	\end{bclogo}
						}

	\newenvironmentx*{methodes}[2][1=,2=]		{	\refstepcounter{cptMethode}
								\renewcommand{\descriptionlabel}[1] {	\hspace{\labelsep}
												$\bullet$
												\ifthenelse{\equal{##1}{}}
													{}
													{\textbf{##1} :}
											}
							\begin{bclogo}[margeG=\margesBoites,margeD=\margesBoites,couleur=fond_methode,couleurBord=bord_methode,arrondi=0.2,logo=\bclampe]{Méthodes \thecptMethode : \emph{#1}}
								\begin{description}}
						{		\end{description}
							\end{bclogo}
						}

	\newenvironmentx*{methodes*}[2][1=,2=]	{		\renewcommand{\descriptionlabel}[1] {	\hspace{\labelsep}
												$\bullet$
												\ifthenelse{\equal{##1}{}}
													{}
													{\textbf{##1} :}
											}	\begin{bclogo}[margeG=\margesBoites,margeD=\margesBoites,couleur=fond_methode,couleurBord=bord_methode,arrondi=0.2,logo=\bclampe]{Méthodes : \emph{#1}}
								\begin{description}}
						{		\end{description}
							\end{bclogo}
						}



%%%%%%%%%%%%%%%%%%%%%%%%%%%%%%%%%%%%%%%
%BAREME
%%%%%%%%%%%%%%%%%%%%%%%%%%%%%%%%%%%%%%%

\reversemarginpar
\setlength{\marginparwidth}{0.7cm}
\newcommand{\bareme}[1]{\ifthenelse{\boolean{afficheBareme}}{\marginpar{(/#1)}}{}}




% CODAGE

%%%%%%%%%%%%%%%%%%%%%%%%%%%%%
% Code python
%%%%%%%%%%%%%%%%%%%%%%%%%%%%

\definecolor{PY-fond}{rgb}{1,1,1}			%Couleur du fond
\definecolor{PY-commentaires}{rgb}{0.5,0.5,0.5}	%Couleur des commentaires
\definecolor{PY-chaines}{rgb}{0,0.63,0}		%Couleur des chaines de caracteres
\definecolor{PY-fonctions}{rgb}{0,0,1}		%Couleur des fonctions de bases
\definecolor{PY-decoration}{rgb}{0.5,0.5,0.5}		%Couleur des fonctions autre (??)
\definecolor{PY-self}{rgb}{0,0,0}		%je sais pas. J'ai recopié d'internet
\definecolor{PY-numeros}{rgb}{0.7,0.7,0.7}		%numéros de ligne
\definecolor{PY-chevrons}{rgb}{1,0.,0.}		%chevrons
\definecolor{PY-ombre}{rgb}{0.,0.,0.}		%Ombre




%\makeatletter
%\preto{\@verbatim}{\topsep=0pt \partopsep=0pt }
%\makeatother

\lstnewenvironment{Python}[1][0.975\linewidth]
		{
         \lstset{
                upquote=true,
                columns=flexible,
                basicstyle=\ttfamily,
                language=Python,
                keywordstyle=\color{blue},
                identifierstyle=\color{black},
                commentstyle=\color{orange},
            % *** titre
                title={\hspace{-2.2em}\color{black}\raggedright \hspace{1cm} Code Python :},
                belowcaptionskip=0.5em, 
                abovecaptionskip=-0.0em,
                nolol=true,
            % *** style
                frame=leftline,%
                framerule=3pt,%
                rulecolor=\color{black!20},%
                framesep=2em,%
                xleftmargin=2em,%
                linewidth=15cm,
            %Numeros de ligne
				numbers=left,	%Rajoute des numéros de ligne
				numberstyle=\footnotesize\color{black!20},	%Taille des numéros de ligne
				numbersep=1em,			%Marge entre le code et les numéros de ligne
            % code
                showstringspaces=false,
                texcl=True,% to activate teX code
            literate=%
                {á}{{\'a}}1 {é}{{\'e}}1 {í}{{\'i}}1 {ó}{{\'o}}1 {ú}{{\'u}}1
                {Á}{{\'A}}1 {É}{{\'E}}1 {Í}{{\'I}}1 {Ó}{{\'O}}1 {Ú}{{\'U}}1
                {à}{{\`a}}1 {è}{{\`e}}1 {ì}{{\`i}}1 {ò}{{\`o}}1 {ù}{{\`u}}1
                {À}{{\`A}}1 {È}{{\'E}}1 {Ì}{{\`I}}1 {Ò}{{\`O}}1 {Ù}{{\`U}}1
                {ä}{{\"a}}1 {ë}{{\"e}}1 {ï}{{\"i}}1 {ö}{{\"o}}1 {ü}{{\"u}}1
                {Ä}{{\"A}}1 {Ë}{{\"E}}1 {Ï}{{\"I}}1 {Ö}{{\"O}}1 {Ü}{{\"U}}1
                {â}{{\^a}}1 {ê}{{\^e}}1 {î}{{\^i}}1 {ô}{{\^o}}1 {û}{{\^u}}1
                {Â}{{\^A}}1 {Ê}{{\^E}}1 {Î}{{\^I}}1 {Ô}{{\^O}}1 {Û}{{\^U}}1
                {œ}{{\oe}}1 {Œ}{{\OE}}1 {æ}{{\ae}}1 {Æ}{{\AE}}1 {ß}{{\ss}}1
                {?}{{\H{u}}}1 {?}{{\H{U}}}1 {?}{{\H{o}}}1 {?}{{\H{O}}}1
                {ç}{{\c c}}1 {Ç}{{\c C}}1 {ø}{{\o}}1 {å}{{\r a}}1 {Å}{{\r A}}1
                {€}{{\EUR}}1 {£}{{\pounds}}1
                {°}{{\degres{}}}1
            }
			%\minipage{#1}
		}
		{	%\endminipage%
		}
		%\BeforeBeginEnvironment{codePython}{\par\noindent}
		%\AfterEndEnvironment{codePython}{\par}

%%%%%%%%%%%%%%%%%%%%%%%%%%%%
% Code Pseudo Code
%%%%%%%%%%%%%%%%%%%%%%%%%%%%

\definecolor{PSD-fond}{rgb}{0.9,1,0.9}			%Couleur du fond
\definecolor{PSD-commentaires}{rgb}{0.5,0.5,0.5}	%Couleur des commentaires
\definecolor{PSD-chaines}{rgb}{0,0.63,0}		%Couleur des chaines de caracteres
\definecolor{PSD-fonctions}{rgb}{0,0,1}		%Couleur des fonctions de bases
\definecolor{PSD-decoration}{rgb}{0.5,0.5,0.5}		%Couleur des fonctions autre (??)
\definecolor{PSD-self}{rgb}{0,0,0}		%je sais pas. J'ai recopié d'internet
\definecolor{PSD-numeros}{rgb}{0.6,0.6,0.6}		%numéros de ligne
\definecolor{PSD-chevrons}{rgb}{1,0.,0.}		%chevrons
\definecolor{PSD-ombre}{rgb}{0.5,0.5,0.5}		%Ombre
\lstnewenvironment{pseudoCode}[1][\linewidth]
		{
		\lstset{
			%Langage
				%language=Python,	%Langage par défaut
			%Mise en forme de l'environnement
				xleftmargin=2em,	%marge a gauche
				%frame=trBL,		%Cadre (double cadre)
				backgroundcolor=\color{PSD-fond},	%Couleur d'arriere plan
				framexleftmargin=5mm,
				%frame=shadowbox,
				rulesepcolor=\color{PSD-ombre},
				texcl=true,	%Activle les commandes LaTex dans le code
				mathescape=true,	
				%escapechar=ø,	%Caractere a echapper
				escapeinside={(*}{*)},
				linewidth=#1,	%taille d'une ligne
			%Titre
				title={\hspace{-2.2em}\color{black}\raggedright \hspace{1cm} Pseudo-code :},
				belowcaptionskip=-0.em,
				abovecaptionskip=-0.0em,
				nolol=true,
				%title={Pseudo-code}
			%Numeros de ligne
				numbers=left,	%Rajoute des numéros de ligne
				numberstyle=\footnotesize\color{PSD-numeros},	%Taille des numéros de ligne
				numbersep=1em,			%Marge entre le code et les numéros de ligne
			%Mise en forme du texte
				columns=fixed,		%Largeur des lettres (fixed, flexible, fullflexible)
				basicstyle=\ttfamily\footnotesize,	%Mise en forme de base
				showspaces=false,	%Pour mettre des especes d'underscore sur les espaces
				tabsize=4,	%Taille des tabulations	
			%Commentaires
				commentstyle=\color{PSD-commentaires}\slshape,	%Couleur des commentaires
			%Chaine de caracteres
				stringstyle=\color{PSD-chaines},
				showstringspaces=false,
				morecomment=[s][\color{PSD-chaines}]{"""}{"""},
				morecomment=[s][\color{PSD-chaines}]{'''}{'''},
				comment=[l]{\#},
			%Fonctions et mot clé de python
				morekeywords=[1]{Pour,POUR,pour,allant,Allant,ALLANT,de,DE,De,jusqu,JUSQU,Jusqu,à,À,FIN,Fin,fin,Effectuer,EFFECTUER,effectuer,si,Si,SI,faire,et,Et,ET,ou,Ou,OU,FAIRE,Faire,alors,Alors,ALORS,sinon,Sinon,SINON,fonction,Fonction,FONCTION,retourner,Retourner,RETOURNER,afficher,Afficher,AFFICHER,concaténer,concatener,Concaténer,Concatener,CONCATÉNER,CONCATENER,nouveau,Nouveau,NOUVEAU,declarer,Declarer,DECLARER,déclarer,Déclarer,DÉCLARER,repeter,répéter,Repeter,Répéter,RÉPÉTER,finsi,Finsi,FinSi,FINSI,finpour,Finpour,FinPour,FINPOUR,debut,Debut,DEBUT,début,Début,DÉBUT,ouvrir,Ouvrir,OUVRIR,fermer,Fermer,FERMER,lire,Lire,LIRE,calculer,Calculer,CALCULER},
				keywordstyle=[1]*\color{PSD-fonctions}\bfseries,
				emph={self},
				emphstyle={\color{PSD-self}\slshape}
				%
            literate=%
                {á}{{\'a}}1 {é}{{\'e}}1 {í}{{\'i}}1 {ó}{{\'o}}1 {ú}{{\'u}}1
                {Á}{{\'A}}1 {É}{{\'E}}1 {Í}{{\'I}}1 {Ó}{{\'O}}1 {Ú}{{\'U}}1
                {à}{{\`a}}1 {è}{{\`e}}1 {ì}{{\`i}}1 {ò}{{\`o}}1 {ù}{{\`u}}1
                {À}{{\`A}}1 {È}{{\'E}}1 {Ì}{{\`I}}1 {Ò}{{\`O}}1 {Ù}{{\`U}}1
                {ä}{{\"a}}1 {ë}{{\"e}}1 {ï}{{\"i}}1 {ö}{{\"o}}1 {ü}{{\"u}}1
                {Ä}{{\"A}}1 {Ë}{{\"E}}1 {Ï}{{\"I}}1 {Ö}{{\"O}}1 {Ü}{{\"U}}1
                {â}{{\^a}}1 {ê}{{\^e}}1 {î}{{\^i}}1 {ô}{{\^o}}1 {û}{{\^u}}1
                {Â}{{\^A}}1 {Ê}{{\^E}}1 {Î}{{\^I}}1 {Ô}{{\^O}}1 {Û}{{\^U}}1
                {œ}{{\oe}}1 {Œ}{{\OE}}1 {æ}{{\ae}}1 {Æ}{{\AE}}1 {ß}{{\ss}}1
                {?}{{\H{u}}}1 {?}{{\H{U}}}1 {?}{{\H{o}}}1 {?}{{\H{O}}}1
                {ç}{{\c c}}1 {Ç}{{\c C}}1 {ø}{{\o}}1 {å}{{\r a}}1 {Å}{{\r A}}1
                {€}{{\EUR}}1 {£}{{\pounds}}1
                {°}{{\degres{}}}1
			}
			\minipage{#1}
		}
		{	\endminipage
		}
		\BeforeBeginEnvironment{pseudoCode}{\par\noindent}
		\AfterEndEnvironment{pseudoCode}{\par}

\endinput
