\documentclass[a4paper,12pt]{article}

	\usepackage{xcolor}
	\usepackage{listings}	\lstset{language=[LaTeX]TeX,basicstyle=\ttfamily,texcsstyle=*\color{blue},identifierstyle=\color{brown},commentstyle=\color{gray}\itshape,escapechar=!,moretexcs={}}
	\usepackage[utf8]{inputenc}
	\usepackage[T1]{fontenc}
	\usepackage{hyperref}
	\usepackage{Raf_Competences}


	\newcommand{\rac}{({\color{red}Raccourci})}
	\newcommand{\ren}{({\color{blue}Renommé})}

\begin{document}

	\begin{center}
		\hrule{\Large Diagramme de compétences en sciences de l'ingénieur}\\15/08/15\\\hrule
	\end{center}


	
	Ce package permet de dessiner des diagrammes représentant les compétences attendues en sciences de l'ingénieur dans les classes préparatoires (voir les \href{http://www.enseignementsup-recherche.gouv.fr/pid20536/bulletin-officiel.html?cid_bo=71640&cbo=1}{programmes}).
	Il permet en outre de mettre en évidence telles ou telles compétences.
	
	\section{Packages requis}
	%-------------------------------------

		\begin{itemize}
			\item \href{http://www.ctan.org/pkg/ifthen}{\textbf{ifthen}} : Package pour faire des compilations conditionnelles (if...then...else....)
			%\item \href{http://www.ctan.org/pkg/amsmath}{\textbf{amsmath}} : Pour des notations mathématiques (notamment l'utilisation de \verb!\text! il me semble).
			\item \href{http://www.ctan.org/pkg/pgf}{\textbf{tikz}} : Package pour faire des dessins (avec library \verb!calc!)
			%\item \href{http://www.ctan.org/pkg/calc}{\textbf{calc}} : Permet de faire des petits calculs au moment de la compilation
			\item \href{http://www.ctan.org/pkg/xargs}{\textbf{xargs}} : Pour créer des commandes avec plusieurs arguments optionnels
			%\item \href{http://www.ctan.org/pkg/array}{\textbf{array}} : Package qui rajoute des possibilités aux tableaux.
			%\item \textbf{circuitikz} : Pour tracer des circuits électrique avec tikz. Les options \emph{european} et \emph{cute inductors} sont activées.
			\item \href{https://www.ctan.org/pkg/xstring}{\textbf{xstring}} : Pour jouer avec les chaînes de caractères
		\end{itemize}
		
		
	\section{Appel du package}
	%-------------------------------------

		Le package est appelé en début de document par la commande :
		\begin{verbatim}
\usepackage{Raf_Competences}
		\end{verbatim}

	\section{Utilisation}
	%-------------------------------------
	
	La principale commande à utiliser est la commande \verb!\diagrammeCompetences!, ne contenant que des arguments optionnels :
	\begin{center}
		\verb!\diagrammeCompetences[<competences>][<mise en valeur>]!
	\end{center}
 	où :
	\begin{itemize}
		\item\verb!<competences>! est une liste de mots (les compétences...) séparés par des virgules.
					Par défaut, c'est la liste des compétences attendues en PTSI/PT qui est utilisée.
		\item\verb!<mise en valeur>! est la liste des compétences que l'on souhaite mettre en valeurs. C'est une liste de chiffre (on ne peux donc aller que jusqu'à 9 compétences pour le moment...).
	\end{itemize}

	\paragraph{Exemple :}
	\begin{verbatim}
\begin{center}
\diagrammeCompetences[Compétence 1,Compétence 2,Compétence 3][2]
\end{center}
	\end{verbatim}

\begin{center}
\diagrammeCompetences[Compétence 1,Compétence 2,Compétence 3][2]
\end{center}



	\section{Compétences pré-programmées}
	%---------------------------------------

		Certaines listes de compétences sont déjà pré-implantées, et accessible au travers de commandes décrites ci-dessous. En voici la liste :
		\begin{itemize}
			\item \verb!\competencesPTSI!
			\item \verb!\competencesPT!
			\item \verb!\competencesTSI!
			\item \verb!\competencesPCSI!
			\item \verb!\competencesPSI!
			\item \verb!\competencesMPSI!
			\item \verb!\competencesMP!
		\end{itemize}



		\subsection{PTSI, PT, TSI}
		%...............................


			\begin{verbatim}
\begin{center}
\diagrammeCompetences		%(Par défaut)
%\diagrammeCompetences[\competencesPTSI]
%\diagrammeCompetences[\competencesPT]
%\diagrammeCompetences[\competencesTSI]
\end{center}
			\end{verbatim}

			\begin{center}
				\diagrammeCompetences
			\end{center}


		\subsection{PSCI, PSI}
		%...............................

			\begin{verbatim}
\begin{center}
\diagrammeCompetences[\competencesPCSI]
%\diagrammeCompetences[\competencesPSI]
\end{center}
			\end{verbatim}

			\begin{center}
				\diagrammeCompetences[\competencesPSI]
			\end{center}

		\subsection{MPSI, MP}
		%...............................

			\begin{verbatim}
\begin{center}
\diagrammeCompetences[\competencesMPSI]
%\diagrammeCompetences[\competencesMP]
\end{center}
			\end{verbatim}

			\begin{center}
				\diagrammeCompetences[\competencesMP]
			\end{center}




	\section{Mise en évidences de compétences...}
	%------------------------------------------------

		Les compétences sont numérotées de haut en bas, d'abord à gauche puis à droite.

		\begin{center}
			\diagrammeCompetences[1,2,3,4,5,6,7,...]
		\end{center}

		Pour mettre en évidence une ou plusieurs compétences, il suffit de lister les numéros des compétences en second argument optionnel.

		\paragraph{Exemple :}
\begin{verbatim}
\begin{center}
\diagrammeCompetences[\competencesPT][2,5,7]
\end{center}
\end{verbatim}

		\begin{center}
			\diagrammeCompetences[\competencesPT][2,5,7]
		\end{center}

		\paragraph{Note :} Ne fonctionne que jusqu'à 9 compétences...
		\paragraph{Note-2 :} Comme c'est le second argument optionnel, cela implique qu'il faille mettre les compétences avant en 1er argument\dots
		\paragraph{Note-3 :} Les autres compétences sont un peu effacées pour accentuer la mise en avant...



	\section{Personnalisation}
	%---------------------------------------


		\subsection{Changement de taille}
		%...............................

			Toutes les dimensions sont relatives.
			Il suffit donc de précéder la commande, par exemple, d'une commande de taille (\verb!\footnotesize!, \verb!\scriptsize!,  \dots)

			\begin{verbatim}
\begin{center}
\scriptsize
\diagrammeCompetences
\end{center}
			\end{verbatim}

		\begin{center}
			\scriptsize
			\diagrammeCompetences
		\end{center}



		\subsection{Mettre ses propres compétences}
		%...................................................


			\begin{verbatim}
\begin{center}
\diagrammeCompetences[S'assoire,Se lever,Se coucher]
\end{center}
			\end{verbatim}

			\begin{center}
				\diagrammeCompetences[S'assoire,Se lever,Se coucher]
			\end{center}


		\subsection{Les paramètres de style}
		%...................................................


			Des commandes ont été crées par le packages, contenant les paramètres dimensionnels du diagramme.
			Il est possible de les modifier via la fonction \verb!renewcommand!, avant l'appel du diagramme.
			En voici la liste :

			\begin{itemize}
				\item \verb!\competencesTexteCentral! : Texte qui est écrit au centre du diagramme (`` \emph{Compétences} '' par défaut).
				\item \verb!\competencesDelimiteur! : Le délimiteur dans la liste des compétences (une virgule, par défaut).
				\item \verb!\competencesRayon! : La distance entre le centre et la case de chaque compétences ($10em$ par défaut. Je sais : j'aurais du utiliser une longueur et non une macro. Tant pis)
				\item \verb!\competencesIncrement! : L'angle qui sépare 2 compétences l'une sous l'autre ($20°$ par défaut)
			\end{itemize}

			\paragraph{Exemple :}

			\begin{verbatim}
\begin{center}
\renewcommand{\competencesTexteCentral}{Toto}
\renewcommand{\competencesDelimiteur}{-}
\renewcommand{\competencesRayon}{15em}
\renewcommand{\competencesIncrement}{60}
\diagrammeCompetences[compétence 1-compétence 2-compétence 3]
\end{center}
			\end{verbatim}

			\begin{center}
				\renewcommand{\competencesTexteCentral}{Toto}
				\renewcommand{\competencesDelimiteur}{-}
				\renewcommand{\competencesRayon}{15em}
				\renewcommand{\competencesIncrement}{60}
				\diagrammeCompetences[compétence 1-compétence 2-compétence 3]
			\end{center}


			Il est également possible de modifier certaines couleurs avec les fonctions usuelles de \verb!xcolor! :

			\begin{itemize}
				\item \verb!competencesCouleurFondCentrale! : Couleur de l'arrière plan la boite centrale.
				\item \verb!competencesCouleurTexteCentrale! : Couleur du texte central.
				\item \verb!competencesCouleurFondFeuille! : Couleur d'arrière plan des boites de compétence.
				\item \verb!competencesCouleurFondFeuilleSelected! : idem, pour les compétences mises en évidence.
				\item \verb!competencesCouleurTexteFeuille! : Couleur du textes des compétences.
				\item \verb!competencesCouleurTexteFeuilleSelected! : Idem pour le texte mis en évidence
				\item \verb!competencesCouleurTexteFeuilleUnselected! : Idem pour le texte non-mis en évidence.
			\end{itemize}

			Bien sur, j'aurais pu paramétrer plus de choses, mais je n'ai pas pris le temps de le faire et je le rajouterai selon la demande...
			
	\section{Problèmes rencontrés}
	%---------------------------------------

		J'ai eu des soucis avec les accents français dans tikz.
		Penser à bien mettre les packages :

			\begin{verbatim}
\usepackage[utf8]{inputenc}
\usepackage[T1]{fontenc}
			\end{verbatim}
\end{document}
