\documentclass[a4paper,10pt]{article}

	\usepackage{xcolor}
	\usepackage{listings}	\lstset{language=[LaTeX]TeX,basicstyle=\ttfamily,texcsstyle=*\color{blue},identifierstyle=\color{brown},commentstyle=\color{gray}\itshape,escapechar=!,moretexcs={}}
	\usepackage[latin1]{inputenc}
	\usepackage{hyperref}


	\usepackage{Raf_Notations_Dynamique}
	\usepackage[francais]{babel}
	\everymath{\displaystyle}


	\newcommand{\rac}{({\color{red}Raccourci})}
	\newcommand{\ren}{({\color{blue}Renomm�})}

\begin{document}





	\begin{center}
		\hrule{\Large Notations de Dynamique}\\\hrule
	\end{center}

	(Version du 01/08/18)

	\section{Packages requis}
	%-------------------------------------

	\begin{itemize}
		\item \href{http://www.ctan.org/pkg/ifthen}{\textbf{ifthen}} : Package permettant une compilation � choix multiple,
		%\item \href{http://tug.ctan.org/tex-archive/macros/latex/contrib/xargs}{\textbf{xarg}} : Package permettant de cr�er des commandes � plusieurs arguments optionnels.
		%\item \href{http://www.ams.org/publications/authors/tex/amsfonts}{\textbf{amsfonts}} : Package qui ajoute des polices d'�critures math�matiques.
		%\item \href{http://www.ams.org/publications/authors/tex/amslatex}{\textbf{amsmath}} : Package qui ajoute des fonctions math�matiques non-standards.
		\item \href{http://www.ctan.org/pkg/mathrsfs}{\textbf{mathrsfs}} : Package qui rajoute des polices d'�critures math�matiques.
		%\item \href{http://www.ctan.org/pkg/color}{\textbf{color}} : Package permettant de mettre en couleur du texte, des lignes, etc.
		%\item \href{http://www.ctan.org/pkg/xspace}{\textbf{xspace}} : Package permettant de mettre des espaces apr�s les commandes.
		%\item \href{http://www.ctan.org/pkg/xstring}{\textbf{xstring}} : Package permettant travailler sur les cha�nes de caract�res (chercher/remplacer, etc.)
		\item \href{http://enseignement.allais.eu/page-latex}{\textbf{Raf\_Notations\_Actions-Meca}} : Package de notations d'actions m�caniques.
		\item \href{http://enseignement.allais.eu/page-latex}{\textbf{Raf\_Notations\_Torseurs}} : Package de notations des torseurs.
	\end{itemize}

	\section{Appel du package}
	%-------------------------------------

	Le package est appel� en d�but de document par la commande :
	\begin{verbatim}
\usepackage{Raf_Notations_Dynamique}
	\end{verbatim}

	Par d�faut, ce package utilise un certain nombre de notations raccourcies, susceptibles de rentrer en conflit avec d'autre package (mais tellement plus rapide � taper !).
	De plus, certaines commandes ont �t� rebaptis�e.
	Ces raccourcis et renommages seront cit�s (\rac\ ou \ren) dans les tableaux suivants.
	Pour ne pas cr�er ces raccourcis/renommage, il faut rentre l'option \verb!noRaccourci! � l'appel du package.

	\begin{verbatim}
usepackage[noRaccourci]{Raf_Notations_Dynamique}
	\end{verbatim}

	\section{Masse}
	%-------------------------------------------
	\noindent
	\begin{tabular}{|p{0.35\linewidth}|p{0.3\linewidth}|p{0.3\linewidth}|}
		\hline
			\textbf{Commandes}&\textbf{Rendus}&\textbf{Commentaires}
		\\\hline\hline
			\verb!\ddm!			&	\ddm			&	Masse �l�mentaire
		\\\hline
	\end{tabular}
	
	
	\section{Inertie}
	%-------------------------------------------
	\noindent
	\begin{tabular}{|p{0.35\linewidth}|p{0.3\linewidth}|p{0.3\linewidth}|}
		\hline
			\textbf{Commandes}&\textbf{Rendus}&\textbf{Commentaires}
		\\\hline\hline
			\verb!\matInertie{P}{S}!			&	\matInertie{P}{S}			&	Matrice d'inertie.
		\\\hline
			\verb!\IGS!			&	\IGS		&	Matrice d'inertie au point $G$ de $S$.
		\\\hline
			\verb!\matInertieComposantes! \verb!{G}{1&2&3\\4&5&6\\7&8! \verb!&9}{R}!			&	\matInertieComposantes{G}{1&2&3\\4&5&6\\7&8&9}{R}		&	Composantes de la matrice
		\\\hline
			\verb!\IGSABCDEF!			&	\IGSABCDEF		&	Composantes du tenseur en $G$ dans le repere $R$.
		\\\hline
			\verb!\IGSABCDEF[G_1][R_1]!			&	\IGSABCDEF[G_1][R_1]		&	Composantes du tenseur en un autre point et une autre base.
		\\\hline
			\verb!\IGSABC!			&	\IGSABC		&	Composantes du tenseur diagonal (similaire � \verb!\IGSABCDEF!)
		\\\hline
			\verb!\IGSABC[G_3][R][A_3]! \verb![B_3][C_3]!			&	\IGSABC[G_3][R][A_3][B_3][C_3]		&	Composantes du tenseur diagonal en choisissant les valeurs
		\\\hline
			\verb!\IGSParallelepipede! \verb!{a}{b}{c}!			&	\IGSParallelepipede{a}{b}{c}		&	Matrice d'inertie d'un parall�l�pip�de
		\\\hline
			\verb!\IGSParallelepipede[A]! \verb![M_2]{a}{b}{c}[R_1]!			&	\IGSParallelepipede[A][M_2]{a}{b}{c}[R_1]		&	idem, en un autre point, et un autre rep�re.
		\\\hline
			\verb!\IGSCylindre! \verb!{R}{H}!			&	\IGSCylindre{R}{H}		&	Matrice d'inertie d'un cylindre de rayon $R$ et de hauteur $H$.
		\\\hline
	\end{tabular}

	\section{Cin�tique}
	%-------------------------------------------
	\noindent
	\begin{tabular}{|p{0.35\linewidth}|p{0.3\linewidth}|p{0.3\linewidth}|}
		\hline
			\textbf{Commandes}&\textbf{Rendus}&\textbf{Commentaires}
		\\\hline\hline
			\verb!\CCallig!			&	\CCallig			&	C calligraphi�
		\\\hline
			\verb!\tCinetique{S_1}{S_2}!			&	\tCinetique{S_1}{S_2}			&	Torseur cin�tique
		\\\hline
			\verb!\resCinetique{S_1}{S_2}!			&	\resCinetique{S_1}{S_2}			&	R�sultante cin�tique
		\\\hline
			\verb!\momCinetique{P}{S_1}! \verb!{S_2}!			&	\momCinetique{P}{S_1}{S_2}			&	Moment cin�tique au point P
		\\\hline
	\end{tabular}

	\section{Dynamique}
	%-------------------------------------------
    \noindent
	\begin{tabular}{|p{0.35\linewidth}|p{0.3\linewidth}|p{0.3\linewidth}|}
		\hline
			\textbf{Commandes}&\textbf{Rendus}&\textbf{Commentaires}
		\\\hline\hline
			\verb!\ACallig!			&	\ACallig			&	\ACallig calligraphi�
		\\\hline
			\verb!\dA!			&	\dA			&	Quantit� d'acc�l�ration (scalaire)
		\\\hline
			\verb!\vdA!			&	\vdA			&	Quantit� d'acc�l�ration (vecteur)
		\\\hline
			\verb!\resDynamique{S}{R}!			&	\resDynamique{S}{R}			&	R�sultante dynamique
		\\\hline
			\verb!\momDynamique{A}{S}{R}!			&	\momDynamique{A}{S}{R}			&	moment dynamique au point $A$
		\\\hline
	\end{tabular}
\end{document}
