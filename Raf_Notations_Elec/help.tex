\documentclass[a4paper,12pt]{article}

	\usepackage{xcolor}
	\usepackage{listings}	\lstset{language=[LaTeX]TeX,basicstyle=\ttfamily,texcsstyle=*\color{blue},identifierstyle=\color{brown},commentstyle=\color{gray}\itshape,escapechar=!,moretexcs={}}
	\usepackage[latin1]{inputenc} 
	\usepackage{hyperref}

	\usepackage{Raf_Notations_Elec}



	\newcommand{\rac}{({\color{red}Raccourci})}
	\newcommand{\ren}{({\color{blue}Renomm�})}

\begin{document}

	\begin{center}
		\hrule{\Large Notations Cotation Fonctionnelle}\\\hrule
	\end{center}


	\section{Packages requis}
	%-------------------------------------

		\begin{itemize}
			\item \href{http://www.ctan.org/pkg/ifthen}{\textbf{ifthen}} : Package pour faire des compilations conditionnelles (if...then...else....)
			%\item \href{http://www.ctan.org/pkg/amsmath}{\textbf{amsmath}} : Pour des notations math�matiques (notamment l'utilisation de \verb!\text! il me semble).
			%\item \href{http://www.ctan.org/pkg/pgf}{\textbf{tikz}} : Package pour faire des dessins (avec library \verb!calc!)
			%\item \href{http://www.ctan.org/pkg/calc}{\textbf{calc}} : Permet de faire des petits calculs au moment de la compilation
			\item \href{http://www.ctan.org/pkg/xargs}{\textbf{xargs}} : Pour cr�er des commandes avec plusieurs arguments optionnels
			%\item \href{http://www.ctan.org/pkg/array}{\textbf{array}} : Package qui rajoute des possibilit�s aux tableaux.
			\item \href{https://www.ctan.org/pkg/circuitikz}{\textbf{circuitikz}} : Pour tracer des circuits �lectrique avec tikz. Les options \emph{european} et \emph{cute inductors} sont activ�es.
		\end{itemize}
		
		
	\section{Appel du package}
	%-------------------------------------

		Le package est appel� en d�but de document par la commande :
		\begin{verbatim}
\usepackage{Raf_Notations_Elec}
		\end{verbatim}

		Par d�faut, ce package utilise un certain nombre de notations raccourcies, susceptibles de rentrer en conflit avec d'autres packages (mais tellement plus rapide � taper !).
		De plus, certaines commandes ont �t� rebaptis�es.
		Ces raccourcis et renommages seront cit�s (\rac\ ou \ren) dans les tableaux suivants.
		Si cela devait poser probl�me, pour ne pas cr�er ces raccourcis/renommage, il faut rentre l'option \verb!noRaccourci! � l'appel du package.

		\begin{verbatim}
usepackage[noRaccourci]{Raf_Notations_Elec}
		\end{verbatim}


	\section{Raccourcis notations}
	%-----------------------------
		\noindent
		\begin{tabular}{|p{0.3\linewidth}|p{0.3\linewidth}|p{0.35\linewidth}|}
			\hline
				\textbf{Commandes}&\textbf{Rendus}&\textbf{Commentaires}
			\\\hline\hline
				\verb!\thevenin{E}{Z}!			&	\thevenin{E}{Z}		&	Configuration de Th�venin.
			\\\hline
				\verb!\thevenin{E}{Z}[C]! \verb![D]!		&	\thevenin{E}{Z}[C][D]	&	Idem avec changement du nom des points.
			\\\hline
				\verb!\norton{I}{Z}!			&	\norton{I}{Z}		&	Configuration de Norton.
			\\\hline
				\verb!\norton{I}{Z}[C][D]!	&	\norton{I}{Z}[C][D]	&	Idem avec changement du nom des points.
			\\\hline
				\verb!\norton{I}{Z}[C][D]! \verb![-3]!	&	\norton{I}{Z}[C][D][-3]	&	Idem avec �cart plus grand entre la source de courant et l'imp�dance.
			\\\hline
		\end{tabular}

		

	\section{La forme des fl�ches de tension}
	%-----------------------------
		\textbf{Note : } Les fl�ches de tension des dip�les sont droites.
		
		
	\section{Moteurs � courant continu}
	%----------------------------------------
	
	Syntaxe : \verb!\mcc[a]{b}[c]! o� :
	\begin{itemize}
		\item \verb!a! est un angle de rotation (en degr�s) [Optionnel] ;
		\item \verb!b! sont les coordonn�es du point de d�part du dessin du moteur (souvent de la forme \verb!(x,y)!) ;
		\item \verb!c! est la tension aux bornes du moteur [Optionnel].
	\end{itemize}
	


		\noindent
		\begin{tabular}{|p{0.3\linewidth}|p{0.3\linewidth}|p{0.35\linewidth}|}
			\hline
				\textbf{Commandes}&\textbf{Rendus}&\textbf{Commentaires}
			\\\hline\hline
\begin{verbatim}
\begin{tikzpicture}
	\mcc{(0,0)}
\end{tikzpicture}
\end{verbatim}
&
\begin{tikzpicture}
	\mcc{(0,0)}
\end{tikzpicture}
&
Affichage minimum
\\\hline
\begin{verbatim}
\begin{tikzpicture}
	\mcc[20]{(0,0)}
\end{tikzpicture}
\end{verbatim}
&
\begin{tikzpicture}
	\mcc[20]{(0,0)}
\end{tikzpicture}
&
Avec une rotation de 20� (dans le sens horaire)
\\\hline
\begin{verbatim}
\begin{tikzpicture}
	\mcc{(0,0)}[2V]
\end{tikzpicture}
\end{verbatim}
&
\begin{tikzpicture}
	\mcc{(0,0)}[2V]
\end{tikzpicture}
&
Avec une tension
\\\hline
\begin{verbatim}
\begin{tikzpicture}
	\mcc[20]{(0,0)}[2V]
\end{tikzpicture}
\end{verbatim}
&
\begin{tikzpicture}
	\mcc[20]{(0,0)}[2V]
\end{tikzpicture}
&
Combo
\\\hline
		\end{tabular}
	
	

	\section{Interrupteur ouvert}
	%----------------------------------------
	
	Syntaxe : \verb!\switchOpen[a]{b}[c][d]! o� :
	\begin{itemize}
		\item \verb!a! est un angle de rotation (en degr�s) [Optionnel] ;
		\item \verb!b! sont les coordonn�es du point de d�part du dessin de l'interrupteur (souvent de la forme \verb!(x,y)!) ;
		\item \verb!c! est le nom de l'interrupteur [Optionnel].
		\item \verb!d! est la tension aux bornes de l'interrupteur [Optionnel].
	\end{itemize}
	
	Voici quelques exemples :

		\noindent
		\begin{tabular}{|p{0.5\linewidth}|p{0.2\linewidth}|p{0.25\linewidth}|}
			\hline
				\textbf{Commandes}&\textbf{Rendus}&\textbf{Commentaires}
			\\\hline\hline
\begin{verbatim}
\begin{tikzpicture}
	\switchOpen{(0,0)}
\end{tikzpicture}
\end{verbatim}
&
\begin{tikzpicture}
	\switchOpen{(0,0)}
\end{tikzpicture}
&
Affichage minimum
\\\hline
\begin{verbatim}
\begin{tikzpicture}
	\switchOpen[20]{(0,0)}
\end{tikzpicture}
\end{verbatim}
&
\begin{tikzpicture}
	\switchOpen[20]{(0,0)}
\end{tikzpicture}
&
Avec une rotation de 20� (dans le sens horaire)
\\\hline
\begin{verbatim}
\begin{tikzpicture}
	\switchOpen{(0,0)}[K]
\end{tikzpicture}
\end{verbatim}
&
\begin{tikzpicture}
	\switchOpen{(0,0)}[K]
\end{tikzpicture}
&
Avec une rotation de 20� (dans le sens horaire)
\\\hline
\begin{verbatim}
\begin{tikzpicture}
	\switchOpen{(0,0)}[][U]
\end{tikzpicture}
\end{verbatim}
&
\begin{tikzpicture}
	\switchOpen{(0,0)}[][U]
\end{tikzpicture}
&
Avec une tension
\\\hline
\begin{verbatim}
\begin{tikzpicture}
	\switchOpen[20]{(0,0)}[K][U]
\end{tikzpicture}
\end{verbatim}
&
\begin{tikzpicture}
	\switchOpen[20]{(0,0)}[K][U]
\end{tikzpicture}
&
Combo
\\\hline
		\end{tabular}
		


	\section{Interrupteur ferm�}
	%----------------------------------------
	
	Syntaxe : \verb!\switchClosed[a]{b}[c][d]! o� :
	\begin{itemize}
		\item \verb!a! est un angle de rotation (en degr�s) [Optionnel] ;
		\item \verb!b! sont les coordonn�es du point de d�part du dessin de l'interrupteur (souvent de la forme \verb!(x,y)!) ;
		\item \verb!c! est le nom de l'interrupteur [Optionnel].
		\item \verb!d! est la tension aux bornes de l'interrupteur [Optionnel].
	\end{itemize}
	
	Voici quelques exemples :

		\noindent
		\begin{tabular}{|p{0.5\linewidth}|p{0.2\linewidth}|p{0.25\linewidth}|}
			\hline
				\textbf{Commandes}&\textbf{Rendus}&\textbf{Commentaires}
			\\\hline\hline
\begin{verbatim}
\begin{tikzpicture}
	\switchClosed{(0,0)}
\end{tikzpicture}
\end{verbatim}
&
\begin{tikzpicture}
	\switchClosed{(0,0)}
\end{tikzpicture}
&
Affichage minimum
\\\hline
\begin{verbatim}
\begin{tikzpicture}
	\switchClosed[20]{(0,0)}
\end{tikzpicture}
\end{verbatim}
&
\begin{tikzpicture}
	\switchClosed[20]{(0,0)}
\end{tikzpicture}
&
Avec une rotation de 20� (dans le sens horaire)
\\\hline
\begin{verbatim}
\begin{tikzpicture}
	\switchClosed{(0,0)}[K]
\end{tikzpicture}
\end{verbatim}
&
\begin{tikzpicture}
	\switchClosed{(0,0)}[K]
\end{tikzpicture}
&
Avec une rotation de 20� (dans le sens horaire)
\\\hline
\begin{verbatim}
\begin{tikzpicture}
	\switchClosed{(0,0)}[][U]
\end{tikzpicture}
\end{verbatim}
&
\begin{tikzpicture}
	\switchClosed{(0,0)}[][U]
\end{tikzpicture}
&
Avec une tension
\\\hline
\begin{verbatim}
\begin{tikzpicture}
	\switchClosed[20]{(0,0)}[K][U]
\end{tikzpicture}
\end{verbatim}
&
\begin{tikzpicture}
	\switchClosed[20]{(0,0)}[K][U]
\end{tikzpicture}
&
Combo
\\\hline
		\end{tabular}

		
		
	\section{Hacheur}
	%----------------------------------------

		\subsection{Notation par d�faut}
		%...................................
	
\begin{verbatim}
\begin{tikzpicture}
	\hacheurQuatreQuadrants
\end{tikzpicture}
\end{verbatim}
	
\begin{tikzpicture}
	\hacheurQuatreQuadrants
\end{tikzpicture}


		\subsection{Choix des interrupteurs ouverts}
		%.................................................;
	
	Pour fermer un ou plusieurs interrupteurs,
	il faut passer en argument une s�rie de \verb![1]! (ferm�) ou de \verb![0]! (ouvert).
	L'ordre est : de haut en bas, puis de gauche � droite.
	
	
\begin{verbatim}
\begin{tikzpicture}
	\hacheurQuatreQuadrants[0][1][0][0]
\end{tikzpicture}
\end{verbatim}
	
\begin{tikzpicture}
	\hacheurQuatreQuadrants[0][1][0][0]
\end{tikzpicture}
\end{document}