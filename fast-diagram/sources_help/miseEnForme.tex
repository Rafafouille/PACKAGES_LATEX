\section{Mise en forme}\label{MIP}
%========================

	\subsection{Reset}\label{reset}
	%--------------------------

		{\color{blue}\verb'\fastReset'} permet de remettre les paramètres par défaut.



	\subsection{Les dimensions}\label{dimensions}
	%-------------------------------

		Les dimensions du diagramme sont réglées via plusieurs commandes.
		En voici la liste :
		\begin{itemize}
			\item {\color{blue}\verb'\fastInterligne'} : espace entre le bas de la boite la plus grande de la ligne en cours, et le haut des boites de la ligne suivante.
									Ce nombre doit être positif.
									(Par défaut : $0.5\uem$)
			\item {\color{blue}\verb'\fastLargeurBoite'} : largeur des boites (Par défaut : $7\uem$)
			\item {\color{blue}\verb'\fastHauteurBoite'} : hauteur \textbf{minimum} des boites (Par défaut : $0$)
			\item {\color{blue}\verb'\fastEspaceColonne'} :  distance entre le coin supérieur gauche d'une boite et le coin supérieur gauche de sa voisine.
									(Par défaut : $10\uem$)
			\item {\color{blue}\verb'\fastDecalageTrait'} : permet de décaler le connecteur par rapport au haut de la boite.
									(Par défaut : $-0.6\uem$)
			\item {\color{blue}\verb'\fastEpaisseurTraits'} : épaisseur des traits (bordures et connecteurs). (Par défaut : $0.05\uem$)
			\item {\color{blue}\verb'\fastDecalageOuVertical'} : Décalage vertical du connecteur ``OU''. (Par défaut : $0.4\uem$)
			\item {\color{blue}\verb'\fastDecalageOuHorizontal'} :	Décalage horizontal du connecteur ``OU''. (Par défaut : $-0.4\uem$)
		\end{itemize}

		Les deux dernières fonctions peuvent être utiles si plusieurs connecteur ``OU'' sont utilisés sur la même lignée.

		Toutes ces commandes peuvent être redéfinies via la fonction la fonction {\color{blue}\verb'\renewcommand'} (ou {\color{blue}\verb'\renewcommand*'}).
		Voici ci-dessous une série d'exemples illustrant chacune de ces fonctions.


		\subsubsection{Exemple : interlignes}\label{interligne}
		%-----------------------------------

%#####################################################
\begin{code}
\begin{fast}{Avant}	%Interligne par défaut
	\fastFT{FT1}{
		\fastFT{FT11 avec un peu de texte}{
			\fastFT{FT111}{}}}
	\fastFT{FT2}{
		\fastFT{FT21}{
			\fastFT{FT211}{}}}
\end{fast}

\renewcommand*{\fastInterligne}{1cm}	%Nouvel interligne
\begin{fast}{Après}
	\fastFT{FT1}{
		\fastFT{FT11 avec un peu de texte}{
			\fastFT{FT111}{}}}
	\fastFT{FT2}{
		\fastFT{FT21}{
			\fastFT{FT211}{}}}
\end{fast}
\fastReset	%Remise à zéro
\end{code}
%#####################################################
		\cqd
%%%%%%%%%%%%%%%%%%%%%%%%%%%%%%%%%%%%%%%%%%%%%%%%%%%%%%%%
\begin{exemple}
\begin{fast}{Avant}	%Interligne par défaut
	\fastFT{FT1}{
		\fastFT{FT11 avec un peu de texte}{
			\fastFT{FT111}{}}}
	\fastFT{FT2}{
		\fastFT{FT21}{
			\fastFT{FT211}{}}}
\end{fast}
\renewcommand*{\fastInterligne}{1cm}	%Nouvel interligne
\begin{fast}{Après}
	\fastFT{FT1}{
		\fastFT{FT11 avec un peu de texte}{
			\fastFT{FT111}{}}}
	\fastFT{FT2}{
		\fastFT{FT21}{
			\fastFT{FT211}{}}}
\end{fast}
\fastReset	%Remise à zéro
\end{exemple}
%%%%%%%%%%%%%%%%%%%%%%%%%%%%%%%%%%%%%%%%%%%%%%%%%%%%%%%%



		\subsubsection{Exemple : largeur des boîtes}\label{largeur}
		%-----------------------------------


%###############################################
\begin{code}
\begin{fast}{Avant}
	\fastFT{FT1}{}
	\fastFT{FT2}{}
\end{fast}

\renewcommand*{\fastLargeurBoite}{1.5cm}	%Nouvelle largeur de boite
\begin{fast}{Après}
	\fastFT{FT1}{}
	\fastFT{FT2}{}
\end{fast}
\fastReset
\end{code}
%###############################################
\cqd
%%%%%%%%%%%%%%%%%%%%%%%%%%%%%%%%%%%%%%%%%%%%%%%%%%%%%%%%
\begin{exemple}
\begin{fast}{Avant}
	\fastFT{FT1}{}
	\fastFT{FT2}{}
\end{fast}
\renewcommand*{\fastLargeurBoite}{1.5cm}
\begin{fast}{Après}
	\fastFT{FT1}{}
	\fastFT{FT2}{}
\end{fast}
\fastReset
\end{exemple}
%%%%%%%%%%%%%%%%%%%%%%%%%%%%%%%%%%%%%%%%%%%%%%%%%%%%%%%%



		\subsubsection{Exemple : hauteur des boîtes}\label{hauteur}
		%-----------------------------------


%###############################################
\begin{code}
\begin{fast}{Avant}
	\fastFT{FT1}{	\FT{FT11}{}
			\FT{FT12 FT12 FT12 FT12}{}}
	\fastFT{FT2}{	\FT{FT21}{}
			\FT{FT22}{}}
\end{fast}
\renewcommand*{\fastHauteurBoite}{3em}
\begin{fast}{Après}
	\fastFT{FT1}{	\FT{FT11}{}
			\FT{FT12 FT12 FT12 FT12}{}}
	\fastFT{FT2}{	\FT{FT21}{}
			\FT{FT22}{}}
\end{fast}
\fastReset
\end{code}
%###############################################
\cqd
%%%%%%%%%%%%%%%%%%%%%%%%%%%%%%%%%%%%%%%%%%%%%%%%%%%%%%%%
\begin{exemple}
\begin{fast}{Avant}
	\fastFT{FT1}{	\FT{FT11}{}
			\FT{FT12 FT12 FT12 FT12}{}}
	\fastFT{FT2}{	\FT{FT21}{}
			\FT{FT22}{}}
\end{fast}
\renewcommand*{\fastHauteurBoite}{3em}
\begin{fast}{Après}
	\fastFT{FT1}{	\FT{FT11}{}
			\FT{FT12 FT12 FT12 FT12}{}}
	\fastFT{FT2}{	\FT{FT21}{}
			\FT{FT22}{}}
\end{fast}
\fastReset
\end{exemple}
%%%%%%%%%%%%%%%%%%%%%%%%%%%%%%%%%%%%%%%%%%%%%%%%%%%%%%%%


		\subsubsection{Exemple : espace entre colonnes}\label{espace}
		%-----------------------------------


%###############################################
\begin{code}
\begin{fast}{Avant}
	\fastFT{FT1}{
		\fastFT{FT11}{}}
	\fastFT{FT2}{
		\fastFT{FT21}{}}
\end{fast}

\renewcommand*{\fastEspaceColonne}{6cm}	%Nouvel espace inter-colonnes
\begin{fast}{Après}
	\fastFT{FT1}{
		\fastFT{FT11}{}}
	\fastFT{FT2}{
		\fastFT{FT21}{}}
\end{fast}
\fastReset
\end{code}
%###############################################
\cqd
%%%%%%%%%%%%%%%%%%%%%%%%%%%%%%%%%%%%%%%%%%%%%%%%%%%%%%%%
\begin{exemple}
\begin{fast}{Avant}
	\fastFT{FT1}{
		\fastFT{FT11}{}}
	\fastFT{FT2}{
		\fastFT{FT21}{}}
\end{fast}
\renewcommand*{\fastEspaceColonne}{6cm}
\begin{fast}{Après}
	\fastFT{FT1}{
		\fastFT{FT11}{}}
	\fastFT{FT2}{
		\fastFT{FT21}{}}
\end{fast}
\fastReset
\end{exemple}
%%%%%%%%%%%%%%%%%%%%%%%%%%%%%%%%%%%%%%%%%%%%%%%%%%%%%%%%





	\subsubsection{Exemple : décalage des connecteurs}\label{decalage}
	%----------------------------------------

%###############################################
\begin{code}
\begin{fast}{Avant}
	\fastFT{FT1}{}
	\fastFT{FT2}{}
\end{fast}

\renewcommand*{\fastDecalageTrait}{-13pt}	%Nouveau décalage des connecteur
\begin{fast}{Après}
	\fastFT{FT1}{}
	\fastFT{FT2}{}
\end{fast}
\fastReset
\end{code}
%###############################################
\cqd
%%%%%%%%%%%%%%%%%%%%%%%%%%%%%%%%%%%%%%%%%%%%%%%%%%%%%%%%
\begin{exemple}
\begin{fast}{Avant}
	\fastFT{FT1}{}
	\fastFT{FT2}{}
\end{fast}
\renewcommand*{\fastDecalageTrait}{-13pt}
\begin{fast}{Après}
	\fastFT{FT1}{}
	\fastFT{FT2}{}
\end{fast}
\fastReset
\end{exemple}
%%%%%%%%%%%%%%%%%%%%%%%%%%%%%%%%%%%%%%%%%%%%%%%%%%%%%%%%







	\subsubsection{Exemple : épaisseur des traits}\label{epaisseur}
	%----------------------------------


%###############################################
\begin{code}
\begin{fast}{Avant}
	\fastFT{FT1}{}
	\fastFT{FT2}{}
\end{fast}

\renewcommand*{\fastEpaisseurTraits}{2pt}	%Nouvelle épaisseur de traits
\begin{fast}{Après}
	\fastFT{FT1}{}
	\fastFT{FT2}{}
\end{fast}
\fastReset
\end{code}
%###############################################
\cqd
%%%%%%%%%%%%%%%%%%%%%%%%%%%%%%%%%%%%%%%%%%%%%%%%%%%%%%%%
\begin{exemple}
\begin{fast}{Avant}
	\fastFT{FT1}{}
	\fastFT{FT2}{}
\end{fast}
\renewcommand*{\fastEpaisseurTraits}{2pt}
\begin{fast}{Après}
	\fastFT{FT1}{}
	\fastFT{FT2}{}
\end{fast}
\fastReset
\end{exemple}
%%%%%%%%%%%%%%%%%%%%%%%%%%%%%%%%%%%%%%%%%%%%%%%%%%%%%%%%



	\subsubsection{Exemple : Décalage des connecteur ``OU''}\label{connecteursOU}
	%----------------------------------


%###############################################
\begin{code}
\begin{fast}{Avant}
	\fastFT{FT1}{}
	\fastFT{FT2}{}
	\fastFT[ou]{FT3}{}
	\fastFT[ou]{FT4}{}
\end{fast}

\renewcommand*{\fastDecalageOuVertical}{3pt}	%Redécalage vertical...
\renewcommand*{\fastDecalageOuHorizontal}{-3pt}	%... et horizontal du 1er "OU"
\begin{fast}{Après}
	\fastFT{FT1}{}
	\fastFT{FT2}{}
	\fastFT[ou]{FT3}{}
	\renewcommand{\fastDecalageOuVertical}{6pt}	%Décalage vertical...
	\renewcommand{\fastDecalageOuHorizontal}{-6pt}	%...et horizontal...
	\fastFT[ou]{FT4}{}				% ...du 2eme "OU"
\end{fast}
\fastReset
\end{code}
%###############################################
\cqd
%%%%%%%%%%%%%%%%%%%%%%%%%%%%%%%%%%%%%%%%%%%%%%%%%%%%%%%%
\begin{exemple}
\begin{fast}{Avant}
	\fastFT{FT1}{}
	\fastFT{FT2}{}
	\fastFT[ou]{FT3}{}
	\fastFT[ou]{FT4}{}
\end{fast}
\renewcommand*{\fastDecalageOuVertical}{3pt}
\renewcommand*{\fastDecalageOuHorizontal}{-3pt}
\begin{fast}{Après}
	\fastFT{FT1}{}
	\fastFT{FT2}{}
	\fastFT[ou]{FT3}{}
	\renewcommand{\fastDecalageOuVertical}{6pt}
	\renewcommand{\fastDecalageOuHorizontal}{-6pt}
	\fastFT[ou]{FT4}{}
\end{fast}
\fastReset
\end{exemple}
%%%%%%%%%%%%%%%%%%%%%%%%%%%%%%%%%%%%%%%%%%%%%%%%%%%%%%%%







	\subsection{Couleurs}\label{couleurs}
	%--------------------------

		Il est possible de modifier les couleurs de plusieurs éléments tels que :
		\begin{itemize}
			\item \textbf{la fonction de service} (la première case),
			\item \textbf{les fonctions techniques},
			\item \textbf{les solutions techniques},
			\item \textbf{les boîtes vides},
			\item \textbf{les connecteurs}.
		\end{itemize}
		Pour chacun des quatre premiers points précédents, on peut définir :
		\begin{itemize}
			\item la couleur du \textbf{texte},
			\item la couleur du \textbf{fond} (sauf boite vide),
			\item la couleur du \textbf{cadre} (sauf boite vide).
		\end{itemize}
		Tout cela donne un total de $11$ couleurs, définies par les noms suivants :
		\begin{itemize}
			\item {\color{blue}\verb'fastCouleurTexteFS'} : Couleur du texte de la fonction de service (la $1^{ere}$ boite),
			\item {\color{blue}\verb'fastCouleurBorduresFS'} : Couleur de bordure de la fonction de service (la $1^{ere}$ boite),
			\item {\color{blue}\verb'fastCouleurFondFS'} : Couleur du fond de la fonction de service (la $1^{ere}$ boite),
			\item {\color{blue}\verb'fastCouleurTexteFT'} : Couleur du texte des fonctions techniques,
			\item {\color{blue}\verb'fastCouleurBorduresFT'} : Couleur de bordure des fonctions techniques,
			\item {\color{blue}\verb'fastCouleurFondFT'} : Couleur du fond des fonctions techniques,
			\item {\color{blue}\verb'fastCouleurTexteST'} : Couleur du texte des solutions techniques,
			\item {\color{blue}\verb'fastCouleurBorduresST'} : Couleur de bordure des solutions techniques,
			\item {\color{blue}\verb'fastCouleurFondST'} : Couleur du fond des solutions techniques,
			\item {\color{blue}\verb'fastCouleurTexteFV'} : Couleur du texte de la fonction de boite vide,
			\item {\color{blue}\verb'fastCouleurConnecteurs'} : Couleur des connecteurs.
		\end{itemize}

		Toutes ces couleurs peuvent être redéfinies par la fonction {\color{blue}\verb'\definecolor'}
		(voir le package \href{http://www.ctan.org/tex-archive/macros/latex/contrib/xcolor/}{xcolor}) :

%###################################################
\begin{code}
\definecolor{fastCouleurTexteFS}	{named}	{white}
\definecolor{fastCouleurBorduresFS}	{named}	{red}
\definecolor{fastCouleurFondFS}		{named}	{red}

\definecolor{fastCouleurTexteFT}	{rgb}	{1,0,1}
\definecolor{fastCouleurBorduresFT}	{rgb}	{0,1,0}
\definecolor{fastCouleurFondFT}		{rgb}	{1,1,0}

\definecolor{fastCouleurTexteST}	{named}	{brown}
\definecolor{fastCouleurBorduresST}	{named}	{blue}
\definecolor{fastCouleurFondST}		{rgb}	{0.5,1,1}

\definecolor{fastCouleurConnecteurs}	{rgb}	{1,0.5,1}
\begin{fast}{FS1}
	\fastFT{FT1}{
		\fastST{Sol 1}}
	\fastFT{}{
		\fastST{Sol2}}
\end{fast}
\fastReset
\end{code}
%###################################################
\cqd
%%%%%%%%%%%%%%%%%%%%%%%%%%%%%%%%%%%%%%%%%%%%%%%%%%%%%%%%
\begin{exemple}
\definecolor{fastCouleurTexteFS}	{named}	{white}
\definecolor{fastCouleurBorduresFS}	{named}	{red}
\definecolor{fastCouleurFondFS}		{named}	{red}

\definecolor{fastCouleurTexteFT}	{rgb}	{1,0,1}
\definecolor{fastCouleurBorduresFT}	{rgb}	{0,1,0}
\definecolor{fastCouleurFondFT}		{rgb}	{1,1,0}

\definecolor{fastCouleurTexteST}	{named}	{brown}
\definecolor{fastCouleurBorduresST}	{named}	{blue}
\definecolor{fastCouleurFondST}		{rgb}	{0.5,1,1}

\definecolor{fastCouleurConnecteurs}	{rgb}	{1,0.5,1}
\begin{fast}{FS1}
	\fastFT{FT1}{
		\fastST{Sol 1}}
	\fastFT{}{
		\fastST{Sol2}}
\end{fast}
\fastReset
\end{exemple}
%%%%%%%%%%%%%%%%%%%%%%%%%%%%%%%%%%%%%%%%%%%%%%%%%%%%%%%%

	Pour aller plus vite, trois commandes servent de raccourci :
	\begin{itemize}
		\item {\color{blue}\verb'\fastSetCouleurBordures[type]{couleur}'} : permet de changer la couleur de toutes les bordures,
		\item {\color{blue}\verb'\fastSetCouleurTexte[type]{couleur}'} : permet de changer la couleur de tout le texte,
		\item {\color{blue}\verb'\fastSetCouleurTraits[type]{couleur}'} : permet de changer la couleur de toutes les lignes (bordures + connecteurs),
		\item {\color{blue}\verb'\fastSetCouleurFond[type]{couleur}'} : permet de changer la couleur du fond de toutes les boites,
	\end{itemize}
	où {\color{blue}\verb'[type]'} est le type d'affectation (\emph{rgb},\emph{cmyk},\emph{named}(par défaut),...)
	et {\color{blue}\verb'[couleur]'} est la couleur, relativement à {\color{blue}\verb'[type]'} (voir {\color{blue}\verb'\definecolor'} du package \href{http://www.ctan.org/tex-archive/macros/latex/contrib/xcolor/}{xcolor}).


%###################################################
\begin{code}
\fastSetCouleurBordures{red}
\fastSetCouleurTexte[rgb]{1,1,1}
\fastSetCouleurFond{black}
\begin{fast}{FS1}
	\fastFT{FT1}{
		\fastST{Sol 1}}
	\fastFT{}{
		\fastST{Sol2}}
\end{fast}
\fastReset
\end{code}
%###################################################
\cqd
%%%%%%%%%%%%%%%%%%%%%%%%%%%%%%%%%%%%%%%%%%%%%%%%%%%%%%%%
\begin{exemple}
\fastSetCouleurBordures{red}
\fastSetCouleurTexte[rgb]{1,1,1}
\fastSetCouleurFond{black}
\begin{fast}{FS1}
	\fastFT{FT1}{
		\fastST{Sol 1}}
	\fastFT{}{
		\fastST{Sol2}}
\end{fast}
\fastReset
\end{exemple}
%%%%%%%%%%%%%%%%%%%%%%%%%%%%%%%%%%%%%%%%%%%%%%%%%%%%%%%%


