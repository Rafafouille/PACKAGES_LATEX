\section{Introduction}\label{intro}
%=====================================

	\subsection{Le pourquoi du comment}\label{auteur}
	%--------------------------------

		En tant qu'enseignant en sciences industrielles pour l'ingénieur, j'ai réalisé ce package en vue de m'aider à rédiger mes cours.
		J'ai toutefois essayé de le rendre le plus paramétrable possible afin qu'il puisse être utilisé dans de nombreux cas.
		(d'autres options/paramètres peuvent éventuellement être rajoutés selon la demande...).

		Il s'agit de mon premier package \LaTeX.
		De plus, ce package fonctionne sur la bibliothèque \emph{TikZ}, que je connaissais jusqu'alors assez mal.
		Il n'est donc pas exclu qu'il y ait des bugs dans sa conception.
		Si vous voyez quelque chose d'anormal ou d'incohérent, ou si vous avez des remarques, n'hésitez pas à m'en faire part à l'adresse suivante :
		\href{mailto:allais.raphael@free.fr}{allais.raphael@free.fr}
		
		Pour le petite histoire, la difficulté pour réaliser ce package a été le caractère récursif du diagramme FAST.
		En effet, il semblerait que \emph{TikZ} gère très mal la portée locale des variables :
		Les variables d'une fonction \emph{enfant} écrasaient les variables de sa fonction \emph{parent}.
		Cela posait des problèmes sur l'alignement des boîtes.
		D'autre part, \emph{TikZ} propose déjà des diagrammes en arborescence, mais je n'ai pas su créer mes propres fonctions par dessus.

		Merci à Yannick Le Bras, Robert Papanicola et Xavier Pessoles pour leur aide et leurs conseils.


	\subsection{Petit rappel}\label{rappel}
	%-----------------------------
		Le diagramme ``\emph{\href{http://fr.wikipedia.org/wiki/Function_Analysis_System_Technique}{Function Analysis System Technique}}'', plus couramment appelé ``\emph{diagramme FAST}''
		est un outil de \textbf{\href{http://fr.wikipedia.org/wiki/Analyse_fonctionnelle_\%28conception\%29}{l'analyse fonctionnelle}},
		permettant de décrire et de décomposer hiérarchique une \emph{fonction de service} en sous-fonctions, appelées \emph{fonctions techniques}.
		L'aboutissement d'un tel schéma doit être un ensemble de choix concrets appelés ``\emph{solutions techniques}''.
		Historiquement, ce type de diagramme a été un passage indispensable dans le domaine de la conception et la rédaction des cahiers des charges.
		Aujourd'hui, une approche plus globale (mais partiellement basée sur des concepts similaires) est proposée au travers des diagrammes \href{http://fr.wikipedia.org/wiki/Systems_Modeling_Language}{SysML}.

		Pour plus de détail, n'hésitez pas à consulter les nombreux cours qui existent sur Internet.
		

	

	\subsection{Limitations - Perspectives}\label{limitations}
	%----------------------------------------

		Le package a été écrit pour répondre \textbf{aux principales attentes} du diagramme FAST.
		Il n'est cependant pas complet.
		Il n'est, par exemple, pas possible de relier \textbf{automatiquement} une solution technique commune à plusieurs fonctions techniques.
		Cette possibilité n'est toutefois pas exclue puisque les commandes de \emph{TikZ} sont autorisées à l'intérieur de l'environnement (voir \ref{tikzz}) et rien n'empêche de le faire ``\emph{à la main}''.
		N'hésitez donc pas à me faire part d'éventuelles autres fonctions à mettre en place.