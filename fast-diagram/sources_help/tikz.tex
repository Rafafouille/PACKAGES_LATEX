\section{Jouons avec TikZ\label{tikzz}}
%=======================================



	\subsection{TikZ dans le diagramme FAST}\label{tikzpartout}
	%------------------------------------------------

		L'environnement FAST est un environnement \emph{TikZ}.
		Il est donc possible d'y ajouter n'importe quelle fonction de dessin de \emph{TikZ}.
		Il en est de m�me pour les descendances des fonctions.
%##########################################
\begin{code}
\begin{fast}{Fonction de Service}
	\FT{FT1}{\draw [shift={(4,-1)},rotate=45,scale=0.5,ball color=blue]
		(0,0) .. controls +(0,2) and +(0,3) .. (3,0)
		.. controls +(0,-2) and +(0,2) .. (0,-4)
		.. controls +(0,2) and +(0,-2) .. (-3,0)
		.. controls +(0,2) and +(0,2) .. (0,0);
		}		%Exemple pris dans ``TikZ pour l'impatient''
	\FT{FT2}{}
\end{fast}
\end{code}
%##########################################
		\cqd
%%%%%%%%%%%%%%%%%%%%%%%%%%%%%%%%%%%%%%%%%%
\begin{exemple}
\begin{fast}{Fonction de Service}
	\FT{FT1}{\draw [shift={(4,-1)},rotate=45,scale=0.5,ball color=blue]
			(0,0) .. controls +(0,2) and +(0,3) .. (3,0)
			.. controls +(0,-2) and +(0,2) .. (0,-4)
			.. controls +(0,2) and +(0,-2) .. (-3,0)
			.. controls +(0,2) and +(0,2) .. (0,0);}
			%Exemple pris dans ``TikZ pour l'impatient''
	\FT{FT2}{}
\end{fast}
\end{exemple}
%%%%%%%%%%%%%%%%%%%%%%%%%%%%%%%%%%%%%%%%%%

		Il est � noter que dans l'exemple pr�c�dent, la seconde ligne du diagramme ne tient pas compte de la ``place'' que prend notre dessin.
		Pour que ce soit le cas, il faut que la descendance (c'est � dire le dessin) ``marque'' sa place en cr�ant une coordonn�e correspondant au point le plus bas du dessin.
		C'est sur ce point que la seconde ligne va se baser.

		Ce point doit �tre enregistr� dans la variable {\color{blue}\verb'BoiteMinimums'} de la mani�re suivante :
%##########################################
\begin{code}
\coordinate (BoiteMinimums) at (X,Y);
\end{code}
%##########################################
		o� le couple $(X, Y)$ est la coordonn�es du minimum.

		Par exemple :
%##########################################
\begin{code}
\begin{fast}{Fonction de Service}
	\FT{FT1}{\draw [shift={(4,-1)},rotate=45,scale=0.5,ball color=blue]
		(0,0) .. controls +(0,2) and +(0,3) .. (3,0)
		.. controls +(0,-2) and +(0,2) .. (0,-4)
		.. controls +(0,2) and +(0,-2) .. (-3,0)
		.. controls +(0,2) and +(0,2) .. (0,0);
		\coordinate (BoiteMinimums) at (0,-2.5);
		}	%Exemple pris dans ``TikZ pour l'impatient''
	\FT{FT2}{}
\end{fast}
\end{code}
%##########################################
		\cqd
%%%%%%%%%%%%%%%%%%%%%%%%%%%%%%%%%%%%%%%%%%
\begin{exemple}
\begin{fast}{Fonction de Service}
	\FT{FT1}{\draw [shift={(4,-1)},rotate=45,scale=0.5,ball color=blue]
		(0,0) .. controls +(0,2) and +(0,3) .. (3,0)
		.. controls +(0,-2) and +(0,2) .. (0,-4)
		.. controls +(0,2) and +(0,-2) .. (-3,0)
		.. controls +(0,2) and +(0,2) .. (0,0);
		\coordinate (BoiteMinimums) at (0,-2.5);}
		%Exemple pris dans ``TikZ pour l'impatient''
	\FT{FT2}{}
\end{fast}
\end{exemple}
%%%%%%%%%%%%%%%%%%%%%%%%%%%%%%%%%%%%%%%%%%

	\subsection{Gestion des bo�tes}\label{boites}
	%-----------------------------------------

	Les boites cr��es dans le diagramme FAST sont r�alis�es par la fonction {\color{blue}\verb'\node'} de \emph{TikZ}.
	Ces bo�tes sont nomm�es sous la forme suivante : {\color{blue}\verb'\fastBoiteX'} o� {\color{blue}\verb'X'} est remplac� par le num�ro de la boite.
	Ce num�ro est d�fini par ordre de cr�ation des boites : de gauche � droite, de haut en bas.
	Voici un exemple faisant appara�tre le nom des boites :
	\begin{center}
		\begin{fast}{fastBoite0}
			\FT{fastBoite1}{\FT{fastBoite2}{}
					\FT{fastBoite3}{\FT{fastBoite4}{}}}
			\FT{fastBoite5}{\FT{fastBoite6}{}
					\FT{fastBoite7}{}}
		\end{fast}
	\end{center}

	Partant de l�, il est alors possible de r�aliser des modifications manuelles sur le diagramme.
	Par exemple, pour avoir une solution technique commune � deux fonctions techniques :
%##########################################
\begin{code}
\begin{fast}{Fonction de service}
	\fastFT{FT1}{\fastST{ST}}
	\fastFT{FT2}{}
	\draw[line width=\fastEpaisseurTraits]
		(fastBoite3.east) -| ($0.5*(fastBoite2.north west)
		+0.5*(fastBoite1.north east)+(0,\fastDecalageTrait)$);
\end{fast}
\end{code}
%##########################################
	\cqd
%%%%%%%%%%%%%%%%%%%%%%%%%%%%%%%%%%%%%%%%%%
\begin{exemple}
\begin{fast}{Fonction de service}
	\fastFT{FT1}{\fastST{ST}}
	\fastFT{FT2}{}
	\draw[line width=\fastEpaisseurTraits](fastBoite3.east)	-| ($0.5*(fastBoite2.north west)+0.5*(fastBoite1.north east)+(0,\fastDecalageTrait)$);
\end{fast}
\end{exemple}
%%%%%%%%%%%%%%%%%%%%%%%%%%%%%%%%%%%%%%%%%%

	\subsection{Cr�er sa propre boite}\label{perso}
	%--------------------------------------

	Les boites sont � peu pr�s toutes cr��es sur le m�me mod�le et il est possible d'en cr�er d'autres :
%##########################################
\begin{code}
\newcommand*{\maBoite}[2]{
	\fastAvanceColonne		%On avance d'une colonne
	\addtocounter{cptBoite}{1}	%On incremente le numero de la boite
	%%%%%%%%%%%%%%%%%%%%%%%
	%Cr�er votre boite ici :
	\node [anchor=north west] (noeud \thecptAbscisse) at
		($(\posX,0)+(BoiteMinimums)$) {#1};
	%%%%%%%%%%%%%%%%%%%%%%
	\node[inner sep=0,fit=(noeud \thecptAbscisse.north west)
		(noeud \thecptAbscisse.south east)]
		(fastBoite\thecptBoite) {};%Boite de nommage
	\fastTraceConnecteurs
	%%%%%%%%%%%%%%%%%%%%%%%%%
	%Votre descendance :
	#2
	%%%%%%%%%%%%%%%%%%%%%%%%%
	\fastEnregistreMinimum		%Enregistre le minimum de la boite
	\fastReculeColonne		%Recule d'une colonne
}
\end{code}
%##########################################


	Le n\oe ud cr�� sous la ligne ``{\color{blue}\verb'Cr�er votre boite ici'}'' est la boite que vous allez afficher.
	C'est elle que vous allez pouvoir modifier pour l'adapter � vos besoins.
	Ce n\oe ud doit obligatoirement porter le nom {\color{blue}\verb'(noeud \thecptAbscisse)'}.
	Les autres commandes ne doivent pas �tre chang�es.

	Voici un exemple :
%##########################################
\begin{code}
  \newcommand*{\maBoite}[2]{
	\fastAvanceColonne		%On avance d'une colonne
	\addtocounter{cptBoite}{1}	%On incremente le numero de la boite
	%%%%%%%%%%%%%%%%%%%%%%%
	%Cr�er votre boite ici
	\node [anchor=north west,draw,rounded corners=3pt,
		aspect=2.5,text=red](noeud \thecptAbscisse)
		at ($(\posX,0)+(BoiteMinimums)$) {#1};
	%%%%%%%%%%%%%%%%%%%%%%
	\node[inner sep=0,fit=(noeud \thecptAbscisse.north west)
		(noeud \thecptAbscisse.south east)]
		(fastBoite\thecptBoite) {};
	\fastTraceConnecteurs
	%%%%%%%%%%%%%%%%%%%%%%%%%
	%Votre descendance
	#2
	%%%%%%%%%%%%%%%%%%%%%%%%%
	\fastEnregistreMinimum		%Enregistre le minimum de la boite
	\fastReculeColonne		%Recule d'une colonne
}

\begin{fast}{Fonction de Service}
	\maBoite{Ma boite}
		{\fastST{Solution}}
	\FT{Fonction}{\maBoite{Ma boite bis}{}
			\fastFT{Fonction}{}}
\end{fast}
\end{code}
%##########################################
\cqd
%%%%%%%%%%%%%%%%%%%%%%%%%%%%%%%%%%%%%%%%%%
\begin{exemple}
  \newcommand*{\maBoite}[2]{
	\fastAvanceColonne		%On avance d'une colonne
	\addtocounter{cptBoite}{1}	%On incremente le numero de la boite
	%%%%%%%%%%%%%%%%%%%%%%%
	%Cr�er votre boite ici
	\node [anchor=north west,draw,rounded corners=3pt,aspect=2.5,text=red](noeud \thecptAbscisse) at ($(\posX,0)+(BoiteMinimums)$) {#1};
	%%%%%%%%%%%%%%%%%%%%%%
	\node[inner sep=0,fit=(noeud \thecptAbscisse.north west)
		(noeud \thecptAbscisse.south east)]
		(fastBoite\thecptBoite) {};%Boite vide par dessus, aux bonne dimension, afin de lui donner un nom
	\fastTraceConnecteurs
	%%%%%%%%%%%%%%%%%%%%%%%%%
	%Votre descendance
	#2
	%%%%%%%%%%%%%%%%%%%%%%%%%
	\fastEnregistreMinimum		%Enregistre le minimum de la boite
	\fastReculeColonne		%Recule d'une colonne
}

\begin{fast}{Fonction de Service}
	\maBoite{Ma boite}
		{\fastST{Solution}}
	\FT{Fonction}{\maBoite{Ma boite bis}{}
			\fastFT{Fonction}{}}
\end{fast}
\end{exemple}
%%%%%%%%%%%%%%%%%%%%%%%%%%%%%%%%%%%%%%%%%%