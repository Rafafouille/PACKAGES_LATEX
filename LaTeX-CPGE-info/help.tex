\documentclass[a4paper,10pt]{article}

\usepackage{CPGE-info}


\newcommand{\rac}{{\color{red}(\emph{Raccourci})}}
\newcommand{\ren}{{\color{blue}(\emph{Renommé})}}

\begin{document}


	\begin{center}
		\Huge Aide pour le package CPGE-info
	\end{center}
	
	\hrule


	\section{appel du package}
	%----------------------------------
	
	Le fichier doit être copié :
	\begin{itemize}
		\item soit dans le même dossier que le fichier \verb!.tex! à compiler ;
		\item soit dans le dossier commun de tous les packages (pour Windows, je crois que c'est :\\ \verb! c:/Program Files/MikTeX x.x/tex/latex!) puis,
		dans une console, lancer la commande \verb!texhash! pour mettre la liste des paquets à jour.
	\end{itemize}

	Ensuite, il faut simplement appeler le package en entête du fichier \verb!.tex! de la manière suivante :
	\begin{verbatim}
\usepackage{CPGE-info}
	\end{verbatim}
	
	Un certain nombre de commandes sont des \gras{raccourcis}, qui peuvent rentrer en conflit avec d'autres packages.
	Ces commandes sont notées avec \rac.
	Pour les annuler, on pourra appeler le package avec l'option << \verb![noRaccourci]! >> :
	\begin{verbatim}
\usepackage[noRaccourci]{CPGE-info}
	\end{verbatim}
	
	De plus, certaines commandes existent déjà et ont été renommées. Dans ce cas, elle seront indiquées par \ren.

	\section{Configuration du document}
	%----------------------------------------

		\begin{tabular}{|m{0.3\linewidth}|m{0.2\linewidth}|m{0.4\linewidth}|}
			\hline
				CODE LaTex			&	RENDU				&	COMMENTAIRES
			\\\hline\hline
				\verb!\version{prof}!		&					&	Passe en << version prof >> (corrigés, commentaires...)
			\\\hline
				\verb!\version{eleve}!		&					&	(élève ou autre...) repasse en version << eleve >>
			\\\hline
				\verb!\numero{123}!		&					&	Définit le numéro du cours (vide par défaut -- a mettre en entête)
			\\\hline
				\verb!\getNumeroCours!		&	\numero{123}\getNumeroCours	&	Renvoie le numéro du cours (précédemment affecté)
			\\\hline
				\verb!\getNumero!		&	\numero{123}\getNumero		&	Identique à \verb!\getNumeroCours!
			\\\hline
				\verb!\classe{PT*}!		&					&	Définit la classe (<< \getClasse\ >> par défaut -- a mettre en entête)
			\\\hline
				\verb!\getClasse!		&	\classe{PT*}\getClasse		&	Renvoie la classe (précédemment affecté)
			\\\hline
				\verb!\partie{Simulation}!	&	\partie{Simulation}		&	Définit la partie de cours dans l'année (vide par défaut -- a mettre en entête)
			\\\hline
				\verb!\getPartie!		&	\partie{Simulation}\getPartie	&	Renvoie la partie de cours dans l'année (précédemment affecté)
			\\\hline
				\verb!\annee{2011}!		&					&	Définit l'année (<< \getAnnee\ >> par défaut -- a mettre en entête)
			\\\hline
				\verb!\getAnnee!		&	\annee{2011}\getAnnee		&	Renvoie l'année (précédemment affectée)
			\\\hline
				\verb!\titre{Dérivation numérique}!		&			&	Définit le titre du cours (vide par défaut -- a mettre en entête)
			\\\hline
				\verb!\competences{\item compet.1! \verb!\item compet.2}!&		&	Définit les compétences sous forme de liste (vide par défaut)
			\\\hline
				\verb!\begin{itemize}! \verb!\getListeCompetences! \verb!\end{itemize}!	&	\competences{\item compet.1\item compet.2}\begin{itemize}\getListeCompetences\end{itemize}	&	Renvoie la liste des compétences (attention : sans les begin et end)
			\\\hline
		\end{tabular}

\numero{123}
\classe{PT*}
\partie{Simulation}
\annee{2011}
\titre{Dérivation numérique}
\competences{\item compet.1\item compet.2}

\newpage
	\section{Entêtes de documents}
	%----------------------------------------
	
		\verb!\makeEnteteCours!
		
		\makeEnteteCours
		
		\verb!\makeEnteteTD!
		
		\makeEnteteTD
		
		\verb!\makeEnteteDM!
		
		\makeEnteteDM
		
		\verb!\makeEnteteDS!
		
		\makeEnteteDS
		
		\verb!\makeEnteteCorrige!
		
		\makeEnteteCorrige
		
		\verb!\makeEnteteControle!
		
		\makeEnteteControle
		
	\section{Raccourcis commandes}
	%--------------------------------------
	
		\begin{tabular}{|m{0.3\linewidth}|m{0.2\linewidth}|m{0.4\linewidth}|}
			\hline
				CODE LaTex			&	RENDU				&	COMMENTAIRES
			\\\hline\hline
				\verb!\gras{texte en gras}!	&	\gras{texte en gras}		&	Raccourci de \verb!\textbf!
			\\\hline
				\verb!\italic{texte en italic}!	&	\italic{texte en italic}		&	Raccourci de \verb!\emph!
			\\\hline
				\verb!\souligne{texte souligné}!&	\souligne{texte souligné}		&	Raccourci de \verb!\underline!
			\\\hline
				\verb!\fig{Label}!&	\fig{Label-Figure}		&	Référence vers une figure. Raccourci de \verb!(fig.\ref{Label})!
			\\\hline
				\verb!$1\er$!	&	$1\er$		&	Symbole de premier
			\\\hline
				\verb!$1\ere$!	&	$1\ere$		&	Symbole de première
			\\\hline
				\verb!$2\eme$!	&	$2\eme$		&	Symbole de n-ième
			\\\hline
				\verb!a\tabulation b!	&	a\tabulation b		&	Tabulation de $1cm$ par défaut
			\\\hline
				\verb!a\tabulation[2cm] b!	&	a\tabulation[2cm] b		&	Tabulation de $2cm$
			\\\hline
				\verb!a\tab b!	&	a\tab b		&	Raccourci de \verb!\tabulation! \rac
			\\\hline
				\verb!a\tab[2cm] b!	&	a\tab[2cm] b		&	Raccourci de \verb!\tabulation!  \rac
			\\\hline
				\verb!\bigO{n}!	&	\bigO{n}		&	Grand ``O'' (Notation de Landeau)
			\\\hline
				\verb!\O{n}!	&	\O{n}		&	Raccourci \rac\ren
			\\\hline
				\verb!\smallO{n}!	&	\smallo{n}		&	Petit ``o'' (Notation de Landeau)
			\\\hline
				\verb!\o{n}!	&	\o{n}		&	Raccourci \rac\ren
			\\\hline
				\verb!\variable{n}!	&	\variable{n}		&	Notation d'une variable
			\\\hline
				\verb!\var{n}!	&	\var{n}		&	Raccourci de Variable \rac
			\\\hline
				\verb!\quoteVar{id}!	&	\quoteVar{id}		&	Mise en évidence d'une variable.
			\\\hline
				\verb!\quotevar{id}!	&	\quoteVar{id}		&	Identique à \verb!\quoteVar! (sans majuscule)
			\\\hline
				\verb!\qvar{id}!		&	\qvar{id}		&	Identique à \verb!\quoteVar! \rac
			\\\hline
				\verb!\qVar{id}!		&	\qVar{id}		&	Identique à \verb!\quoteVar! \rac
			\\\hline
				\verb!\qvr{id}!		&	\qvr{id}		&	Identique à \verb!\quoteVar! \rac
			\\\hline
		\end{tabular}

		
	\section{Numération}
	%--------------------------------------
	
	
		\begin{tabular}{|m{0.3\linewidth}|m{0.2\linewidth}|m{0.4\linewidth}|}
			\hline
				CODE LaTex			&	RENDU				&	COMMENTAIRES
			\\\hline\hline
				\verb!\lBase{101010}[2]!	&	\lBase{101010}[2]		&	Format d'un nombre dans une base donnée (décimal par défaut)
			\\\hline
				\verb!\binaire{101010}!		&	\binaire{101010}		&	Format binaire
			\\\hline
				\verb!\Bin{101010}!		&	\Bin{101010}		&	Raccourci de \verb!\binaire!
			\\\hline
				\verb!\octal{1234}!		&	\octal{1234}		&	Format octal
			\\\hline
				\verb!\Oct{1234}!		&	\Oct{1234}		&	Raccourci de \verb!\octal!
			\\\hline
				\verb!\decimal{1234}!		&	\decimal{1234}		&	Format décimal
			\\\hline
				\verb!\Dec{1234}!		&	\Dec{1234}		&	Raccourci de \verb!\decimal!
			\\\hline
				\verb!\hexadecimal{1A2B}!	&	\hexadecimal{1A2B}		&	Format hexadécimal
			\\\hline
				\verb!\Hex{1A2B}!		&	\Hex{1A2B}		&	Raccourci de \verb!\hexadecimal!
			\\\hline
				\verb!\poseDivision{num}{denom}! \verb!{reste}{quotient}!	&	\poseDivision{num}{denom}{reste}{quotient}		&	Division posée
			\\\hline
				\verb!\poseDivision{7}{2}{1}{3}!	&	\poseDivision{7}{2}{1}{3}		&	Exemple de division
			\\\hline
				\verb!\poseDivision{6}{2}{0}{! \verb!\poseDivision{3}{2}{1}{! \verb!\poseDivision{1}{2}{1}{0}!  \verb!}! \verb!}!	&	\poseDivision{6}{2}{0}{\poseDivision{3}{2}{1}{\poseDivision{1}{2}{1}{0}}}		&	Divisions en cascades
			\\\hline
		\end{tabular}
		
		
	\section{Les questions / réponses}
	%-----------------------------------------
	
		\verb!\question{Quelle est la question ?}!
		
		\question{Quelle est la question ?}
		
		\verb!\question{Une deuxième question ?}!
		
		\question{Une deuxième question ?}
		
		\verb!\resetQuestions!
		
		\verb!\question{Remise à zéro des questions ?}!
		
		\resetQuestions
		\question{Remise à zéro des questions ?}
		
	\section{Les boites}
	%----------------------------
	
		\subsection{Les définitions}
		%.............................
		
		\begin{verbatim}
\begin{definition}
	Ceci est une définition
\end{definition}
		\end{verbatim}

\begin{definition}
	Ceci est une définition
\end{definition}

		\begin{verbatim}
\begin{definition}[Titre]
	Ceci est une définition avec un titre
\end{definition}
		\end{verbatim}
		
\begin{definition}[Titre]
	Ceci est une définition avec un titre
\end{definition}


		\begin{verbatim}
\begin{definition}[Titre][nobreak=true]
	Ceci est une définition avec option passée à BClogo
	(ici : pour ne passer casser la boite en 2 dans la version 3 de bclogo...)
\end{definition}
		\end{verbatim}
		
\begin{definition}[Titre][nobreak=true]
	Ceci est une définition avec option passée à BClogo
	(ici : pour ne passer casser la boite en 2 dans la version 3 de bclogo...)
\end{definition}

		\begin{verbatim}
\begin{definition*}[Titre]
	Ceci est une définition avec un titre et sans numéro
\end{definition*}
		\end{verbatim}
		
\begin{definition*}[Titre]
	Ceci est une définition avec un titre et sans numéro
\end{definition*}


		\begin{verbatim}
\begin{definitions}[Titre]
	\item Ceci est une liste de définitions
	\item attention au << s >> de << definitions >>
\end{definitions}
		\end{verbatim}
		
\begin{definitions}[Titre]
	\item Ceci est une liste de définitions
	\item attention au << s >> de << definitions >>
\end{definitions}


		\begin{verbatim}
\begin{definitions*}[Titre]
	\item Ceci est une liste de définitions
	\item attention au << s >> de << definitions >>
\end{definitions*}
		\end{verbatim}
		
\begin{definitions*}[Titre]
	\item Combo !
	\item Liste de définitions, sans numéro !
\end{definitions*}







		\subsection{Les propriétés}
		%.............................

		Le principe est le même que pour les définitions !
		
		\begin{verbatim}
\begin{propriete}[Titre]
	Ceci est une propriété avec un titre
\end{propriete}
		\end{verbatim}
		
\begin{propriete}[Titre]
	Ceci est une propriété avec un titre
\end{propriete}





		\subsection{Les exemples}
		%.............................

		Le principe est le même que pour les définitions !
		
		\begin{verbatim}
\begin{exemple}[Titre]
	Ceci est un exemple avec un titre
\end{exemple}
		\end{verbatim}
		
\begin{exemple}[Titre]
	Ceci est un exemple avec un titre
\end{exemple}

		\subsection{Les remarques}
		%.............................

		Le principe est le même que pour les définitions !
		
		\begin{verbatim}
\begin{remarque}[Titre]
	Ceci est une remarque avec un titre
\end{remarque}
		\end{verbatim}
		
\begin{remarque}[Titre]
	Ceci est une remarque avec un titre
\end{remarque}


		\subsection{Important}
		%.............................

		La, il n'y a qu'une seule boite (pas comme ``définition'')
		
		\begin{verbatim}
\begin{important}[Titre]
	Ceci est une chose importante avec un titre
\end{important}
		\end{verbatim}
		
\begin{important}[Titre]
	Ceci est une chose importante avec un titre
\end{important}



		\subsection{Les compétences}
		%.............................

			\begin{verbatim}
\competences{\item compet.1 \item compet.2}

\begin{boiteCompetences}
	\getListeCompetences
\end{boiteCompetences}
			\end{verbatim}

		
		
			\bgroup
			\competences{\item compet.1 \item compet.2}
			\begin{boiteCompetences}
				\getListeCompetences
			\end{boiteCompetences}
			\egroup
			
			
		\subsection{Les exercices}
		%.............................

		Les exercices peuvent être définis sous forme de boite ou sous forme de sous-section (au choix...)
		
			\subsubsection{Sous forme de boite :}
		
		\begin{verbatim}
\begin{exercice}[Exo1]
	Ceci est un exercice.
	\question{Quelle est la question ?}
	\question{Quelle est la réponse ?}
\end{exercice}
\begin{exercice}[Exo2]
	\question{Est-ce que le compteur de question se remet à zéro tout seul ?}
\end{exercice}
		\end{verbatim}
		
\begin{exercice}[Exo1]
	Ceci est un exercice.
	\question{Quelle est la question ?}
	\question{Quelle est la réponse ?}
\end{exercice}
\begin{exercice}[Exo2]
	\question{Est-ce que le compteur de question se remet à zéro tout seul ?}
\end{exercice}

			\subsubsection{Sous forme de sous-section :}

\verb!\sectionExercice{Nouvel exercice sous forme de sous-section (au sens de LaTex)}!

\verb!\question{$1\ere$ question}!

\verb!\question{$2\eme$ question}!

\verb!\question{$3\eme$ question}!

\sectionExercice{Nouvel exercice sous forme de sous-section (au sens de LaTex)}
\question{$1\ere$ question}
\question{$2\eme$ question}
\question{$3\eme$ question}


		\subsection{Les réponses}
		%.............................

			Les réponse fonctionne conjointement avec la commande \verb!\version! (voir plus haut).
			Elle affiche (ou non) la réponse en rouge.
			Si on ne veut pas afficher la réponse, il est possible de mettre un texte de remplacement en option.
			Par défaut de cette option, en mode élève, le texte sera invisible, mais prendra de la place sur la page.
		\begin{verbatim}
\version{eleve}
\question{Comment rendre une reponse invisible ?}
\reponse{comme ceci !}

\version{prof}
\question{Et là, peut-on voir la réponse ?}
\reponse{Oui, on peut la voir !}

\question{Et si je veux un certain texte pour les élèves, et un autre pour le prof ?}
\version{prof}
\reponse{Réponse pour le prof}[Texte pour les élèves]

\version{eleve}
\reponse{Réponse pour le prof}[Texte pour les élèves]
		\end{verbatim}

\version{eleve}
\question{Comment rendre une reponse invisible ?}
\reponse{comme ceci !}

\version{prof}
\question{Et là, peut-on voir la réponse ?}
\reponse{Oui, on peut la voir !}

\question{Et si je veux un certain texte pour les élèves, et un autre pour le prof ?}
\version{prof}
\reponse{Réponse pour le prof}[Texte pour les élèves]

\version{eleve}
\reponse{Réponse pour le prof}[Texte pour les élèves]


\vspace{1cm}Pour les réponses faisant plusieurs lignes, on pourra utiliser l'environnement :
		\begin{verbatim}

\begin{grosseReponse}
	Réponse sur plusieurs ligne...
\end{grosseReponse}

ou éventuellement :
\begin{bigReponse}
	Ca marche aussi !
\end{bigReponse}
		\end{verbatim}
		
\version{prof}
\begin{grosseReponse}
	Réponse sur plusieurs ligne...
\end{grosseReponse}

ou éventuellement :
\begin{bigReponse}
	Ca marche aussi !
\end{bigReponse}

	\paragraph{Les pointillés :}
	
	Pour pallier aux problèmes de la commande \verb!\dotfill! (qui nécessite qu'il y ait d'autres caractères sur la même ligne), voici la commande \verb!\pointilles! :
	\begin{verbatim}
\pointilles\\
\pointilles
	\end{verbatim}
	

\noindent\pointilles\\
\pointilles

	\vspace{0.5cm}Avec une longueur définie (\verb!\linewidth! par défaut) :
	\begin{verbatim}
\pointilles[3cm]\\
\pointilles[3cm]
	\end{verbatim}
	

\noindent\pointilles[3cm]\\
\pointilles[3cm]

	\vspace{0.5cm}Avec un interligne défini : ($0.5cm$ par défaut) :
	\begin{verbatim}
\pointilles[1cm][2cm]\\
\pointilles[3cm][2cm]
	\end{verbatim}
	
\noindent\pointilles[1cm][2cm]\\
\pointilles[3cm][2cm]


	\section{algèbre relationnelle}
	%=========================
	
		\begin{tabular}{|m{0.3\linewidth}|m{0.2\linewidth}|m{0.4\linewidth}|}
			\hline
				CODE LaTex			&	RENDU				&	COMMENTAIRES
			\\\hline\hline
				\verb!\selection{a>b}{R}!	&	\selection{a>b}{R}		&	La sélection (en algèbre relationnelle)
			\\\hline
				\verb!\selection{a\is 1\and! \verb!c\is 2}{R}! 		&	\selection{a\is 1\and b\is 2}{R}		&	Idem avec une mise en forme un poil mieux sur les conditions. \verb!\and! a été renommé \ren
			\\\hline
				\verb!\selection{a\sup 1\and! \verb!c\inf 2}{R}! 	&	\selection{a\sup 1\and b\inf 2}{R}		&	Idem. \verb!\sup! et \verb!\inf! ont été renommé \ren
			\\\hline
				\verb!\selection{a\supeg 1\and! \verb!c\infeg 2}{R}! 	&	\selection{a\supeg 1\and b\infeg 2}{R}		&	Idem
			\\\hline
				\verb!\projection{attr1,attr2}{R}!	&	\projection{attr1,attr2}{R}		&	La projection (en algèbre relationnelle)
			\\\hline
				\verb!\renommage{attr1\to attr2}{R}!	&	\renommage{attr1\to attr2}{R}		&	Le renommage. \verb!\to! a été renommé \ren
			\\\hline
				\verb!\join!	&	\join		&	La jointure (raccourci de la fonction \verb!join!)
			\\\hline
				\verb!$\dom{R}$!	&	$\dom{R}$		&	Domaine d'une relation
			\\\hline
		\end{tabular}


	\section{Les extraits de code}
	%================================
	

		\subsection{Pseudo-code}
		%-------------------------------
		
		\begin{verbatim}
\begin{pseudoCode}
         Pour i allant de 0 jusqu'à 1000
                Afficher i et "ouse"
        Fin pour
\end{pseudoCode}
		\end{verbatim}

\begin{pseudoCode}
	Pour i allant de 0 jusqu'à 1000
		Afficher i et "ouse"
	Fin pour
\end{pseudoCode}

Possibilité de réduire la largeur en option :
		\begin{verbatim}
\begin{pseudoCode}[7cm]
         Pour i allant de 0 jusqu'à 1000
                Afficher i et "ouse"
        Fin pour
\end{pseudoCode}
		\end{verbatim}
123
\begin{pseudoCode}[7cm]
	Pour i allant de 0 jusqu'à 1000
		Afficher i et "ouse"
	Fin pour
\end{pseudoCode}

La liste des mots clés reconnus par le pseudo code est :

Pour, POUR, pour, allant, Allant, ALLANT, de, DE, De, jusqu, JUSQU, Jusqu, à, À, FIN, Fin, fin, Effectuer, EFFECTUER, effectuer, si, Si, SI, faire, et, Et, ET, ou,
Ou, OU, FAIRE, Faire, alors, Alors, ALORS, fonction, Fonction, FONCTION, retourner, Retourner, RETOURNER, afficher, Afficher, AFFICHER, concaténer, concatener,
Concaténer, Concatener, CONCATÉNER, CONCATENER, Nouveau, NOUVEAU, nouveau.

On peut en rajouter...


		\subsection{Code python}
		%-------------------------------
		
			\begin{verbatim}
\begin{codePython}
for in in range(1001):
        print(str(i)+"-ouse")	#Ne cherchez pas à comprendre ce code...
\end{codePython}
			\end{verbatim}
\begin{codePython}
for in in range(1001):
	print(str(i)+"-ouse")	#Ne cherchez pas à comprendre ce code...
\end{codePython}

			\begin{verbatim}
\begin{codePython}[13cm]
for in in range(1001):
        print(str(i)+"-ouse")	#Ne cherchez pas à comprendre ce code...
\end{codePython}
			\end{verbatim}
			
\begin{codePython}[13cm]
for in in range(1001):
	print(str(i)+"-ouse")	#Ne cherchez pas à comprendre ce code...
\end{codePython}


			

		\subsection{Requête SQL}
		%-------------------------------

			\begin{verbatim}
\begin{requeteSQL}
Select a,b from table1 JOIN table2 ON table1.a<table2.a AND table1.b>table2.b
\end{requeteSQL}
			\end{verbatim}
			
\begin{requeteSQL}
Select a,b from table1 JOIN table2 ON table1.a<table2.a AND table1.b>table2.b
\end{requeteSQL}
		
			\begin{verbatim}
\begin{requeteSQL}[15cm]
Select a,b from table1 JOIN table2 ON table1.a<table2.a AND table1.b>table2.b
\end{requeteSQL}
			\end{verbatim}
			
\begin{requeteSQL}[15cm]
Select a,b from table1 JOIN table2 ON table1.a<table2.a AND table1.b>table2.b
\end{requeteSQL}


		\subsection{Console de sortie}
		%-------------------------------

			\begin{verbatim}
\begin{console}
test test test
\end{console}
			\end{verbatim}
			
\begin{console}
test test test
\end{console}
		
			\begin{verbatim}
\begin{console}[10cm]
test test test
\end{console}
			\end{verbatim}
			
\begin{console}[10cm]
test test test
\end{console}


\end{document}