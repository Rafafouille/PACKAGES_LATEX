\documentclass[a4paper,12pt]{article}

	\usepackage{xcolor}
	\usepackage{listings}	\lstset{language=[LaTeX]TeX,basicstyle=\ttfamily,texcsstyle=*\color{blue},identifierstyle=\color{brown},commentstyle=\color{gray}\itshape,escapechar=!,moretexcs={}}
	\usepackage[latin1]{inputenc} 
	\usepackage{hyperref}

	\usepackage{Raf_Notations_Materiaux}


% 	\newcommand{\rac}{({\color{red}Raccourci})}
% 	\newcommand{\ren}{({\color{blue}Renommé})}

\begin{document}

	\begin{center}
		\hrule{\Large Notation Matériaux}\\\hrule
	\end{center}

	\section{Packages requis}
	%-------------------------------------
		\begin{itemize}
			\item \href{http://www.ctan.org/pkg/ifthen}{\textbf{ifthen}} : Package pour faire des compilations conditionnelles (if...then...else....)
			\item \href{http://www.ctan.org/pkg/xargs}{\textbf{xargs}} : Pour créer des commandes avec plusieurs arguments optionnels
			\item \href{http://www.ctan.org/pkg/amsmath}{\textbf{amsmath}} : Pour des notations mathématiques (notamment l'utilisation de \verb!\text! il me semble).
			\item \href{http://www.ams.org/publications/authors/tex/amsfonts}{\textbf{amsfont}} : Pour faire des maths (ensemble des reels, notamment)
		\end{itemize}

	\section{Appel du package}
	%-------------------------------------

		Le package est appelé en début de document par la commande :
		\begin{verbatim}
\usepackage{Raf_Notations_Materiaux}
		\end{verbatim}

% 	Par défaut, ce package utilise un certain nombre de notations raccourcies, susceptibles de rentrer en conflit avec d'autres packages (mais tellement plus rapide à taper !).
% 	De plus, certaines commandes ont été rebaptisées.
% 	Ces raccourcis et renommages seront cités (\rac\ ou \ren) dans les tableaux suivants.
% 	Pour ne pas créer ces raccourcis/renommage, il faut rentre l'option \verb!noRaccourci! à l'appel du package.
% 
% 	\begin{verbatim}
% usepackage[noRaccourci]{Raf_Notations_Materiaux}
% 	\end{verbatim}


	\section{Déformations / Contraintes des matériaux}
	%-----------------------------
		\noindent
		\begin{tabular}{|p{0.4\linewidth}|p{0.2\linewidth}|p{0.4\linewidth}|}
			\hline
				\textbf{Commandes}&\textbf{Rendus}&\textbf{Commentaires}
			\\\hline\hline
				\verb!\tauxDeformation!		&	\tauxDeformation		&	Déformation relative (ou taux de déformation longitudinal).
			\\\hline
				\verb!\tauxDeformation[xx]!	&	\tauxDeformation[xx]		&	Déformation relative dans la direction $x$.
			\\\hline
				\verb!\matDef!			&	\matDef				&	Raccourci direct de \verb'\tauxDeDeformation'.
			\\\hline
				\verb!\deformationElastique!	&	\deformationElastique		&	Déformation élastique.
			\\\hline
				\verb!\matDefE!			&	\matDefE			&	Raccourci de \verb!\deformationElastique!.
			\\\hline
				\verb!\defE!			&	\defE				&	Raccourci de \verb!\deformationElastique!.
			\\\hline
				\verb!\deformationPlastique!	&	\deformationPlastique		&	Déformation plastique.
			\\\hline
				\verb!\matDefP!			&	\matDefP			&	Raccourci de \verb!\deformationPlastique!.
			\\\hline
				\verb!\defP!			&	\defP				&	Raccourci de \verb!\deformationPlastique!.
			\\\hline
				\verb!\matContrainte{\sigma}!	&	\matContrainte{\sigma}		&	Contrainte (scalaire)
			\\\hline
				\verb!\matContrainte{\sigma}[xx]!&	\matContrainte{\sigma}[xx]	&	Contrainte dans la direction $x$
			\\\hline
				\verb!\contrainteNormale!	&	\contrainteNormale		&	Contrainte normale (raccourci de \verb!\matContrainte{\sigma}!)
			\\\hline
				\verb!\contrainteNormale[xx]!	&	\contrainteNormale[xx]		&	idem dans la direction $x$
			\\\hline
				\verb!\matS!,  \verb!\matS[xx]!	&	\matS, \matS[xx]				&	Raccourci direct de \verb!contrainteNormale!
			\\\hline
				\verb!\contrainteTangentielle!	&	\contrainteTangentielle				&	Contrainte tangentielle
			\\\hline
				\verb!\contrainteTangentielle[xy]!	&	\contrainteTangentielle[xy]				&	Contrainte tangentielle avec indice
			\\\hline
				\verb!\matT!,  \verb!\matT[yx]!	&	\matT, \matT[xy]				&	Raccourci direct de \verb!contrainteTangentielle!
			\\\hline
		\end{tabular}

	\section{Caractéristiques matériaux}
	%-----------------------------
		\noindent
		\begin{tabular}{|p{0.4\linewidth}|p{0.2\linewidth}|p{0.4\linewidth}|}
			\hline
				\textbf{Commandes}&\textbf{Rendus}&\textbf{Commentaires}
			\\\hline\hline
				\verb!\moduleYoung!	&	\moduleYoung	&	Module de Young.
			\\\hline
				\verb!\matE!		&	\matE		&	Raccourci \verb!moduleYoung!
			\\\hline
				\verb!\limiteElastique!	&	\limiteElastique&	Limite élastique
			\\\hline
				\verb!\matRe!		&	\matRe		&	Raccourci \verb!\limiteElastique!
			\\\hline
				{\small \verb!\limiteElastiqueDeuxPourCent!}	&	\limiteElastiqueDeuxPourCent	&	Limite élastique à $2\%$
			\\\hline
				\verb!\matRep!		&	\matRep		&	Raccourci {\small\verb!\limiteElastiqueDeuxPourcent!}
			\\\hline
				\verb!\limiteRupture!	&	\limiteRupture	&	Limite de rupture
			\\\hline
				\verb!\matRm!		&	\matRm		&	Raccourci \verb!\limiteRupture!
			\\\hline
				\verb!\limitePratique!		&	\limitePratique		&	Résistance pratique (comportant un coefficient de sécurité)
			\\\hline
				\verb!\resistancePratique!		&	\resistancePratique		&	Identique à \verb!\limitePratique!
			\\\hline
				\verb!\matRp!		&	\matRp		&	Raccourci de \verb!\limitePratique!
			\\\hline
				\verb!\allongementPourCent!&	\allongementPourCent	&	Allongement pour cent
			\\\hline
				\verb!\matAp!		&	\matAp		&	Raccourci \verb!\allongementPourCent!
			\\\hline
				\verb!\coefficientStriction!&	\coefficientStriction	&	Coefficient de striction
			\\\hline
				\verb!\matZp!		&	\matZp		&	Raccourci \verb!\coefficientStriction!
			\\\hline
				\verb!\coefficientPoisson!&	\coefficientPoisson	&	Coefficient de Poisson
			\\\hline
				\verb!\matPoisson!	&	\matPoisson		&	Raccourci \verb!\coefficientPoisson!
			\\\hline
				\verb!\coefPoisson!	&	\coefPoisson		&	Raccourci \verb!\coefficientPoisson!
			\\\hline
				\verb!\matnu!		&	\matnu		&	Raccourci \verb!\coefficientPoisson!
			\\\hline
				\verb!\matEndurance!	&	\matEndurance		&	Coefficient d'endurance
			\\\hline
		\end{tabular}


\section{Désignation matériaux}
%=======================================

		\noindent
		\begin{tabular}{|p{0.4\linewidth}|p{0.2\linewidth}|p{0.4\linewidth}|}
			\hline
				\textbf{Commandes}&\textbf{Rendus}&\textbf{Commentaires}
			\\\hline\hline
				\verb!\designationMateriaux! \verb!{C\ 25}!	&	\designationMateriaux{C\ 25}	&	Format de base de désignation de matériau.
			\\\hline
		\end{tabular}


	\subsection{Aciers non-alliés}
	%-----------------------------------

		\noindent
		\begin{tabular}{|p{0.4\linewidth}|p{0.2\linewidth}|p{0.4\linewidth}|}
			\hline
				\textbf{Commandes}&\textbf{Rendus}&\textbf{Commentaires}
			\\\hline\hline
				\verb!\acier{12}!	&	\acier{12}	&	acier non-allié (par défaut pour construction générale).
			\\\hline
				\verb!\acier[E]{12}!	&	\acier[E]{12}	&	acier non-allié (pour un usage différent).
			\\\hline
				\verb!\acier{12}[F]!	&	\acier{12}[F]	&	acier non-allié avec complément (ici : pour forgeage).
			\\\hline
				\verb!\acierMoule[12]!	&	\acierMoule{12}	&	acier non-allié moulé (s'utilise comme \verb!acier!).
			\\\hline
				\verb!\acierG[12]!	&	\acierG{12}	&	Raccourci de \verb!acierMoule!.
			\\\hline
				\verb!\acierS{12}!\newline\verb!\acierGS{12}!	&	\acierS{12}\newline\acierGS{12}	&	Raccourci vers acier non allié (non-moulé et moulé) pour construction générale.
			\\\hline
				\verb!\acierE{12}!\newline\verb!\acierGE{12}!	&	\acierE{12}\newline\acierGE{12}	&	Raccourci vers acier non allié (non-moulé et moulé) pour construction mécanique.
			\\\hline
				\verb!\acierC{12}!\newline\verb!\acierGC{12}!	&	\acierC{12}\newline\acierGC{12}	&	Raccourci vers acier non allié (non-moulé et moulé) pour traitement technique.
			\\\hline
				\verb!\acierP{12}!\newline\verb!\acierGP{12}!	&	\acierP{12}\newline\acierGP{12}	&	Raccourci vers acier non allié (non-moulé et moulé) pour appareil à pression.
			\\\hline
				\verb!\acierB{12}!\newline\verb!\acierGB{12}!	&	\acierB{12}\newline\acierGB{12}	&	Raccourci vers acier non allié (non-moulé et moulé) pour armature béton.
			\\\hline
				\verb!\acierH{12}!\newline\verb!\acierGH{12}!	&	\acierH{12}\newline\acierGH{12}	&	Raccourci vers acier non allié (non-moulé et moulé) pour en forme de pièces plates.
			\\\hline
		\end{tabular}

	\subsection{Aciers faiblement alliés}
	%------------------------

		\noindent
		\begin{tabular}{|p{0.4\linewidth}|p{0.2\linewidth}|p{0.4\linewidth}|}
			\hline
				\textbf{Commandes}&\textbf{Rendus}&\textbf{Commentaires}
			\\\hline\hline
				\verb!\acierFaiblementAllie{35}! \verb!{\Ni\Cr}[16]!	&	\acierFaiblementAllie{35}{\Ni\Cr}[16]	&	Acier faiblement allié ($0.35\%$ de carbone et $4\%$ de nickel et des traces de chrome).
			\\\hline
				\verb!\acierFA{35}{\Ni\Cr}[16]!	&	\acierFA{35}{\Ni\Cr}[16]	&	Raccourci de \verb!\acierFaiblementAllie!.
			\\\hline
		\end{tabular}

	\subsection{Aciers fortement alliés}
	%------------------------

		\noindent
		\begin{tabular}{|p{0.4\linewidth}|p{0.25\linewidth}|p{0.35\linewidth}|}
			\hline
				\textbf{Commandes}&\textbf{Rendus}&\textbf{Commentaires}
			\\\hline\hline
				\verb!\acierFortementAllie{15}! \verb!{\Cr\Ni}{18-10}!	&	\acierFortementAllie{15}{\Cr\Ni}{18-10}	&	Acier fortement allié.
			\\\hline
				\verb!\acierX{15}{\Cr\Ni}! \verb!{18-10}!			&	\acierX{15}{\Cr\Ni}{18-10}			&	Raccourci de \verb!\acierFortementAllie!.
			\\\hline
		\end{tabular}

	\subsection{Fonte}
	%------------------------

		\begin{tabular}{|p{0.4\linewidth}|p{0.25\linewidth}|p{0.35\linewidth}|}
			\hline
				\textbf{Commandes}&\textbf{Rendus}&\textbf{Commentaires}
			\\\hline\hline
				\verb!\fonte{L}{120}!	&	\fonte{L}{120}		&	Fonte (ici : à graphite lamellaire, $\matRm=120\text{MPa}$).
			\\\hline
				\verb!\fonte{L}{120}[5]!&	\fonte{L}{120}[5]	&	Fonte avec \matAp.
			\\\hline
				\verb!\fonteL{120}!	&	\fonteL{120}		&	Fonte à graphite lamellaire.
			\\\hline
				\verb!\fonteS{120}!	&	\fonteS{120}		&	Fonte à graphite sphéroïdal.
			\\\hline
				\verb!\fonteMW{120}!	&	\fonteMW{120}		&	Fonte malléable à c\oe ur blanc.
			\\\hline
				\verb!\fonteMB{120}!	&	\fonteMB{120}		&	Fonte malléable à c\oe ur noir.
			\\\hline
		\end{tabular}

	\subsection{Métaux non-ferreux}
	%-----------------------------------------

		\begin{tabular}{|p{0.4\linewidth}|p{0.25\linewidth}|p{0.35\linewidth}|}
			\hline
				\textbf{Commandes}&\textbf{Rendus}&\textbf{Commentaires}
			\\\hline\hline
				\verb!\metal{\Al}{99.5}!	&	\metal{\Al}{99.5}		&	Métal non ferreux (ici : aluminium à $99.5\%$).
			\\\hline
				\verb!\alliage{\Al}! \verb!{\Cu4\ \Mg\ \Ti}!	&	\alliage{\Al}{\Cu4\ \Mg\ \Ti}		&	Alliage d'aluminium.
			\\\hline
		\end{tabular}


	\subsection{Éléments chimiques}
	%------------------------------------------


		\begin{tabular}{|p{0.4\linewidth}|p{0.25\linewidth}|p{0.35\linewidth}|}
			\hline
				\textbf{Commandes}&\textbf{Rendus}&\textbf{Commentaires}
			\\\hline\hline
				\verb!\elementChimique{X}!	&	\elementChimique{X}		&	Format de base d'un élément chimique.
			\\\hline
				\verb!\elementChimique{X}[2]!	&	\elementChimique{X}[2]		&	idem avec nombre en indice.
			\\\hline
				\verb!\eChim{X}[2]!	&	\eChim{X}[2]		&	Raccourci direct de \verb!\elementChimique{X}!.
			\\\hline
		\end{tabular}

		\begin{tabular}{|p{0.4\linewidth}|p{0.25\linewidth}|p{0.35\linewidth}|}
			\hline
				\textbf{Commandes}&\textbf{Rendus}&\textbf{Commentaires}
			\\\hline\hline
				\verb!\Al!	&	\Al		&	Aluminium
			\\\hline
				\verb!\Ag!	&	\Ag		&	Argent
			\\\hline
				\verb!\C!	&	\C		&	Carbone
			\\\hline
				\verb!\Cr!	&	\Cr		&	Chrome
			\\\hline
				\verb!\Co!	&	\Co		&	Cobalt
			\\\hline
				\verb!\Cu!	&	\Cu		&	Cuivre
			\\\hline
				\verb!\Sn!	&	\Sn		&	Étain
			\\\hline
				\verb!\Fe!	&	\Fe		&	Fer
			\\\hline
				\verb!\Mn!	&	\Mn		&	Manganèse
			\\\hline
				\verb!\Mg!	&	\Mg		&	Magnésium
			\\\hline
				\verb!\Molybdene!	&	\Molybdene		&	Molybdène
			\\\hline
				\verb!\Ni!	&	\Ni		&	Nickel
			\\\hline
				\verb!\Au!	&	\Au		&	Or
			\\\hline
				\verb!\Pt!	&	\Pt		&	Plomb
			\\\hline
				\verb!\Pb!	&	\Pb		&	Plomb
			\\\hline
				\verb!\Si!	&	\Si		&	Silicium
			\\\hline
				\verb!\soufre!	&	\soufre		&	Soufre
			\\\hline
				\verb!\Zn!	&	\Zn		&	Zinc
			\\\hline
				\verb!\Ti!	&	\Ti		&	Titane
			\\\hline
				\verb!\phosphore!&	\phosphore		&	Phosphore
			\\\hline
				\verb!\vanadium!&	\vanadium		&	Vanadium
			\\\hline
				\verb!\tungstene!&	\tungstene		&	Tungstène
			\\\hline
		\end{tabular}


\end{document}
