\documentclass[a4paper,10pt]{article}

	\usepackage{xcolor}
	\usepackage{listings}	\lstset{language=[LaTeX]TeX,basicstyle=\ttfamily,texcsstyle=*\color{blue},identifierstyle=\color{brown},commentstyle=\color{gray}\itshape,escapechar=!,moretexcs={}}
    \usepackage[utf8]{inputenc}
	\usepackage{hyperref}


	\usepackage{Raf_Notations_MMC}
	\usepackage[francais]{babel}
	\everymath{\displaystyle}


	\newcommand{\rac}{({\color{red}Raccourci})}
	\newcommand{\ren}{({\color{blue}Renommé})}

\begin{document}





	\begin{center}
		\hrule{\Large Notations de Mécanique des Milieux Continus}\\\hrule
	\end{center}


	\section{Packages requis}
	%-------------------------------------

	\begin{itemize}
		\item \href{http://www.ctan.org/pkg/ifthen}{\textbf{ifthen}} : Package permettant une compilation à choix multiple,
		\item \href{http://www.ctan.org/pkg/mathbbol}{\textbf{mathbbol}} : Permet de d'avoir des notations mathématiques particulières avec des doubles traits (comme %$\bbepsilon$)
		%\item \href{http://tug.ctan.org/tex-archive/macros/latex/contrib/xargs}{\textbf{xarg}} : Package permettant de créer des commandes à plusieurs arguments optionnels.
		%\item \href{http://www.ams.org/publications/authors/tex/amsfonts}{\textbf{amsfonts}} : Package qui ajoute des polices d'écritures mathématiques.
		%\item \href{http://www.ams.org/publications/authors/tex/amslatex}{\textbf{amsmath}} : Package qui ajoute des fonctions mathématiques non-standards.
		%\item \href{http://www.ctan.org/pkg/mathrsfs}{\textbf{mathrsfs}} : Package qui rajoute des polices d'écritures mathématiques.
		%\item \href{http://www.ctan.org/pkg/color}{\textbf{color}} : Package permettant de mettre en couleur du texte, des lignes, etc.
		%\item \href{http://www.ctan.org/pkg/xspace}{\textbf{xspace}} : Package permettant de mettre des espaces après les commandes.
		%\item \href{http://www.ctan.org/pkg/xstring}{\textbf{xstring}} : Package permettant travailler sur les chaînes de caractères (chercher/remplacer, etc.)
	\end{itemize}

	\section{Appel du package}
	%-------------------------------------

	Le package est appelé en début de document par la commande :
	\begin{verbatim}
\usepackage{Raf_Notations_Maths}
	\end{verbatim}

	Par défaut, ce package utilise un certain nombre de notations raccourcies, susceptibles de rentrer en conflit avec d'autre package (mais tellement plus rapide à taper !).
	De plus, certaines commandes ont été rebaptisée.
	Ces raccourcis et renommages seront cités (\rac\ ou \ren) dans les tableaux suivants.
	Pour ne pas créer ces raccourcis/renommage, il faut rentre l'option \verb!noRaccourci! à l'appel du package.

	\begin{verbatim}
usepackage[noRaccourci]{Raf_Notations_MMC}
	\end{verbatim}
	
	
	
	\section{Transformation}
	%-----------------------------------------------
    \noindent
	\begin{tabular}{|p{0.35\linewidth}|p{0.3\linewidth}|p{0.3\linewidth}|}
		\hline
			\textbf{Commandes}&\textbf{Rendus}&\textbf{Commentaires}
		\\\hline\hline
			\verb!\fTransfo{X}!			&	\fTransfo{X}			&	
		\\\hline
			\verb!\dep{X}!			&	\dep{X}			&	
		\\\hline
	\end{tabular}
	
	\section{Opérateur}
	%-----------------------------------------------
	\begin{tabular}{|p{0.35\linewidth}|p{0.3\linewidth}|p{0.3\linewidth}|}
		\hline
			\textbf{Commandes}&\textbf{Rendus}&\textbf{Commentaires}
		\\\hline\hline
			\verb!\grad{f}! ou 	\verb!\grad[][nabla]{f}!		&	\grad{f} ou \grad[][nabla]{f}			&	
		\\\hline
			\verb!\grad[1]{f}! ou 	\verb!\grad[1][nabla]{f}!			&	\grad[1]{f}	ou \grad[1][nabla]{f}		&	
		\\\hline
			\verb!\grad[2]{f}! ou 	\verb!\grad[2][nabla]{f}!			&	\grad[2]{f}	ou \grad[2][nabla]{f}			&	
		\\\hline
			\verb!\grad[3]{f}! ou 	\verb!\grad[3][nabla]{f}!			&	\grad[3]{f}	ou \grad[3][nabla]{f}			&	
		\\\hline
			\verb!\grad[4]{f}! ou 	\verb!\grad[4][nabla]{f}!			&	\grad[4]{f}	ou \grad[4][nabla]{f}			&	
		\\\hline
			\verb!\grad[25]{f}! ou 	\verb!\grad[25][nabla]{f}!			&	\grad[25]{f}	ou \grad[25][nabla]{f}			&	
		\\\hline
			\verb!\dfij{i}{j}!			&	\dfij{i}{j}			&	
		\\\hline 
			\verb!\dfij[f]{i}{j}!			&	\dfij[f]{i}{j}			&	
		\\\hline
			\verb!\div!			       &	\div			&	Divergence \ren sans arguement
		\\\hline
			\verb!\div[A]!			&	\div[A]		&	Divergence \ren avec un seul argument
		\\\hline
			\verb!\div[0][A]! \verb!\div[1][A]! \verb!\div[2][A]! \verb!\div[3][A]!			&	\div[0][A]\div[1][A]\div[2][A]\div[3][A]		&	Divergence \ren avec deux argument (dimension + argument)
		\\\hline
			\verb!\rot!			&	\rot		&	 Rotationnel
		\\\hline
			\verb!\rot[A]!			&  \rot[A]			&	 Rotationnel
		\\\hline
            \verb!\laplacien{\vecteur{A}}!  &   
            \laplacien{\vecteur{A}} &
		\\\hline
	\end{tabular}
	
	\section{Tenseurs}
	%-------------------------------------------
	\noindent
	\begin{tabular}{|p{0.35\linewidth}|p{0.3\linewidth}|p{0.3\linewidth}|}
		\hline
			\textbf{Commandes}&\textbf{Rendus}&\textbf{Commentaires}
		\\\hline\hline
			\verb!\dbarre{A}!			&	\dbarre{A}			&	Double barre
		\\\hline
			\verb!\symbolTenseurF!			&	\symbolTenseurF			&	Gradient de transformation
		\\\hline
			\verb!\tenseurF!		&	\tenseurF			&	Gradient de transformation
		\\\hline
			\verb!\symbolTenseurE!		&	\symbolTenseurE			&	Tenseur de Green-Lagrange
		\\\hline
			\verb!\tenseurE!		&	\tenseurE			&	Tenseur de Green-Lagrange
		\\\hline
			\verb!\symbolTenseurEps!		&	\symbolTenseurEps			&	Tenseur de Green-Lagrange en HPP
		\\\hline
			\verb!\tenseurEps!		&	\tenseurEps			&	Tenseur de Green-Lagrange en HPP
		\\\hline
			\verb!\symbolTenseure!		&	\symbolTenseure			&	Tenseur de Euler-Almansi
		\\\hline
			\verb!\tenseure!		&	\tenseure			&	Tenseur de Euler-Almansi
		\\\hline
			\verb!\symbolTenseurI!		&	\symbolTenseurI			&	Tenseur Identité
		\\\hline
			\verb!\tenseurI!		&	\tenseurI			&	Tenseur Identité
		\\\hline
			\verb!\symbolTenseurC!		&	\symbolTenseurC			&	Tenseur de dilatation de Cauchy Green
		\\\hline
			\verb!\tenseurC!		&	\tenseurC			&	Tenseur de dilatation de Cauchy Green
		\\\hline
			\verb!\symbolTenseurb!		&	\symbolTenseurb			&	Tenseur de dilatation de Cauchy Green gauche
		\\\hline
			\verb!\tenseurb!		&	\tenseurb			&	Tenseur de dilatation de Cauchy Green gauche
		\\\hline
			\verb!\tenseurSigma!		&	\tenseurSigma			&	Tenseur des contraintes de Cauchy
		\\\hline
			\verb!\tenseurNul!		&	\tenseurNul			&	Tenseur nul
		\\\hline
			\verb!\begin{tenseur}1&2&3\\!
			\verb!4&5&6\\7&8&9!
			\verb!\end{tenseur}!		&	$\begin{tenseur}1&2&3\\4&5&6\\7&8&9\end{tenseur}$			&	
		\\\hline
	\end{tabular}

	\noindent
	\begin{tabular}{|p{0.35\linewidth}|p{0.3\linewidth}|p{0.3\linewidth}|}
		\hline
			\textbf{Commandes}&\textbf{Rendus}&\textbf{Commentaires}
		\\\hline\hline
            \verb!\sig{i}{j}, \sig12!   &   \sig{i}{j}, \sig12  &   Composantes du tenseur de contrainte
		\\\hline
            \verb!\eps{i}{j}, \eps12!   &   \eps{i}{j}, \eps12  &   Composantes du tenseur de déformation
		\\\hline
	\end{tabular}

    



\end{document}
