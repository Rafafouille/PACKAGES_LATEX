\documentclass[a4paper,10pt]{article}

	\usepackage{xcolor}
	\usepackage{listings}	\lstset{language=[LaTeX]TeX,basicstyle=\ttfamily,texcsstyle=*\color{blue},identifierstyle=\color{brown},commentstyle=\color{gray}\itshape,escapechar=!,moretexcs={}}
	\usepackage[latin1]{inputenc}
	\usepackage{hyperref}

	\usepackage{Raf_Notations_Autom}

	\everymath{\displaystyle}
	
	
	\newcommand{\rac}{({\color{red}Raccourci})}
	\newcommand{\ren}{({\color{blue}Renommé})}
	
	
	
\begin{document}

	\begin{center}
		\hrule{\Large Notations d'automatisme}\\\hrule
	\end{center}
	(Version du 14/08/16)

	\section{Packages requis}
	%-------------------------------------

		\begin{itemize}
			\item \href{https://www.ctan.org/pkg/xargs}{\textbf{xargs}} : Pour faire (notamment) des commandes à plusieurs arguments optionnels.
			\item \href{https://www.ctan.org/pkg/ifthen}{\textbf{ifthen}} : Package pour faire des compilations conditionnelles.
			\item \href{http://www.ams.org/publications/authors/tex/amslatex}{\textbf{amsmath}} : Package de notations mathématiques.
			\item \href{http://www.ctan.org/pkg/pgf}{\textbf{tikz}} : Package pour faire des dessins (avec library \verb!calc!)
		\end{itemize}

	\section{Appel du package}
	%-------------------------------------

		Le package est appelé en début de document par la commande :
		\begin{verbatim}
\usepackage{Raf_Notations_Autom}
		\end{verbatim}

		
		% PAS ENCORE DE RACCOURCIS !!!!!!!!!!!!!!!!!!!!
		
% 	Par défaut, ce package utilise un certain nombre de notations raccourcies, susceptibles de rentrer en conflit avec d'autres packages (mais tellement plus rapide à taper !).
% 	De plus, certaines commandes ont été rebaptisées.
% 	Ces raccourcis et renommages seront cités (\rac\ ou \ren) dans les tableaux suivants.
% 	Pour ne pas créer ces raccourcis/renommage, il faut rentrer l'option \verb!noRaccourci! à l'appel du package.
% 
% 	\begin{verbatim}
% usepackage[noRaccourci]{Raf_Notations_Autom}
% 	\end{verbatim}


	\section{Numération}
	%-------------------------------------------
	\noindent
	\begin{tabular}{|p{0.35\linewidth}|p{0.3\linewidth}|p{0.3\linewidth}|}
		\hline
			\textbf{Commandes}&\textbf{Rendus}&\textbf{Commentaires}
		\\\hline\hline
			\verb!lBase{20}[n]!			&	\lBase{20}[n]			&	Nombre dans une base $n$
		\\\hline
			\verb!lBase{20}!			&	\lBase{20}			&	idem, avec par défaut : $n=10$
		\\\hline
			\verb!\binaire{100111}!			&	\binaire{100111}		&	Raccourci de \verb!\lBase{...}[2]!
		\\\hline
			\verb!\lBin{100111}!			&	\lBin{100111}			&	Raccourci direct de \verb!\binaire!
		\\\hline
			\verb!\hexadecimal{1A2B3C}!		&	\hexadecimal{1A2B3C}		&	Raccourci de \verb!\lBase{...}[16]!
		\\\hline
			\verb!\lHex{1A2B3C}!			&	\lHex{1A2B3C}			&	Raccourci direct de \verb!\hexadecimal!
		\\\hline
			\verb!\poseDivision! \verb!{312}{100}{12}{3}!			&	\poseDivision{312}{100}{12}{3}			&	Pose une division
		\\\hline
			\verb!\poseDivision! \verb!{5}{2}{1}! \verb!{\poseDivision{2}{2}{0}! \verb!{\poseDivision{1}{2}{1}! \verb!{0}}}!			&	\poseDivision{5}{2}{1}{\poseDivision{2}{2}{0}{\poseDivision{1}{2}{1}{0}}}			&	Pose plusieurs divisions en cascade
		\\\hline
	\end{tabular}

	\section{Opérateurs Logiques}
	%-------------------------------------------
	\noindent
	\begin{tabular}{|p{0.35\linewidth}|p{0.3\linewidth}|p{0.3\linewidth}|}
		\hline
			\textbf{Commandes}&\textbf{Rendus}&\textbf{Commentaires}
		\\\hline\hline
			\verb!a\ET b!			&	a\ET b			&	Opérateur ``et''
		\\\hline
			\verb!a\AND b!			&	a\AND b			&	idem que \verb!\ET!
		\\\hline
			\verb!a\OU b!			&	a\OU b			&	Opérateur ``ou''
		\\\hline
			\verb!a\OR b!			&	a\OR b			&	idem que \verb!\OU!
		\\\hline
			\verb!\NON{a\OU b}!		&	\NON{a\OU b}		&	Opérateur ``non''
		\\\hline
			\verb!\NOT{b}!			&	\NOT{b}			&	idem que \verb!\NON!
		\\\hline
			\verb!\NOR ab!			&	\NOR ab			&	Opérateur ``non-ou''
		\\\hline
			\verb!\NAND ab!			&	\NAND ab		&	Opérateur ``non-et''
		\\\hline
			\verb!\XOR ab!			&	\XOR ab			&	Opérateur ``ou-exclusif''
		\\\hline
			\verb!\XNOR ab!			&	\XNOR ab		&	Opérateur ``et-inclusif''
		\\\hline
	\end{tabular}

	\section{Séquentiel}
	%-------------------------------------------
	\noindent
	\begin{tabular}{|p{0.35\linewidth}|p{0.3\linewidth}|p{0.3\linewidth}|}
		\hline
			\textbf{Commandes}&\textbf{Rendus}&\textbf{Commentaires}
		\\\hline\hline
			\verb!\frontMontant{a}!			&	\frontMontant{a}		&	Front montant d'un événement.
		\\\hline
			\verb!\frontDescendant{a}!			&	\frontDescendant{a}	&	Front descendant d'un événement.
		\\\hline
	\end{tabular}
\end{document}
