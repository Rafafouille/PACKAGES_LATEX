\documentclass[a4paper,10pt]{article}

	\usepackage{xcolor}
	\usepackage{listings}	\lstset{language=[LaTeX]TeX,basicstyle=\ttfamily,texcsstyle=*\color{blue},identifierstyle=\color{brown},commentstyle=\color{gray}\itshape,escapechar=!,moretexcs={}}
	\usepackage[latin1]{inputenc}
	\usepackage{hyperref}

	\usepackage{Raf_Notations_Energetique}

	\everymath{\displaystyle}


	\newcommand{\rac}{({\color{red}Raccourci})}
	\newcommand{\ren}{({\color{blue}Renommé})}

\begin{document}

	\begin{center}
		\hrule{\Large Notations Mathématiques ``de base''}\\\hrule
	\end{center}

	(Version du 27/04/13)

	\section{Packages requis}
	%-------------------------------------

	\begin{itemize}
		\item \href{http://www.ctan.org/pkg/ifthen}{\textbf{ifthen}} : Package permettant une compilation à choix multiple,
		%\item \href{http://tug.ctan.org/tex-archive/macros/latex/contrib/xargs}{\textbf{xarg}} : Package permettant de créer des commandes à plusieurs arguments optionnels.
		%\item \href{http://www.ams.org/publications/authors/tex/amsfonts}{\textbf{amsfonts}} : Package qui ajoute des polices d'écritures mathématiques.
		%\item \href{http://www.ams.org/publications/authors/tex/amslatex}{\textbf{amsmath}} : Package qui ajoute des fonctions mathématiques non-standards.
		\item \href{http://www.ctan.org/pkg/mathrsfs}{\textbf{mathrsfs}} : Package qui rajoute des polices d'écritures mathématiques.
		%\item \href{http://www.ctan.org/pkg/color}{\textbf{color}} : Package permettant de mettre en couleur du texte, des lignes, etc.
		\item \href{http://enseignement.allais.eu/page-latex}{\textbf{Raf\_Notations\_Maths}} : Un autre de mes packages, permettant de faire des notations mathématiques
	\end{itemize}

	\section{Appel du package}
	%-------------------------------------

	Le package est appelé en début de document par la commande :
	\begin{verbatim}
\usepackage{Raf_Notations_Energetique}
	\end{verbatim}

	Par défaut, ce package utilise un certain nombre de notations raccourcies, susceptibles de rentrer en conflit avec d'autre package (mais tellement plus rapide à taper !).
	De plus, certaines commandes ont été rebaptisées.
	Ces raccourcis et renommages seront cités (\rac\ ou \ren) dans les tableaux suivants.
	Pour ne pas créer ces raccourcis/renommage, il faut rentre l'option \verb!noRaccourci! à l'appel du package.

	\begin{verbatim}
\usepackage[noRaccourci]{Raf_Notations_Energetique}
	\end{verbatim}

	\section{Puissances}
	%-------------------------------------------
	\noindent
	
	\begin{tabular}{|p{0.35\linewidth}|p{0.3\linewidth}|p{0.3\linewidth}|}
		\hline
			\textbf{Commandes}&\textbf{Rendus}&\textbf{Commentaires}
		\\\hline\hline
			\verb!\PCallig!			&	\PCallig			&	\PCallig\ calligraphié
		\\\hline
			\verb!\puissance{F}{V}!		&	\puissance{F}{V}		&	Puissance d'une force $F$, pour une vitesse $V$.
		\\\hline
			\verb!\puissance{F}{}!		&	\puissance{F}{}			&	Puissance d'une force $F$.
		\\\hline
			\verb!\puissance{F}{}[t]!	&	\puissance{F}{}[t]		&	Puissance instantanée à la date $t$.
		\\\hline
			\verb!\puissance{S_1}[S_2]{R}!		&	\puissance{S_1}[S_2]{R}			&	Puissance d'une action mécanique de $S_1$ sur $S_2$
		\\\hline
			\verb!\puissance{S_1}[S_2]{}!		&	\puissance{S_1}[S_2]{}			&	Idem, sans référentiel
		\\\hline
			\verb!\P{F}{}[t]!		&	\P{F}{}[t]			&	Raccourci direct de \verb!\puissance! \rac -- \ren
		\\\hline
			\verb!\puissanceMutuelle{S_1}! \verb!{S_2}!		&	\puissanceMutuelle{S_1}{S_2}			&	Puissance des actions mutuelles.
		\\\hline
			\verb!\pMutuelle{S_1}{S_2}!		&	\pMutuelle{S_1}{S_2}			&	Identique à \verb!\puissanceMutuelle!
		\\\hline
			\verb!\Pint{\Sigma}!		&	\Pint{\Sigma}			&	Puissance intérieure à l'ensemble $\Sigma$.
		\\\hline
			\verb!\Pint{}!		&	\Pint{}			&	Idem, sans préciser l'ensemble isolé.
		\\\hline
	\end{tabular}


	\section{Travail}
	%-------------------------------------------
	\noindent
	
	\begin{tabular}{|p{0.35\linewidth}|p{0.3\linewidth}|p{0.3\linewidth}|}
		\hline
			\textbf{Commandes}&\textbf{Rendus}&\textbf{Commentaires}
		\\\hline\hline
			\verb!\travail{F}{d}!		&	\travail{F}{d}			&	travail d'une force $F$, pour un déplacement $d$.
		\\\hline
			\verb!\travail{F}{}!		&	\travail{F}{}			&	travail d'une force $F$.
		\\\hline
			\verb!\travail{F}{t_1}[t_2]!	&	\travail{F}{t_1}[t_2]		&	travail d'une force $F$, pour un déplacement entre l'instant $t_1$ et l'instant $t_2$.
		\\\hline
			\verb!\Wk{F}{}!			&	\Wk{F}{}			&	Raccourci direct de \verb!\travail! \rac. Le \verb!\W! a été gardée pour les Watt. \verb!Wk! vient de \verb!Work!.
		\\\hline
	\end{tabular}

	\section{Énegie Cinétique}
	%-------------------------------------------
	
	\noindent
	\begin{tabular}{|p{0.35\linewidth}|p{0.3\linewidth}|p{0.3\linewidth}|}
		\hline
			\textbf{Commandes}&\textbf{Rendus}&\textbf{Commentaires}
		\\\hline\hline
			\verb!\eCinetique{S}{R_0}!		&	\eCinetique{S}{R_0}			&	Énergie cinétique d'une ensemble $S$ par rapport à un référentiel $R_0$. 
		\\\hline
			\verb!\eCin{S}{R_0}!		&	\eCin{S}{R_0}			&	Raccourci de \verb!\eCinetique!
		\\\hline
	\end{tabular}
	
	
	
\end{document}
