\documentclass[a4paper,12pt]{article}

	\usepackage{xcolor}
	\usepackage{listings}	\lstset{language=[LaTeX]TeX,basicstyle=\ttfamily,texcsstyle=*\color{blue},identifierstyle=\color{brown},commentstyle=\color{gray}\itshape,escapechar=!,moretexcs={}}
	\usepackage[latin1]{inputenc} 
	\usepackage{hyperref}

	\usepackage{Raf_Notations_Cotation}



	\newcommand{\rac}{({\color{red}Raccourci})}
	\newcommand{\ren}{({\color{blue}Renommé})}

\begin{document}

	\begin{center}
		\hrule{\Large Notations Cotation Fonctionnelle}\\\hrule
	\end{center}


	\section{Packages requis}
	%-------------------------------------

		\begin{itemize}
			\item \href{http://www.ctan.org/pkg/ifthen}{\textbf{ifthen}} : Package pour faire des compilations conditionnelles (if...then...else....)
			\item \href{http://www.ctan.org/pkg/amsmath}{\textbf{amsmath}} : Pour des notations mathématiques (notamment l'utilisation de \verb!\text! il me semble).
			\item \href{http://www.ctan.org/pkg/pgf}{\textbf{tikz}} : Package pour faire des dessins (avec library \verb!calc!)
			\item \href{http://www.ctan.org/pkg/calc}{\textbf{calc}} : Permet de faire des petits calculs au moment de la compilation
			\item \href{http://www.ctan.org/pkg/xargs}{\textbf{xargs}} : Pour créer des commandes avec plusieurs arguments optionnels
			\item \href{http://www.ctan.org/pkg/array}{\textbf{array}} : Package qui rajoute des possibilités aux tableaux.
		\end{itemize}
		
		
	\section{Appel du package}
	%-------------------------------------

		Le package est appelé en début de document par la commande :
		\begin{verbatim}
\usepackage{Raf_Notations_Cotation}
		\end{verbatim}

		Par défaut, ce package utilise un certain nombre de notations raccourcies, susceptibles de rentrer en conflit avec d'autres packages (mais tellement plus rapide à taper !).
		De plus, certaines commandes ont été rebaptisées.
		Ces raccourcis et renommages seront cités (\rac\ ou \ren) dans les tableaux suivants.
		Si cela devait poser problème, pour ne pas créer ces raccourcis/renommage, il faut rentre l'option \verb!noRaccourci! à l'appel du package.

		\begin{verbatim}
usepackage[noRaccourci]{Raf_Notations_Cotation}
		\end{verbatim}


	\section{Raccourcis notations}
	%-----------------------------
		\noindent
		\begin{tabular}{|p{0.3\linewidth}|p{0.1\linewidth}|p{0.5\linewidth}|}
			\hline
				\textbf{Commandes}&\textbf{Rendus}&\textbf{Commentaires}
			\\\hline\hline
				\verb!\dNominal!			&	\dNominal	&	Diamètre nominal
			\\\hline
				\verb!\dNom ou \dnom!			&	\dNom ou \dnom	&	Identique à \verb!\dNominal!\rac
			\\\hline
				\verb!\dMoyen!				&	\dMoyen		&	Diamètre moyen
			\\\hline
				\verb!\dMoy ou \dmoy!			&	\dMoy ou \dmoy	&	Identique à \verb!\dMoy!\rac
			\\\hline
				\verb!\dInferieur!			&	\dInferieur	&	Diamètre inférieur
			\\\hline
				\verb!\dInf ou \dinf!			&	\dInf ou \dinf	&	Identique à \verb!\dInferieur!\rac
			\\\hline
				\verb!\dSuperieur!			&	\dSuperieur	&	Diamètre supérieur
			\\\hline
				\verb!\dSup ou \dsup!			&	\dSup ou \dsup	&	Identique à \verb!\dSuperieur!\rac
			\\\hline
				\verb!\dInterieur!			&	\dInterieur	&	Diamètre intérieur
			\\\hline
				\verb!\dInt ou \dint!			&	\dInt ou \dint	&	Identique à \verb!\dInterieur!\rac
			\\\hline
				\verb!\dExterieur!			&	\dExterieur	&	Diamètre extérieur
			\\\hline
				\verb!\dExt ou \dext!			&	\dExt ou \dext	&	Identique à \verb!\dExterieur!\rac
			\\\hline
		\end{tabular}

	\section{Symboles}
	%-----------------------------

		\subsection{Paramètres}
		%.................................

			\noindent
			\begin{tabular}{|p{0.5\linewidth}|p{0.1\linewidth}|p{0.3\linewidth}|}
				\hline
					\textbf{Commandes}&\textbf{Rendus}&\textbf{Commentaires}
				\\\hline\hline
					\verb!\tSymbCot!			&	\tSymbCot			&	Taille des symboles (par défaut \tSymbCot)
				\\\hline
					\verb!\renewcommand{\tSymbCot}{10cm}!	&	\renewcommand{\tSymbCot}{1cm}	&	Redéfini la taille des symboles
				\\\hline
					\verb!\eSymbCot!			&	\eSymbCot			&	Coefficient (pourcentage) d'épaisseur des traits par rapport à \tSymbCot\ (\eSymbCot\ par défaut).
				\\\hline
					\verb!\renewcommand{\eSymbCot}{0.3}!	&	\renewcommand{\eSymbCot}{0.3}	&	Redéfini l'épaisseur des traits.
				\\\hline
					\verb!\cotationApply!			&	\cotationApply			&	Fonction qu'il faut appeler pour appliquer les modification des paramètres ci-dessus.
				\\\hline
			\end{tabular}
	
		\subsection{Défauts de forme}
		%.....................................

			\noindent
			\begin{tabular}{|p{0.4\linewidth}|p{0.2\linewidth}|p{0.3\linewidth}|}
				\hline
					\textbf{Commandes}&\textbf{Rendus}&\textbf{Commentaires}
				\\\hline\hline
					\verb!\rectitude!	&	\rectitude	&	Symbole rectitude
				\\\hline
					\verb!\planeite!	&	\planeite	&	Symbole planeïté
				\\\hline
					\verb!\circularite!	&	\circularite	&	Symbole circularité
				\\\hline
					\verb!\cylindricite!	&	\cylindricite	&	Symbole cylindricité
				\\\hline
					\verb!\profilLigne!	&	\profilLigne	&	Symbole profile d'une ligne
				\\\hline
					\verb!\profilSurface!	&	\profilSurface	&	Symbole profile d'une surface
				\\\hline
			\end{tabular}

		\subsection{Défauts d'orientation}
		%..........................................

			\noindent
			\begin{tabular}{|p{0.4\linewidth}|p{0.2\linewidth}|p{0.3\linewidth}|}
				\hline
					\textbf{Commandes}&\textbf{Rendus}&\textbf{Commentaires}
				\\\hline\hline
					\verb!\parallelisme!	&	\parallelisme	&	Symbole parallélisme
				\\\hline
					\verb!\perpendicularite!&	\perpendicularite&	Symbole parallélisme
				\\\hline
					\verb!\inclinaison!	&	\inclinaison	&	Symbole inclinaison
				\\\hline
			\end{tabular}

		\subsection{Défauts de position}
		%..........................................

			\noindent
			\begin{tabular}{|p{0.4\linewidth}|p{0.2\linewidth}|p{0.3\linewidth}|}
				\hline
					\textbf{Commandes}&\textbf{Rendus}&\textbf{Commentaires}
				\\\hline\hline
					\verb!\concentricite!	&	\concentricite	&	Symbole concentricité
				\\\hline
					\verb!\coaxialite!	&	\coaxialite	&	Symbole coaxialité
				\\\hline
					\verb!\symetrie!	&	\symetrie	&	Symbole symétrie
				\\\hline
					\verb!\localisation!	&	\localisation	&	Symbole localisation
				\\\hline
			\end{tabular}

		\subsection{Tolérance de battement}
		%..........................................

			\noindent
			\begin{tabular}{|p{0.4\linewidth}|p{0.2\linewidth}|p{0.3\linewidth}|}
				\hline
					\textbf{Commandes}&\textbf{Rendus}&\textbf{Commentaires}
				\\\hline\hline
					\verb!\battementSimple!	&	\battementSimple	&	Symbole battement simple
				\\\hline
					\verb!\battement!	&	\battement		&	Symbole battement simple(raccourci de \verb!\battementSimple!).
				\\\hline
					\verb!\battementTotal!	&	\battementTotal		&	Symbole battement total
				\\\hline
			\end{tabular}


		\subsection{Tolérancement}
		%..........................................

			\noindent
			\begin{tabular}{|p{0.4\linewidth}|p{0.3\linewidth}|p{0.2\linewidth}|}
				\hline
					\textbf{Commandes}&\textbf{Rendus}&\textbf{Commentaires}
				\\\hline\hline
					\verb!\specification{\planeite}! \verb!{0.1}!	&	\specification{\planeite}{0.1}	&	Tolérance sans référence
				\\\hline
					\verb!\specification! \verb!{\localisation}{0.1}[A]!	&	\specification{\localisation}{0.1}[A]		&	Tolérance avec référence
				\\\hline
					\verb!\specification! \verb!{\localisation}{0.1}[A][B]!	&	\specification{\localisation}{0.1}[A][B]	&	Tolérance avec 2 références
				\\\hline
					\verb!\specification! \verb!{\localisation}{0.1}! \verb![A][B][C]!	&	\specification{\localisation}{0.1}[A][B][C]	&	Tolérance avec 2 références
				\\\hline
					\verb!\specification! \verb![$2\times\emptyset20$]! \verb!{\localisation}{0.1}[A]!	&	\specification[$2\times\emptyset20$]{\localisation}{0.1}[A]	&	Tolérance avec référence
				\\\hline
			\end{tabular}

		\subsection{Enveloppe -- Maxi-matière -- etc.}
		%..........................................

		\noindent
			\begin{tabular}{|p{0.4\linewidth}|p{0.2\linewidth}|p{0.3\linewidth}|}
				\hline
					\textbf{Commandes}&\textbf{Rendus}&\textbf{Commentaires}
				\\\hline\hline
					\verb!\lettreEntouree{a}!	&	\lettreEntouree{a}	&	Permet de mettre une lettre dans un cercle, pour les notations suivantes.
				\\\hline
			\end{tabular}
		
			\noindent
			\begin{tabular}{|p{0.4\linewidth}|p{0.2\linewidth}|p{0.3\linewidth}|}
				\hline
					\textbf{Commandes}&\textbf{Rendus}&\textbf{Commentaires}
				\\\hline\hline
					\verb!\enveloppe!	&	\enveloppe	&	Symbole d'exigence d'enveloppe
				\\\hline
					\verb!\maxiMatiere!	&	\maxiMatiere	&	Symbole du maximum de matière
				\\\hline
					\verb!\miniMatiere!	&	\miniMatiere	&	Symbole du minimum de matière
				\\\hline
					\verb!\toleranceProjetee!&	\toleranceProjetee	&Symbole d'une tolérance projetée
				\\\hline
					\verb!\etatLibre!	&	\etatLibre	&	Condition de l'état libre
				\\\hline
			\end{tabular}

			
		\subsection{Matrice GPS}
		%......................................
			\begin{verbatim}
\begin{matriceGPS}
	\specification{\profilSurface}{0.1}
	&	Profil de surface
	&	Surface réputée conique
	&	-
	&	-
	& surface comprise entre deux cône coaxiaux distant de 0.1
\end{matriceGPS}
			\end{verbatim}
			\begin{matriceGPS}
				{\specification{\profilSurface}{0.1}}
					Profil de surface
				&	Surface réputée conique
				&	-
				&	-
				& surface comprise entre deux cône coaxiaux distant de 0.1
				%\\
				%\hline
			\end{matriceGPS}
		
\end{document}
