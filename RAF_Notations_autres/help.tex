\documentclass[a4paper,12pt]{article}

	\usepackage[french]{babel}
	\usepackage{xcolor}
	\usepackage{listings}	\lstset{language=[LaTeX]TeX,basicstyle=\ttfamily,texcsstyle=*\color{blue},identifierstyle=\color{brown},commentstyle=\color{gray}\itshape,escapechar=!,moretexcs={}}
	\usepackage[latin1]{inputenc}
	\usepackage{hyperref}
	

	\usepackage{Raf_Notations_Autres}


	\newcommand{\rac}{({\color{red}Raccourci})}
	\newcommand{\ren}{({\color{blue}Renomm�})}


\begin{document}

	\begin{center}
		\hrule{\Large Autres Notations}\\\hrule
	\end{center}
	
	\section{Packages requis}
	%-------------------------------------

		\begin{itemize}
			\item \href{http://www.ctan.org/pkg/ifthen}{\textbf{ifthen}} : Package pour faire des compilations conditionnelles (if...then...else....)
			\item \href{http://www.ctan.org/pkg/amsmath}{\textbf{amsmath}} : Pour des notations math�matiques (notamment l'utilisation de \verb!\text! il me semble).
			\item \href{https://www.ctan.org/pkg/etoolbox}{\textbf{etoolbox}} : Permet notamment de tester si une commande existe avec \ifdef .
			\item \href{https://www.ctan.org/pkg/xspace}{\textbf{xspace}} : Permet de g�rer les espaces apr�s les commandes.
			\item \href{https://www.ctan.org/pkg/marvosym}{\textbf{marvosym}}: Plusieurs symboles (dont le clic de souris)
		\end{itemize}
		
		
	\section{Appel du package}
	%-------------------------------------

		Le package est appel� en d�but de document par la commande :
		\begin{verbatim}
\usepackage{Raf_Notations_Autres}
		\end{verbatim}

		Par d�faut, ce package utilise un certain nombre de notations raccourcies, susceptibles de rentrer en conflit avec d'autres packages (mais tellement plus rapide � taper !).
		De plus, certaines commandes ont �t� rebaptis�es.
		Ces raccourcis et renommages seront cit�s (\rac\ ou \ren) dans les tableaux suivants.
		Si cela devait poser probl�me, pour ne pas cr�er ces raccourcis/renommage, il faut rentre l'option \verb!noRaccourci! � l'appel du package.

		\begin{verbatim}
usepackage[noRaccourci]{Raf_Notations_Autres}
		\end{verbatim}

	\section{Caract�res sp�ciaux}
	%----------------------------------
		\noindent
		\begin{tabular}{|p{0.2\linewidth}|p{0.3\linewidth}|p{0.4\linewidth}|}
			\hline
				\textbf{Commandes}&\textbf{Rendus}&\textbf{Commentaires}
			\\\hline\hline
				\verb!\Mu!	&	\Mu	&	$\mu$ majuscule
			\\\hline
				\verb!\antislash!	&	$\antislash$	&	Antislash
			\\\hline
				\verb!\clic!	&	\clic	&	Clic de souris
			\\\hline
		\end{tabular}	

	\section{Mise en forme}
	%----------------------------------
		\noindent
		\begin{tabular}{|p{0.4\linewidth}|p{0.3\linewidth}|p{0.3\linewidth}|}
			\hline
				\textbf{Commandes}&\textbf{Rendus}&\textbf{Commentaires}
			\\\hline\hline
				\verb!\gras{texte}!	&	\gras{texte}	&	Gras
			\\\hline
				\verb!\italic{texte}!	&	\italic{texte}	&	Italic
			\\\hline
				\verb!\souligne{texte}!	&	\souligne{texte}&	Soulign�
			\\\hline
				\verb!\bouton{texte}!	&	\bouton{texte}	&	Bouton (pour illustrer une commande en informatique, par exemple)
			\\\hline
				\verb!\toutpetit{texte}!&	\toutpetit{texte}	&	Tout petit
			\\\hline
				\verb!\PETIT{texte}!	&	\PETIT{texte}	&	tr�s tr�s petit
			\\\hline
				\verb!\Petit{texte}!	&	\Petit{texte}	&	tr�s petit
			\\\hline
				\verb!\petit{texte}!	&	\petit{texte}	&	petit
			\\\hline
				\verb!\normal{texte}!	&	\normal{texte}	&	taille normale
			\\\hline
				\verb!\grand{texte}!	&	\grand{texte}	&	grand
			\\\hline
				\verb!\Grand{texte}!	&	\Grand{texte}	&	tr�s grand
			\\\hline
				\verb!\GRAND{texte}!	&	\GRAND{texte}	&	tr�s tr�s grand
			\\\hline
				\verb!\enorme{texte}!	&	\enorme{texte}	&	�norme
			\\\hline
				\verb!\Enorme{texte}!	&	\Enorme{texte}	&	tr�s �norme
			\\\hline
				\verb!\miniCentre[2cm]! 
					\verb!{texte texte texte}!	&	\miniCentre[2cm]{texte texte texte}	&	Combo entre ``minipage'' et ``center''. Largeur par d�faut = taille de la ligne
			\\\hline
				\verb!\dotparagraph{titre}! 	&	\dotparagraph{titre}	&	identique � la commande ``paragraph'', avec un point devant.
			\\\hline
				\verb!\dotparagraph[*]{titre}! 	&	\dotparagraph[*]{titre}	&	idem avec une puce personnalis�e.
			\\\hline
		\end{tabular}

	
	\section{Raccourcis de mots}

		\noindent
		\begin{tabular}{|p{0.3\linewidth}|p{0.3\linewidth}|p{0.3\linewidth}|}
			\hline
				\textbf{Commandes}&\textbf{Rendus}&\textbf{Commentaires}
			\\\hline\hline
				\verb!\oeuvre!	&	\oeuvre 	&	avec le ``e dans l'o''
			\\\hline
				\verb!\oeuvres!	&	\oeuvres	&	avec le ``e dans l'o''
			\\\hline
				\verb!\cad!	&	\cad		&  C'est-�-dire
			\\\hline
		\end{tabular}

	\section{Mots Latins}
	
		\noindent
		\begin{tabular}{|p{0.3\linewidth}|p{0.3\linewidth}|p{0.3\linewidth}|}
			\hline
				\verb!\ie!	&	\ie		&  Id est \rac
			\\\hline
				\verb!\apriori!	&	\apriori	&  a priori
			\\\hline
				\verb!\Apriori!	&	\Apriori	&  a priori
			\\\hline
				\verb!\ps!	&	\ps		&  Post-Scriptum \rac
			\\\hline
		\end{tabular}


	\section{Maths}

		\begin{tabular}{|p{0.3\linewidth}|p{0.3\linewidth}|p{0.3\linewidth}|}
			\hline
				\textbf{Commandes}&\textbf{Rendus}&\textbf{Commentaires}
			\\\hline\hline
				\verb!$1\er$!	&	$1\er$	 	&	Premier
			\\\hline
				\verb!$1\ers$!	&	$1\ers$	 	&	Premiers
			\\\hline
				\verb!$1\ere$!	&	$1\ere$		&	Premi�re
			\\\hline
				\verb!$1\eres$!	&	$1\eres$	&	Premi�res
			\\\hline
				\verb!$2\nd$!	&	$2\nd$		&	Second
			\\\hline
				\verb!$2\nde$!	&	$2\nde$		&	Seconde
			\\\hline
				\verb!$n\eme$!	&	$n\eme$		&	n-i�me
			\\\hline
				\verb!$n\emes$!	&	$n\emes$	&	n-i�mes
			\\\hline
				\verb!$n\ieme$!	&	$n\ieme$		&	n-i�me
			\\\hline
				\verb!$n\iemes$!&	$n\iemes$	&	n-i�mes
			\\\hline
		\end{tabular}
	

	
	\section{Dessin\label{REF}}

		\begin{tabular}{|p{0.3\linewidth}|p{0.3\linewidth}|p{0.3\linewidth}|}
			\hline
				\textbf{Commandes}&\textbf{Rendus}&\textbf{Commentaires}
			\\\hline\hline
				\verb!\fig{REF}!	&\fig{REF} 	&	raccourci de \verb!(fig \ref{REF})!
			\\\hline
		\end{tabular}
	

	\section{Autres}

		\noindent
		\begin{tabular}{|p{0.3\linewidth}|p{0.3\linewidth}|p{0.3\linewidth}|}
			\hline
				\textbf{Commandes}&\textbf{Rendus}&\textbf{Commentaires}
			\\\hline\hline
				\verb!\qEmph{oh}!	&	\qEmph{oh} 	&	Citation entre guillemets (Probl�me d'encodage : il faut utiliser Babel)
			\\\hline
		\end{tabular}
\end{document}
