\documentclass[a4paper,10pt]{article}

	\usepackage{xcolor}
	\usepackage{listings}	\lstset{language=[LaTeX]TeX,basicstyle=\ttfamily,texcsstyle=*\color{blue},identifierstyle=\color{brown},commentstyle=\color{gray}\itshape,escapechar=!,moretexcs={}}
	\usepackage[utf8]{inputenc}
	\usepackage{hyperref}


	\usepackage{Raf_Notations_Maths}
	\usepackage[francais]{babel}
	\everymath{\displaystyle}


	\newcommand{\rac}{({\color{red}Raccourci})}
	\newcommand{\ren}{({\color{blue}Renommé})}

\begin{document}





	\begin{center}
		\hrule{\Large Notations Mathématiques ``de base''}\\\hrule
	\end{center}

	(Version du 27/04/13)

	\section{Packages requis}
	%-------------------------------------

	\begin{itemize}
		\item \href{http://www.ctan.org/pkg/ifthen}{\textbf{ifthen}} : Package permettant une compilation à choix multiple,
		\item \href{http://tug.ctan.org/tex-archive/macros/latex/contrib/xargs}{\textbf{xarg}} : Package permettant de créer des commandes à plusieurs arguments optionnels.
		\item \href{http://www.ams.org/publications/authors/tex/amsfonts}{\textbf{amsfonts}} : Package qui ajoute des polices d'écritures mathématiques.
		\item \href{http://www.ams.org/publications/authors/tex/amslatex}{\textbf{amsmath}} : Package qui ajoute des fonctions mathématiques non-standards.
		\item \href{http://www.ctan.org/pkg/mathrsfs}{\textbf{mathrsfs}} : Package qui rajoute des polices d'écritures mathématiques.
		\item \href{http://www.ctan.org/pkg/color}{\textbf{color}} : Package permettant de mettre en couleur du texte, des lignes, etc.
		\item \href{http://www.ctan.org/pkg/xspace}{\textbf{xspace}} : Package permettant de mettre des espaces après les commandes.
		\item \href{http://www.ctan.org/pkg/xstring}{\textbf{xstring}} : Package permettant travailler sur les chaînes de caractères (chercher/remplacer, etc.)
	\end{itemize}

	\section{Appel du package}
	%-------------------------------------

	Le package est appelé en début de document par la commande :
	\begin{verbatim}
\usepackage{Raf_Notations_Maths}
	\end{verbatim}

	Par défaut, ce package utilise un certain nombre de notations raccourcies, susceptibles de rentrer en conflit avec d'autre package (mais tellement plus rapide à taper !).
	De plus, certaines commandes ont été rebaptisée.
	Ces raccourcis et renommages seront cités (\rac\ ou \ren) dans les tableaux suivants.
	Pour ne pas créer ces raccourcis/renommage, il faut rentre l'option \verb!noRaccourci! à l'appel du package.

	\begin{verbatim}
usepackage[noRaccourci]{Raf_Notations_Maths}
	\end{verbatim}

	\section{Notations/annotations}
	%-------------------------------------------
	\noindent
	\begin{tabular}{|p{0.35\linewidth}|p{0.3\linewidth}|p{0.3\linewidth}|}
		\hline
			\textbf{Commandes}&\textbf{Rendus}&\textbf{Commentaires}
		\\\hline\hline
			\verb!\ssi!			&	\ssi			&	\rac
		\\\hline
			\verb!\indiceGauche{i}{R}!	&	\indiceGauche{i}{R}	&	Indice à gauche (même pour les grands objets)
		\\\hline
			\verb!\exposantGauche{i}{R}!	&	\exposantGauche{i}{R}	&	Exposant à gauche (même pour les grands objets)
		\\\hline
			\verb!\transposee{M}!		&	\transposee{M}		&	Symbole ``transposée''
		\\\hline
			\verb!\Cte!			&	\Cte			&	Constante \rac
		\\\hline
			\verb!\equivaut!		&	\equivaut		&	Équivaut
		\\\hline
			\verb!\LR!			&	\LR			&	Équivaut \rac
		\\\hline
			\verb!$2.1\E{-2}$!		&	 $2.1\E{-2}$		&	Puissances de 10\rac
		\\\hline
			\verb!$a=\mathbox{2}$!		&	 $a=\mathbox{2}$	&	équivaut à \verb!\fbox{}! en mode math.
		\\\hline
			\verb!\begin{align}! \verb!\boxedalign{resultat}{=2}! \verb!\end{align}!		&		&	Boite type \verb!\fbox{}!, pour une ligne de l'environnement \verb!align!.
		\\\hline
			\verb!\jComplexe!		&	\jComplexe		&	Nombre complexe \jComplexe\ pour l'élec ($\jComplexe^2=-1$).
		\\\hline
			\verb!\j!			&	\j			&	Nombre complexe \jComplexe\ pour l'élec \rac.
		\\\hline
			\verb!\iComplexe!		&	\iComplexe		&	Nombre complexe \iComplexe\ classique ($\iComplexe^2=-1$).
		\\\hline
            \verb!\valPi!       &      \valPi & Valeur de Pi, à quelques décimales prés
		\\\hline
            \verb!$\grasMath{a}+a$!       &      $\grasMath{a}+a$ & Mets en gras en mode math.
		\\\hline
            \verb!$\mathGras{a}+a$!       &      $\mathGras{a}+a$ & Idem que \verb!\grasMath!
		\\\hline
            \verb!$\inconnue{x}+1=2$!       &      $\inconnue{x}+1=2$ & Mise en évidence d'une inconnue
		\\\hline
	\end{tabular}

	\section{Fonctions}
	%-------------------------------------------

		\subsection{Commandes de bases}

			\noindent
			\begin{tabular}{|p{0.35\linewidth}|p{0.3\linewidth}|p{0.3\linewidth}|}
				\hline
					\textbf{Commandes}&\textbf{Rendus}&\textbf{Commentaires}
				\\\hline\hline
					\verb!\fonction{Fonction}{t}!	&	\fonction{Fonction}{t}		&	Mise en forme d'une fonction
				\\\hline
					\verb!\f{Fonction}{t}!	&	\f{Fonction}{t}		&	raccourci de \verb!\fonction! \rac
				\\\hline
					\verb!\f F{t}!		&	\f F{t}			&	Idem avec un nom de fonction à une seule lettre \rac
				\\\hline
			\end{tabular}


			
		\subsection{Dérivées et calculs différentiels}

			\noindent
			\begin{tabular}{|p{0.35\linewidth}|p{0.3\linewidth}|p{0.3\linewidth}|}
				\hline
					\textbf{Commandes}&\textbf{Rendus}&\textbf{Commentaires}
				\\\hline\hline
					\verb!\derivee{F}{t}!		&	\derivee {F}{t}			&	Dérivée
				\\\hline
					\verb!\deriv{F}{t}!		&	\deriv {F}{t}			&	Raccourci de \verb!\derivee!
				\\\hline	
					\verb!\deriveePartielle{F}{t}!		&	\deriveePartielle {F}{t}			&	Dérivée partielle
				\\\hline
					\verb!\derivP{F}{t}!		&	\derivP {F}{t}			&	Raccourci de \verb!\deriveePartielle!
				\\\hline
					\verb!\deriv[n]{F}{t}!		&	\deriv[n] {F}{t}			&	Dérivée $n^\text{ième}$
				\\\hline
					\verb!\deriv{\vec F}{t}[B]!		&	\deriv {\vec F}{t}[B]			&	Dérivée dans une base
				\\\hline
					\verb!\deriv{\vec F}{t}[B]!		&	\deriv {\vec F}{t}[B]			&	Dérivée dans une base
				\\\hline
					\verb!\deriv{\vec F}{t}[B]!		&	\deriv {\vec F}{t}[B]			&	Dérivée dans une base
				\\\hline
					\verb!\dDroit{a}!		&		\dDroit{a}		&	Notation différentielle avec un d-droit.
				\\\hline
					\verb!\dRond{a}!		&		\dRond{a}		&	Notation différentielle avec un d-rond.
				\\\hline
					\verb!\dx, \dy, \dz, \dtheta,! \verb!\dphi, \dr, \du, \dv! \verb!\dw, \dl, \dS, \dV!		&	\dx, \dy, \dz, \dtheta, \dphi, \du, \dv, \dw, \dr, \dl, \dS, \dV			&	Diverses variables d'intégration
				\\\hline
					\verb!\dx[1], \dy[2], \dz[3],! \verb!\dtheta[4], \dphi[5],! \verb!\dr[6], \du[7], \dv[8],! \verb!\dw[9], \dl[10],! \verb!\dS[11], \dV[12]!		&		\dx[1], \dy[2], \dz[3], \dtheta[4], \dphi[5], \dr[6], \du[7], \dv[8], \dw[9], \dl[10], \dS[11], \dV[12]		&	Diverses variables d'intégration avec indices
				\\\hline
			\end{tabular}


		\subsection{Fonctions usuelles}

			\noindent
			\begin{tabular}{|p{0.35\linewidth}|p{0.3\linewidth}|p{0.3\linewidth}|}
				\hline
					\textbf{Commandes}&\textbf{Rendus}&\textbf{Commentaires}
				\\\hline\hline
					\verb!\atan!		&	\atan			&	Arctangente \rac
				\\\hline
					\verb!\atan[\frac 1x]!		&	\atan[\frac 1x]		&	Arctangente avec paramètre \rac
				\\\hline
					\verb!\acos!		&	\acos			&	Arccosinus \rac
				\\\hline
					\verb!\acos[\frac 1x]!		&	\acos[\frac 1x]		&	Arccosinus avec paramètre \rac
				\\\hline
					\verb!\asin!		&	\asin			&	Arcsinus \rac
				\\\hline
					\verb!\asin[\frac 1x]!		&	\asin[\frac 1x]		&	Arcsinus avec paramètre \rac
				\\\hline
					\verb!\cotan!		&	\cotan			&	Cotangeante \rac
				\\\hline
					\verb!\cotan[\frac 1x]!		&	\cotan[\frac 1x]			&	Cotangeante \rac
				\\\hline
					\verb!\reel{x}!			&	\reel{x}	&	Partie réelle.
				\\\hline
					\verb!\Re{x}!			&	\Re{x}		&	Partie réelle \rac.
				\\\hline
					\verb!\imaginaire{x}!		&	\imaginaire{x}	&	Partie Imaginaire.
				\\\hline
					\verb!\Im{x}!			&	\Im{x}		&	Partie Imaginaire \rac.
				\\\hline
			\end{tabular}
					
		\subsection{Notations de Landau}

			\noindent
			\begin{tabular}{|p{0.35\linewidth}|p{0.3\linewidth}|p{0.3\linewidth}|}
				\hline
					\textbf{Commandes}&\textbf{Rendus}&\textbf{Commentaires}
				\\\hline\hline
					\verb!\bigO{n}!		&	\bigO{n}			&	Grand ``O''(notation de Landau)
				\\\hline
					\verb!\grandO{n}!	&	\grandO{n}			&	Autre commande pour \verb!\bigO!
				\\\hline
					\verb!\O{n}!		&	\O{n}				&	Raccourci de \verb!\bigO! \rac
				\\\hline
					\verb!\smallo{n}!	&	\smallo{n}			&	Petit ``o''(notation de Landau)
				\\\hline
					\verb!\petito{n}!	&	\petito{n}			&	Autre commande pour \verb!\smallo!
				\\\hline
					\verb!\o{n}!		&	\o{n}				&	Raccourci de \verb!\smallo! \rac
				\\\hline
			\end{tabular}
			
	\section{Équations}
		
			\noindent
			\begin{verbatim}
\begin{align}
		&\vecteur{V}=\vNul\\
	\LR	&\begin{alignSysteme}[rl]
			\vecteur{V}\cdot \vecteur{x}&=0	\\
			\vecteur{V}\cdot \vecteur{y}&=0
	   	\end{alignSysteme}
\end{align}
			\end{verbatim}
			
\begin{align}
		&\vecteur{V}=\vNul\\
	\LR	&\begin{alignSysteme}[rl]
			\vecteur{V}\cdot \vecteur{x}&=0	\\
			\vecteur{V}\cdot \vecteur{y}&=0
	   	\end{alignSysteme}
\end{align}
	\section{Ensembles}
	%----------------------------

	\noindent
	\begin{tabular}{|p{0.35\linewidth}|p{0.3\linewidth}|p{0.3\linewidth}|}
		\hline
			\textbf{Commandes}&\textbf{Rendus}&\textbf{Commentaires}
		\\\hline\hline
			\verb!\R!	&	\R		&	Nombre réel \rac
		\\\hline
			\verb!\couple{A}{B}!	&	\couple{A}{B}		&	Couple d'éléments \rac
		\\\hline
			\verb!\triplet{A}{B}{C}!	&	\triplet{A}{B}{C}			&	Triplet d'éléments
		\\\hline
			\verb!\quadruplet{A}{B}{C}{D}!		&	\quadruplet{A}{B}{C}{D}			&	Quadruplet d'éléments
		\\\hline
	\end{tabular}



	\section{Géométrie}
	%----------------------------

	\noindent
	\begin{tabular}{|p{0.35\linewidth}|p{0.3\linewidth}|p{0.3\linewidth}|}
		\hline
			\textbf{Commandes}&\textbf{Rendus}&\textbf{Commentaires}
		\\\hline\hline
			\verb!\segment{AB}!	&	\segment{AB}		&	Segment \rac
		\\\hline
			\verb!\droite{AB}!	&	\droite{AB}		&	droite \rac
		\\\hline
			\verb!\arc{AB}!		&	\arc{AB}			&	Arc \rac
		\\\hline
			\verb!\angle{ABC}!		&	\angle{ABC}		&	Angle (anciennement symbole ``angle'') \ren
		\\\hline
	\end{tabular}


	\section{Vecteurs}
	%-------------------------------------------

	De manière générale, tout ce qui concerne les vecteurs est précédé de la lettre ``\emph{v}'',
	tout ce qui concerne les base est précédé de la lettre ``\emph{b}''
	et tout ce qui concerne les repères est précédé de la lettre ``\emph{r}''.


	\subsection{Commandes de base}
	\noindent
	\begin{tabular}{|p{0.5\linewidth}|p{0.2\linewidth}|p{0.3\linewidth}|}
		\hline
			\verb!\vecteur{AB}!		&	\vecteur{AB}		&	Vecteur (commande de base)
		\\\hline
			\verb!\vecteur{e}[1]!		&	\vecteur{e}[1]		&	Vecteur avec indice
		\\\hline
			\verb!\vecteurIndice{e}{1}!	&	\vecteurIndice{e}{1}	&	Identique à \verb!\vecteur!, sauf que l'indice est obligatoire...
		\\\hline
			\verb!\vInd {e}{1}!		&	\vecteurIndice{e}{1}	&	
		\\\hline
			\verb!\vInd e1!			&	\vInd e1		&	exemple de simplification d'écriture.
		\\\hline
			\verb!\vecteurChamp{V}{x}!	&	\vecteurChamp{V}{x}	&	Vecteur champ
		\\\hline
			\verb!\vChamp{V}{x}!		&	\vChamp{V}{x}		&	Raccourci de \verb!\vecteurChamp!
		\\\hline
			\verb!\vChampOpt{V}[x]!		&	\vChampOpt{V}[x]	&	Identique à \verb!\vecteurChamp! avec le paramètre optionnel
		\\\hline
			\verb!\bipoint{A}{B}!		&	\bipoint{A}{B}		&	Bipoint
		\\\hline
			\verb!\vLie{A}{\vecteur{V}}!	&	\vLie{A}{\vecteur{V}}	&	Vecteur lié à un point 
		\\\hline
			\verb!\vGlissant{(\Delta)}{\vecteur{V}}!&\vGlissant{(\Delta)}{\vecteur{V}}	&	Vecteur Glissant
		\\\hline
	\end{tabular}


	\subsection{Espaces}
	\noindent
	\begin{tabular}{|p{0.5\linewidth}|p{0.2\linewidth}|p{0.3\linewidth}|}
		\hline
			\verb!\eAffine[n]!		&	\eAffine[n]		&	Espace affine de dimension $n$
		\\\hline
			\verb!\eAffine!		&	\eAffine		&	Espace affine de dimension $3$ (par défaut)
		\\\hline
			\verb!\eVectoriel[n]!		&	\eVectoriel[n]		&	Espace vectoriel de dimension $n$
		\\\hline
			\verb!\eVectoriel!		&	\eVectoriel		&	Espace vectoriel de dimension $3$ (par défaut)
		\\\hline
	\end{tabular}
	\subsection{Base/Repère}
	%---------------------------------
	\noindent
	\begin{tabular}{|p{0.5\linewidth}|p{0.2\linewidth}|p{0.3\linewidth}|}
		\hline
			\verb!\bB{}! ou 	\verb!\bB{1}! ou \verb!\bB2!	&	\bB{} ou \bB{1} ou \bB2		&	Symbole d'un base (avec ou sans indice) \rac
		\\\hline
			\verb!\base UVW!		&	\base UVW		&	Triplet représentant un base
		\\\hline
			\verb!\bxyz!			&	\bxyz			&	Base pré-fabriquée \rac
		\\\hline
			\verb!\buvw!			&	\buvw			&	Base pré-fabriquée \rac
		\\\hline
			\verb!\bCartesien!		&	\bCartesien			&	Base pré-fabriquée
		\\\hline
			\verb!\bCylindrique!	&	\bCylindrique			&	Base pré-fabriquée
		\\\hline
			\verb!\bSpherique!			&	\bSpherique			&	Base pré-fabriquée
		\\\hline
			\verb!\rR! ou 	\verb!\rR{1}!	&	\rR\ ou \rR{1}		&	Symbole d'un repère (avec ou sans indice) \rac
		\\\hline
			\verb!\repere Ouvw!		&	\repere Ouvw		&	Quadruplet représentant un base
		\\\hline
			\verb!\rOxyz!			&	\rOxyz			&	Base pré-fabriquée
		\\\hline
			\verb!\rOuvw!			&	\rOuvw			&	Base pré-fabriquée
		\\\hline
	\end{tabular}



	\subsection{Représentation}
	\noindent
	\begin{tabular}{|p{0.5\linewidth}|p{0.2\linewidth}|p{0.3\linewidth}|}
		\hline
			\verb!\vColonne{X\\Y\\Z}{B}!	&	\vColonne{X\\Y\\Z}{B}		&	Vecteur colonne (avec base !)
		\\\hline
			\verb!\vColonne{X\\Y\\Z}{}!	&	\vColonne{X\\Y\\Z}{}		&	Vecteur colonne sans base (mal !)
		\\\hline
			\verb!\vColonne{X;Y;Z}{B}!	&	\vColonne{X;Y;Z}{B}		&	Vecteur colonne avec remplacement des points virgules (par défaut) par des saut de ligne
		\\\hline
			\verb!\vColonne{X|Y|Z}{B}[|]!	&	\vColonne{X|Y|Z}{B}[|]		&	Idem que précédemment, en choisissant le séparateur de lignes
		\\\hline
			\verb!\vColonne{X;Y;Z}{B}[]!	&	\vColonne{X;Y;Z}{B}[]		&	Un séparateur ``vide'' permet d'échapper les points virgules.
		\\\hline
	\end{tabular}

	\subsection{Opérateurs}
	\noindent
	\begin{tabular}{|p{0.5\linewidth}|p{0.2\linewidth}|p{0.3\linewidth}|}
		\hline
			\verb!\norme{X}!	&	\norme{X}	&	Norme
		\\\hline
			\verb!\abs{X}!		&	\abs{X}		&	Valeur absolue / module \rac
		\\\hline
			\verb!\prodMixte{U}{V}{W}!	&	\prodMixte{U}{V}{W}		&	Produit mixte
		\\\hline
			\verb!\doubleProdVect{U}{V}{W}!		&	\doubleProdVect{U}{V}{W}	&	Double produit vectoriel
		\\\hline
			\verb!\dbPVect UVW!		&	\dbPVect UVW	&	Raccourci de \verb!\doubleProdVect!
		\\\hline
			\verb!\scalaire!			&	\scalaire		&	Opérateur produit scalaire
		\\\hline
			\verb!\scal!				&	\scal			&	Raccourci de \verb!\scalaire! \rac
		\\\hline
			\verb!\vectoriel!			&	\vectoriel		&	Opérateur produit vectoriel
		\\\hline
			\verb!\vect!				&	\vect			&	Raccourci de \verb!\vectoriel! \rac
		\\\hline
	\end{tabular}

	\subsection{Vecteurs pré-fariqués}
	\noindent
	\begin{tabular}{|p{0.5\linewidth}|p{0.2\linewidth}|p{0.3\linewidth}|}
		\hline
			\verb!\vNul!			&	\vNul	&	vecteur nul
		\\\hline
			\verb!\vCte!			&	\vCte	&	vecteur-constante
		\\\hline
			\verb!\ve{1}! ou \verb!\ve1!	&	\ve1	&	vecteur \vecteur e, avec indice \rac
		\\\hline
			\verb!\vex!			&	\vex	&	Identique à \verb!\ve{x}! \rac
		\\\hline
			\verb!\vey!			&	\vey	&	Identique à \verb!\ve{y}! \rac
		\\\hline
			\verb!\vez!			&	\vez	&	Identique à \verb!\ve{z}! \rac
		\\\hline
	\end{tabular}
	\begin{tabular}{|p{0.5\linewidth}|p{0.2\linewidth}|p{0.3\linewidth}|}
		\hline
			\verb!\vx{}! ou \verb!\vx1!	&	\vx{} ou \vx1	&	Vecteur \vx{} avec indice \rac
		\\\hline
			\verb!\vy{}! ou \verb!\vy2!	&	\vy{} ou \vy2	&	Vecteur \vy{} avec indice \rac
		\\\hline
			\verb!\vz{}! ou \verb!\vz3!	&	\vz{} ou \vz3	&	Vecteur \vz{} avec indice \rac
		\\\hline
	\end{tabular}
	\begin{tabular}{|p{0.5\linewidth}|p{0.2\linewidth}|p{0.3\linewidth}|}
		\hline
			\verb!\vn! ou	\verb!\vn[1]! 	&	\vn\ ou \vn[1]	&	Vecteur \vn\ avec ou sans indice. \rac
		\\\hline
			\verb!\ver! ou 	\verb!\ver[x]!	&	\ver\ ou \ver[x]&	Vecteur \ver\ avec ou sans paramètre. \rac
		\\\hline
			\verb!\vetheta! ou \verb!\vetheta[y]!	&	\vetheta\ ou \vetheta[y]	&	Vecteur \vetheta\ avec ou sans paramètre.
		\\\hline
	\end{tabular}
	\begin{tabular}{|p{0.5\linewidth}|p{0.2\linewidth}|p{0.3\linewidth}|}
		\hline
			\verb!vu! ou 	\verb!vu[1]!	&	\vu\ ou \vu[1]	&	Vecteur \vu\ avec ou sans indice. \rac
		\\\hline
			\verb!vU! ou 	\verb!vU[1]!	&	\vU\ ou \vU[1]	&	Vecteur \vU\ avec ou sans indice. \rac
		\\\hline
			\verb!\ux,\uy,\uz!		&	\ux,\uz,\uz	&	Coordonnées de \vu. \rac
		\\\hline\hline
			\verb!vv! ou 	\verb!vv[1]!	&	\vv\ ou \vv[1]	&	Vecteur \vv\ avec ou sans indice. \rac
		\\\hline
			\verb!vV! ou 	\verb!vV[1]!	&	\vV\ ou \vV[1]	&	Vecteur \vV\ avec ou sans indice. \rac
		\\\hline
	\end{tabular}
	\begin{tabular}{|p{0.5\linewidth}|p{0.2\linewidth}|p{0.3\linewidth}|}
		\hline
			\verb!vw! ou 	\verb!vw[1]!	&	\vw\ ou \vw[1]	&	Vecteur \vw\ avec ou sans indice. \rac
		\\\hline
			\verb!vW! ou 	\verb!vW[1]!	&	\vW\ ou \vW[1]	&	Vecteur \vW\ avec ou sans indice. \rac
		\\\hline
			\verb!\wx,\wy,\wz! 		&	\wx,\wy,\wz	&	Coordonnées de \vw. \rac
		\\\hline
	\end{tabular}
	\begin{tabular}{|p{0.5\linewidth}|p{0.2\linewidth}|p{0.3\linewidth}|}
		\hline
			\verb!\vOM! ou  \verb!\vOM[t]!	&	\vOM\ ou  \vOM[t]	&	Vecteur ou vecteur champ \vOM.
		\\\hline
			\verb!\Mx,\My,\Mz!		&	\Mx,\My,\Mz	&	Coordonnées de \vOM.\rac
		\\\hline
			\verb!\vOP! ou 	\verb!\vOP[t]!	&	\vOP\ ou  \vOP[t]	&	Vecteur ou vecteur champ \vOP.
		\\\hline
			\verb!\vAB! ou 	\verb!\vAB[t]!	&	\vAB\ ou  \vAB[t]	&	Vecteur ou vecteur champ \vAB.
		\\\hline
			\verb!\vBA! ou 	\verb!\vBA[t]!	&	\vBA\ ou  \vBA[t]	&	Vecteur ou vecteur champ \vBA.
		\\\hline
			\verb!\vOA! ou 	\verb!\vOA[t]!	&	\vOA\ ou  \vOA[t]	&	Vecteur ou vecteur champ \vOA.
		\\\hline
			\verb!\vOB! ou 	\verb!\vOB[t]!	&	\vOB\ ou  \vOB[t]	&	Vecteur ou vecteur champ \vOB.
		\\\hline
	\end{tabular}
	\begin{tabular}{|p{0.5\linewidth}|p{0.2\linewidth}|p{0.3\linewidth}|}
		\hline
			\verb!\vi{}! ou \verb!\vi1!	&	\vi\ ou  \vi1		&	Vecteur \vi{} avec indice.\rac
		\\\hline
			\verb!\vj{}! ou \verb!\vj2!	&	\vj\ ou  \vj2		&	Vecteur \vj{} avec indice.\rac
		\\\hline
			\verb!\vk{}! ou \verb!\vk3!	&	\vk\ ou  \vk3		&	Vecteur \vk{} avec indice.\rac
		\\\hline
	\end{tabular}





\end{document}
