\documentclass[a4paper,10pt]{article}

	\usepackage{xcolor}
	\usepackage{listings}	\lstset{language=[LaTeX]TeX,basicstyle=\ttfamily,texcsstyle=*\color{blue},identifierstyle=\color{brown},commentstyle=\color{gray}\itshape,escapechar=!,moretexcs={}}
	\usepackage[latin1]{inputenc} 
	\usepackage{hyperref}

	\usepackage[]{Raf_Notations_Torseurs}

	\everymath{\displaystyle}


	\newcommand{\rac}{({\color{red}Raccourci})}
	\newcommand{\ren}{({\color{blue}Renomm�})}

\begin{document}

	\begin{center}
		\hrule{\Large Notations de torseurs ``de base''}\\\hrule
	\end{center}

	(Version du 01/06/16)

	Ce package sert de \textbf{base} pour les torseurs qui seront utilis�s dans les packages \verb!Raf_Notations_Cinematique! et \verb!Raf_Notations_Actions_Meca! (et autres...)

	\section{Packages requis}
	%-------------------------------------

	\begin{itemize}
		\item \href{http://www.ctan.org/pkg/ifthen}{\textbf{ifthen}} : Package permettant une compilation � choix multiple,
		\item \href{http://tug.ctan.org/tex-archive/macros/latex/contrib/xargs}{\textbf{xarg}} : Package permettant de cr�er des commandes � plusieurs arguments optionnels.
		\item \href{http://www.ctan.org/pkg/mathrsfs}{\textbf{mathrsfs}} : Package qui rajoute des polices d'�critures math�matiques.
		\item \href{http://enseignement.allais.eu/page-latex}{\textbf{Raf\_Notations\_Maths}} : Package de mise en forme d'objets math�matiques
	\end{itemize}

	\section{Appel du package}
	%-------------------------------------
	Le package est appel� en d�but de document par la commande :
	\begin{verbatim}
\usepackage{Raf_Notations_Torseurs}
	\end{verbatim}

	Par d�faut, ce package utilise un certain nombre de notations raccourcies, susceptibles de rentrer en conflit avec d'autres packages (mais tellement plus rapide � taper !).
	De plus, certaines commandes ont �t� rebaptis�e.
	Ces raccourcis et renommages seront cit�s (\rac\ ou \ren) dans les tableaux suivants.
	Pour ne pas cr�er ces raccourcis/renommage, il faut rentre l'option \verb!noRaccourci! � l'appel du package.

	\begin{verbatim}
usepackage[noRaccourci]{Raf_Notations_Torseurs}
	\end{verbatim}

	\section{�criture g�n�rale}
	%-------------------------------------------
	\noindent
	\begin{tabular}{|p{0.35\linewidth}|p{0.3\linewidth}|p{0.3\linewidth}|}
		\hline
			\textbf{Commandes}&\textbf{Rendus}&\textbf{Commentaires}
		\\\hline\hline
			\verb!\TCallig!			&	\TCallig		&	Symbole ``T'' de torseur calligraphi� du torseur
		\\\hline
			\verb!\torseur{X}!		&	\torseur{X}		&	Objet torseur
		\\\hline
			\verb!\tT!			&	\tT			&	Torseur \TCallig \rac
		\\\hline
			\verb!\tC!			&	\tC			&	Torseur couple \rac
		\\\hline
			\verb!\tNul!			&	\tNul			&	Torseur nul
		\\\hline
	\end{tabular}

	\section{�l�ments de r�duction}
	%-------------------------------------------
	\noindent
	\begin{tabular}{|p{0.35\linewidth}|p{0.3\linewidth}|p{0.3\linewidth}|}
		\hline
			\textbf{Commandes}&\textbf{Rendus}&\textbf{Commentaires}
		\\\hline\hline
			\verb!\Mom!			&	\Mom			&	Symbole ``M'' de base de moment
		\\\hline
			\verb!\Res!			&	\Res			&	Symbole ``R'' de base d'une r�sultante
		\\\hline
			\verb!\resultante{\TCallig}!		&	\resultante{\TCallig}		&	R�sultante (de \tT)
		\\\hline
			\verb!\resultante{1}[2]!	&	\resultante{1}[2]		&	R�sultante d'un objet par rapport � un autre
		\\\hline
			\verb!\moment{A}{\TCallig}!		&	\moment{A}{\TCallig}		&	Moment de \tT au point $A$
		\\\hline
			\verb!\moment{A}{1}[2]!		&	\moment{A}{1}[2]		&	Moment de $(1/2)$ au point $A$
		\\\hline
			\verb!\torseurLigne{A}!\qquad\verb!{\vecteur{X}}!\qquad\verb!{\vecteur{Y}}!		&	\torseurLigne{A}{\vecteur{X}}{\vecteur{Y}}	&	Torseur ligne
		\\\hline
			\verb!\tLigne{A}!\qquad\verb!{\vecteur{X}}!\qquad\verb!{\vecteur{Y}}!		&	\torseurLigne{A}{\vecteur{X}}{\vecteur{Y}}	&	Raccourci de \verb!\torseurLigne!
		\\\hline
			\verb!\torseurColonne{A}!\qquad\verb!{X\\Y\\Z}!\qquad\verb!{L\\M\\N}{R}!		&	\torseurColonne{A}{X\\Y\\Z}{L\\M\\N}{R}	&	Torseur Colonne
		\\\hline
			\verb!\tColonne{A}!\qquad\verb!{X\\Y\\Z}!\qquad\verb!{L\\M\\N}{R}!		&	\tColonne{A}{X\\Y\\Z}{L\\M\\N}{R}	&	Raccourci de \verb!\torseurColonne!
		\\\hline
	\end{tabular}

	\section{Op�rateurs}
	%-------------------------------------------
	\noindent
	\begin{tabular}{|p{0.35\linewidth}|p{0.3\linewidth}|p{0.3\linewidth}|}
		\hline
			\textbf{Commandes}&\textbf{Rendus}&\textbf{Commentaires}
		\\\hline\hline
			\verb!\automoment{\TCallig}!		&	\automoment{\TCallig}	&	Automoment
		\\\hline
			\verb!\automoment{1}[2]!	&	\automoment{1}[2]	&	Automoment
		\\\hline
			\verb!\axeCentral{\TCallig}!	&	\axeCentral{\TCallig}	&	Axe Central du torseur
		\\\hline
			\verb!\axeCentral{1}[2]!	&	\axeCentral{1}[2]	&	Axe Central du torseur
		\\\hline
			\verb!\comoment{\TCallig_1}! \verb!{\TCallig_2}!	&	\comoment{\TCallig_1}{\TCallig_2}	&	Comoment de deux torseurs
		\\\hline
			\verb!\devComoment{A}! \verb!{\TCallig_1}! \verb!{\TCallig_2}!	&	\devComoment{A}{\TCallig_1}{\TCallig_2}	&	Comoment en d�veloppant avec les �l�ments de r�duction
		\\\hline
	\end{tabular}

\end{document}
