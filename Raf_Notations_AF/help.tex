\documentclass[a4paper,12pt]{article}

	\usepackage{xcolor}
	\usepackage{listings}	\lstset{language=[LaTeX]TeX,basicstyle=\ttfamily,texcsstyle=*\color{blue},identifierstyle=\color{brown},commentstyle=\color{gray}\itshape,escapechar=!,moretexcs={}}
	\usepackage[latin1]{inputenc} 

	\usepackage[raccourcis]{Raf_Notations_AF}



\begin{document}

	\begin{center}
		\hrule{\Large Notations Analyse Fonctionnelle}\\\hrule
	\end{center}

	\section{Inclusion dans le fichier .tex :}

		\verb!\usepackage[raccourcis]{Raf_Notation_AF}!

		\textbf{Note :} Oui, je sais... l'option \verb!raccourci! est un peu cr�tine dans le sens o� il n'y a que des raccourcis dans ce package.
								C'est simplement pour avoir une uniformit� dans tous mes packages.

	\section{Commandes}

		\noindent
		\begin{tabular}{|p{0.2\linewidth}|p{0.3\linewidth}|p{0.4\linewidth}|}
			\hline
				\textbf{Commandes}&\textbf{Rendus}&\textbf{Commentaires}
			\\\hline\hline
				\verb!\MO!	&	\MO	&	``Mati�re d'\oe uvre'' (raccourci)
			\\\hline
				\verb!\MOs!	&	\MOs	&	``Mati�res d'\oe uvre'' au pluriel (raccourci)
			\\\hline
				\verb!\VA!	&	\VA	&	``Valeur ajout�e'' (raccourci)
			\\\hline
				\verb!\VAs!	&	\VAs	&	``Mati�re d'\oe uvre'' au pluriel (raccourci)
			\\\hline
		\end{tabular}

	\section{Cha�ne Fonctionnelle}

	
\begin{verbatim}
\begin{chaineFonctionnelle}
     \fontsize{1}{1}\selectfont
     \CFAcquerir{Capteur1\\Capteur2}
     \CFTraiter{PC\\Automate}
     \CFAlimenter{Prise\\Transformateur}
     \CFDistribuer{Hacheur}
     \CFConvertir{MCC}
     \CFTransmettre{R�ducteur}
     \CFAgir{Effecteur}

     \CFEnergieEntree{Puissance\\�lectrique}
     \CFInfosEntree{Consignes\\utilisateur}
     \CFInfosSortie{Rapport\\utilisateur}
     \CFOrdres{Ordres}
     \CFAcquisition{Donn�es\\r�colt�es}

     \CFMO{Mon produit}
     \CFMOVA{Mon produit\\modifi�}
\end{chaineFonctionnelle}
\end{verbatim}

	
\begin{chaineFonctionnelle}
\fontsize{1}{1}\selectfont
 	\CFAcquerir{Capteur1\\Capteur2}
 	\CFTraiter{PC\\Automate}
	\CFAlimenter{Prise\\Transformateur}
	\CFDistribuer{Hacheur}
	\CFConvertir{MCC}
	\CFTransmettre{R�ducteur}
	\CFAgir{Effecteur}

	\CFEnergieEntree{Puissance\\�lectrique}
	\CFInfosEntree{Consignes\\utilisateur}
	\CFInfosSortie{Rapport\\utilisateur}
	\CFOrdres{Ordres}
	\CFAcquisition{Donn�es\\r�colt�es}

	\CFMO{Mon produit}
	\CFMOVA{Mon produit\\modifi�}
\end{chaineFonctionnelle}



\end{document}
